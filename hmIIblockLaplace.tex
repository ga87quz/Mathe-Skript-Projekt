
%%%%  ! T  E X root=hm2ln.tex
% !TEX root=MA9603.WZW.tex
% !TeX program = pdflatex
% !TEX spellcheck = de_DE

%===============================================================================

\section{Laplacetransformation}
\zbox{
{\bf Ziele}:
\begin{itemize}
\item Wissen: Was ist die Laplace Transformation \& wozu ist die gut?
\item Sie kennen die Regeln der Laplace-Transformation \& k\"onnen diese anwenden
\item Sie k\"onnen eine gew\"ohnliche Differentialgleichung mittels Laplacetransformation l\"osen
\item Sie k\"onnen die R\"ucktransformation ausf\"uhren, 
und kennen die wichtigsten Funktionen
\end{itemize}}
%===============================================================================
\par\medskip
L\"osen Sie zun\"achst die Basisaufgaben 
\ref{lpaplaceAuf1}, \ref{lpaplaceAuf3},  \ref{lpaplaceAuf11},  \ref{lpaplaceAuf12}, 
um sich dann die Vertiefungsaufgaben \ref{lpaplaceAuf2},  \ref{lpaplaceAuf4},  \ref{lpaplaceAuf8}, \ref{lpaplaceAuf9}. 
anzusehen. Eine Knobelaufgabe ist \ref{lpaplaceAuf10}.


\subsection{Laplace Transformation - Definitionen}
Es gibt viele M\"oglichkeiten, Funktionen zu transformieren. Ziel ist immer eine 
alternative Darstellung, die f\"ur eine gegebene Fragestellung besonders geeignet ist.
Beispiele:\\
{\bf Fourier-Transformation:} Statt $f(t)$ betrachte 
$c_i = \int_0^T e^{i\, k\, t} f(t)\, d\tau$, $i\in\Z$.\\
Diese Transformation analysiert eine periodische Funktion (mit Periode $T$) 
auf ihre Frequenzanteile (Grundton, Oberton, \ldots).\par\medskip
{\bf Radon-Transformation:} Hier hat man die Computer-Tomographie im Hinterkopf: F\"ur eine Funktion 
$f(x,y)$ betrachte alle Gerade $y=a\,x+b$, und integriere die Funktion \"uber diese Geraden, 
$g(a,b) = \int_{-\infty}^\infty f(x,a\,x+b)\, dx$. Hier entspricht $f(x,y)$ der Gewebedichte
eines Patienten, und $g(a,b)$ der gemessenen D\"ampfung von R\"ontgenstrahlen beim 
Durchgang durch den Patienten auf der gegebenen Geraden. Ziel ist dann, aus $g(a,b)$ wieder $f(x,y)$ 
zu rekonstruieren.\par\medskip

Die Laplace-Transformation zielt auf gew\"ohnliche Differentialgleichungen. 
\begin{sdefi}
\index{Laplacetransformation}
Sei $f:\{x\geq 0\}\rightarrow \R$ gegeben, wobei ein $\sigma\geq 0$, $C>0$ existiert, sodass
$$ |f(t)|\leq C e^{\sigma t}.$$
Dann definiere f\"ur $s>\sigma$
$$ F(s) = {\cal L}\{f(t)\}(s) = \int_0^\infty e^{-s\, t}\,f(t)\, dt.$$
Falls die Funktion $f$ Vektorwertig ist, $f:\{x\geq 0\}\rightarrow \R^n$, $f(t)=(f_1(t),\ldots,f_n(t))^T$, 
so ist die Laplace-Transformation komponentenweise definiert, 
$$ F(s) = {\cal L}\{f(t)\}(s) = 
\bigg(
\int_0^\infty e^{-s\, t}\,f_1(t)\, dt,\ldots,
\int_0^\infty e^{-s\, t}\,f_n(t)\, dt
 \bigg)^T.$$
\end{sdefi}\par\medskip

Wir haben schon gesehen, dass die L\"osungen von gew\"ohnlichen Differentialgleichungen 
auf das Engste mit Exponentialfunktionen verkn\"upft sind. Daher ist die Laplacetransformation 
in diesem Fall besonders n\"utzlich: Die Laplacetransformierte stellt die 
Funktion als eine unendliche Linearkombination aus Exponentialfunktionen dar. 
Die folgende Eigenschaft 
beweisen wir nicht (dieses Resultat ist etwas zu 
technisch im Beweis). 

\begin{ssatz}
Die Laplace-Transformation ist bijektiv, d.h. die inverse Transformation 
${\cal L}^{-1}\{F(s)\}(t) = f(t)$ existiert und ist eindeutig (wenn die Funktion $F(s)$ vern\"unftige 
Eigenschaften besitzt, was bei uns immer gegeben sein wird).
\end{ssatz}

Wir Transformieren ein paar Funktionen, um ein Gef\"uhl f\"ur die Laplace-Transformation 
zu erhalten.
\begin{bspX}
\underline{$f(t) = 1$}:\\
$$ F(s) = \int_0^\infty e^{-s\ t} \, 1\, dt = \frac 1 s.$$
\underline{$f(t) = t$}; Wir nutzen partielle Integration:
\begin{eqnarray*} 
F(s) 
&=& \int_0^\infty e^{-s\ t} \, t\, dt 
= \int_0^\infty \frac {-1}{s}\,\,\left(\frac d {dt}e^{-s\ t}\right) \, t\, dt \\
&=& \frac {-1}{s}\,\,\left(\frac d {dt}e^{-s\ t}\right) \, t\bigg|_{t=0}^\infty 
- \int_0^\infty \frac {-1}{s}\,\,e^{-s\ t} dt \\
&=& \frac 1 {s^2}.
\end{eqnarray*}
\underline{$f(t)=e^{a\, t}$ f\"ur $a\in\R$}:\\
$$F(s) 
= \int_0^\infty e^{-s \, t}\, e^{a\, t}\, dt 
= \int_0^\infty e^{-(s-a) \, t}\, dt 
= \frac {1}{s-a}.
$$
\end{bspX}

Auf diese Weise k\"onnen wir eine Tabelle erstellen, die die 
Laplacetransformierten einiger Funktionen angibt. Diese Tabelle
werden wir nutzen, um die Laplacetransformation r\"uckg\"angig
zu machen. Daher ist in der Tabelle~\ref{LaplaceTable} 
in der linken Spalte die Laplace-Transformierte, und in der rechten 
Spalte die Ursprungsfunktion
angegeben.

\begin{table}[htb]
\fpbox{
\begin{minipage}{\textwidth}
\begin{center}
\begin{tabular}{l|l}
${\cal L}\{f(t)\}(s)=F(s)$, Laplacetransformierte & $f(t)$, Ursprungsfunktion\\
\hline
$F(s)=1/s$ & $f(t)=1$\\
$F(s)=n!/s^{n+1}$ & $f(t)=t^n$\qquad\qquad f\"ur $n\in\N_0$\\
$F(s)=1/(s+a)$ & $f(t)=e^{-a t}$\\
$F(s)=n!/(s+a)^{n+1}$ & $f(t)=t^n\,e^{-a t}$\qquad f\"ur $n\in\N_0$\\
$F(s)=\omega/(s^2+\omega^2)$ & $f(t)=\sin(\omega t)$\\
$F(s)=s/(s^2+\omega^2)$ & $f(t)= \cos(\omega t)$\\
\end{tabular}
\end{center}
\caption{Laplacetranformierte Funktionen (links) 
und die urspr\"ungliche Funktion (rechts).}
\label{LaplaceTable}
\end{minipage}
}
\end{table}

\begin{bspX}
Finde die R\"ucktransformierte $f(t)$ von 
$${\cal L}\{f(t)\}(s) = \frac {2}{s^2-1}
$$
Schritt 1: Partialbruchzerlegung.\\
$$\frac{2}{s^2-1} = \frac 1{s-1} - \frac 1 {s+1}.$$
Schritt 2: Nutze die Tabelle. Da die inverse Laplacetransformation linear ist, finden wir 
$$f(t) = e^{t}-e^{-t}.$$
\end{bspX}

%===============================================================================
\begin{auf}\chb\label{lpaplaceAuf1}
\input{../../Aufgabensammlung/hmIIlaplaceA1.tex}
\end{auf}
%===============================================================================


%===============================================================================
\begin{auf}\chc\label{lpaplaceAuf2}
\input{../../Aufgabensammlung/hmIIlaplaceA2.tex}
\end{auf}
%===============================================================================


%===============================================================================
\begin{auf}\chc\label{lpaplaceAuf3}
\input{../../Aufgabensammlung/hmIIlaplaceA3.tex}
\end{auf}
%===============================================================================

%AUFGABE: Bestimmen Sie die Laplace-Transformierte 
%f\"ur (a) $f(t)=a$ wobei $a\in\R$, 
%(b) $f(t)=t^2$.\\
%LOSUNG\\
%(a) $F(s) = \int_0^\infty e^{-st}\, a\, dt = \frac a s$\\
%(b) \begin{eqnarray*}
%F(s) & = & \int_0^\infty e^{-st}\, t^2\, dt 
%= \frac {-1}{s}\, \int_0^\infty \left(\frac {d}{dt}\, e^{-st}\right)\, %t^2\, dt \\
%%&=& \frac {2}{s}\, \int_0^\infty \frac e^{-st}\, t\, dt = \frac 2 {s^3}.
%\end{eqnarray*}
%
%AUFGABE: Zeigen sie, dass die Laplace-Transformierte 
%von $f_n(t)=t^n$ mit $n\in\N$ durch $F_n(s) = n!/s^{n+1}$ gegeben ist.\\
%LOESUNG:\\
%\begin{eqnarray*}
%F_n(s) & = & \int_0^\infty e^{-st}\, t^n\, dt 
%= \frac {-1}{s}\, \int_0^\infty \left(\frac {d}{dt}\,e^{-st}\,\right) %t^n\, dt 
%= \frac {n}{s}\, \int_0^\infty  e^{-st}\, t^{n-1}\, dt 
%= \frac n s F_{n-1}(s).
%\end{eqnarray*}
%Daher, 
%$$F_n(s) 
%= \frac{n}{s}\, F_{n-1}(s) 
%= \frac{n(n-1)}{s^2}\, F_{n-2}(s) 
%= \frac{n(n-1)(n-2)}{s^3}\, F_{n-3}(s) 
%=\ldots=\frac {n!}{s^{n-1}}\, F_1(s) = \frac{n!}{s^{n+1}}.$$
%

%AUFGABE: R\"ucktransformation.
%Bestimmen Sie $f(t)$, falls 
%${\cal L}\{f(t)\}(s) = (s^2-6s+12)/(s^3-7s^2+12 s)$\\
%LOESUNG:\\
%Partialbruchzerlegung. Nullstellen von $s^3-7s^2+12 s= s(s^2-7s+12)$ 
%sind $0$, $3$, $4$.
%$$
%\frac{s^2-6s+12}{s^3-7s^2+12 s}
%=
%\frac A s + \frac B {s-3} + \frac C {s-4}
%= \frac{A(s^2-7s+12)+Bs(s-4)+Cs(s-3)}{s^3-7s^2+12 s}
%$$
%F\"ur $A$, $B$, $C$ finden wir also das lineare Gleichungssystem
%$$
%\left(\begin{array}{ccc}
%1     & 1  & 1\\
%-7 & -4 & -3\\
%12 & 0 & 0
%\end{array}\right)\, 
%\left(\begin{array}{c}
%A\\B\\C
%\end{array}\right)
%=
%\left(\begin{array}{c}
%1\\-6\\12
%\end{array}\right).
%$$
%mit der L\"osung $A=1$, $B=-1$, $C=1$, und also
%$$
%\frac{s^2-6s+12}{s^3-7s^2+12 s}
%=
%\frac 1 {s} -  \frac 1 {s-3} + \frac 1 {s-4}.$$
%Mit Hilfe der magischen Tabelle finden wir ${\cal L}\{e^{- at}\%(s)=1/(s+a)$,
%und also
%$$ f(t) = 1 - e^{3\, t} + e^{4 t}.$$
%% (s-3)*(s-4)=s^2-7 s+12 
%% s(s-4) = s^2-4s
%% s(s-3) = s^2-3s
%% (s^2-7s+12)-(s^2-4s)+(s^2-3s) = s^2-6 s +12
%%
%% %
%LOESUNGSHAPPEN: 
%Partialbruchzerlegung ergibt: ${\cal L}\{f(t)\}(s)=1/s\,\,-\,\,1/(s-3)\,\,%+\,\,1/(s-4).$ 
%Tabelle ergibt dann $ f(t) = 1 - e^{3\, t} + e^{4 t}$.\\
%AUFGABE ENDE.



\subsection{Einige Eigenschaften der Laplacetransformation}
Bevor wir uns an die Regeln machen, ein kurzer Exkurs/Vorgriff auf 2-dimensionale 
Integrale.



\begin{figure}[htb]
\begin{center}
\includegraphics[width=10cm]{../figures/vertauschIntReihenfolge-eps-converted-to.pdf}
%{\huge TODO}
\end{center}
\caption{Vertauschen der Integrationsreihenfolge: 
$\{(x,y)\, |\, 0\leq x <\infty,\,\,\, 0\leq y\leq x\}$ (links) ist gleich
$\{(x,y)\, |\, 0\leq y <\infty,\,\,\, y\leq x < \infty\}$ (rechts).}
\label{vertauschFig}
\end{figure}

Sei $f(x,y)$ ein Integrand, der von zwei Variablen abh\"angt. 
Wir m\"ochten das Integral $\int_0^\infty\int_0^x f(x,y)\, dy\, dx$ ausrechnen. 
Wir haben also zwei geschachtelte Integrale: bei gegebenen $y$ integrieren wir erst 
\"uber $y$, und zwar von $0$ nach $x$. Das ergibt $g(y) = \int_0^x f(x,y)\, dy$, 
d.h., eine Funktion die nur von $x$ abh\"angt. Diese integrieren wir dann von $0$ bis $\infty$. \\
Es ist in den F\"allen, die wir uns unten ansehen, geschickter die Integrationsreihenfolge
zu vertauschen. Beachte dabei, dass die folgenden Mengen gleich sind (siehe 
Fig.~\ref{vertauschFig}):
$$ \{(x,y)\, |\, 0\leq x <\infty,\,\,\, 0\leq y\leq x\} 
=  \{(x,y)\, |\, 0\leq y <\infty,\,\,\, y\leq x < \infty\}.
$$
Daher folgt 
$$ \int_0^\infty\int_0^x f(x,y)\, dy\, dx 
= \int_0^\infty\int_y^ \infty f(x,y)\, dx\, dy.$$
\par\bigskip\bigskip

\fpbox{
\begin{minipage}{\textwidth}
\paragraph{Linearit\"at}
$ 
{\cal L}\{a\,f_1(t)+b\, f_2(t)\}(s)
=
a\,{\cal L}\{f_1(t)\}(s)
+
b\,{\cal L}\{f_2(t)\}(s).$
\end{minipage}
}\par\medskip
 Analog  f\"ur die 
R\"ucktransformation: Wenn $F_1(s)={\cal L}\{f_1(t)\}(s)$,
 $F_2(s)={\cal L}\{f_2(t)\}(s)$, so gilt 
 $${\cal L}^{-1}\{a F_1(s)+b F_2(s)\}(t) = a f_1(t)+bf_2(t).$$
(Folgt aus Linearit\"at des Integrals). 
 

\fpbox{
\begin{minipage}{\textwidth}
\paragraph{Verschiebungssatz}
Es ist ${\cal L}\{e^{a\,t}f(t)\}(s)={\cal L}\{f(t)\}(s+a)$.
\end{minipage}
}\par\medskip
$$ \int_0^\infty e^{-st}\,e^{as}\, f(s)\, ds = \int_0^\infty e^{-(s+a)\,t}f(t)\, dt=F(s+a).$$



\index{Faltung}
Um den n\"achtens Satz zu formulieren, 
ben\"otigen wir eine neue Verkn\"upfung zwischen zwei Funktionen: 
die Faltung.


\begin{sdefi}
Definiere f\"ur zwei Funktionen $f(t)$, $g(t)$ die Faltung (in Symbolen: $f\ast g(t)$, 
nicht zu verwechseln mit der Multiplikation $f(t)\cdot g(t)$) durch
$$ (f\ast g)(t) = \int_0^t f(t-\tau)\, g(\tau)\, d\tau.$$
\end{sdefi}

Warum definieren wir die Faltung? Erinnern wir uns an 
die Variation-der-Konstanten Formel f\"ur $x'=-a x + b(t)$, 
so k\"onnen wir schreiben
$$x(t) 
= x(0)e^{-at} + \int_0^t e^{-a(t-\tau)}\b(\tau)\, d\tau 
= x(0)e^{-at} + (e^{-a(\cdot)}\ast b)(t).$$

\fpbox{
\begin{minipage}{\textwidth}
Es gilt der {\normalfont\sffamily\bfseries Faltungssatz}
$$  {\cal L}\{f\ast g(t)\}(s) =
 {\cal L}\{f(t)\}(s) \,\cdot\,  {\cal L}\{g(t)\}(s).$$
 \end{minipage}
}\par\medskip
 Eine Faltung geht also in ein Produkt \"uber.
Beweis in \"Ubungsaufgabe~\ref{lpaplaceAuf4}.
\par\medskip


\fpbox{
\begin{minipage}{\textwidth}
\paragraph{Differenzieren}
Es gilt
$$ {\cal L}\{f'(t)\}(s) = s{\cal L}\{f(t)\}(s)- f(0)$$
und f\"ur h\"ohere Ableitungen
$$ {\cal L}\{f^{(n)}(t)\}(s) 
= s^n\, {\cal L}\{f(t)\}(s)- s^{n-1}\,f(0)-s^{n-2}\, f'(0)-\ldots-f^{(n-1)}(0).$$
 \end{minipage}
}\par\medskip
Beweis: zun\"achst f\"ur die erste Ableitung (wir nutzen partieller Integration)
$$
{\cal L}\{f'(t)\}(s) 
= \int_0^\infty e^{- s t} \frac d {dt}f(t)\, dt
= e^{-st}f(t)\bigg|_{t=0}^{t=\infty} - \int_0^\infty (-s)\,e^{- s t} f(t)\, dt
= s\,{\cal L}\{f\}(s)-f(0).
$$
Analog l\"auft auch der Beweis f\"ur h\"ohere Ableitungen \"uber partielle Integration
$${\cal L}\{f^{(n)}(t)\}(s)
= \int_0^\infty e^{-st}\, \frac d {dt}\,\,\,\frac{d^{n-1}}{dt^{n-1}}\, f(t)\, dt 
= -f^{(n-1)}(0) + s {\cal L}\{f^{(n-1)}(t)\}(s)$$
und Induktion.

\fpbox{
\begin{minipage}{\textwidth}
\paragraph{Integrieren}
Es gilt
$$ {\cal L}\left\{\int_0^t f(x)\, dx\right\}(s) = \frac 1 s {\cal L}\{f(t)\}(s).$$
 \end{minipage}
}\par\medskip
Beweis: (via partieller Integration)
\begin{eqnarray*}
&&{\cal L}\left\{\int_0^t f(x)\, dx\right\}(s) 
= \int_0^\infty e^{- s t} \int_0^t f(x)\, dx\, dt
= \int_0^\infty\int_0^t e^{- s t}  f(x)\, dx\, dt\\
&=& \int_0^\infty\int_x^\infty e^{- s t}  f(x)\, dt\, dx
= \int_0^\infty\left(\frac{e^{-st}}{-s}\bigg|_{t=0}^{t=\infty}\right) f(x)\, dt\, dx\\
&=& \int_0^\infty\left(\frac{e^{-st}}{-s}\bigg|_{t=0}^{t=\infty}\right) f(x)\, dx
= \frac 1 s \,\int_0^\infty\, e^{-sx}\,\, f(x)\, dx 
= \frac 1 s \,\int_0^\infty\, e^{-st}\,\, f(t)\, dt = 
\frac 1 s {\cal L}\{f(t)\}(s).
\end{eqnarray*}



%===============================================================================
\begin{auf}\chc\label{lpaplaceAuf4}
\input{../../Aufgabensammlung/hmIIlaplaceA4.tex}
\end{auf}
%===============================================================================

%===============================================================================
\begin{auf}\chb\label{lpaplaceAuf5}
\input{../../Aufgabensammlung/hmIIlaplaceA5.tex}
\end{auf}
%===============================================================================

%===============================================================================
\begin{auf}\chb\label{lpaplaceAuf6}
\input{../../Aufgabensammlung/hmIIlaplaceA6.tex}
\end{auf}
%===============================================================================


%===============================================================================
\begin{auf}\chb\label{lpaplaceAuf7}
\input{../../Aufgabensammlung/hmIIlaplaceA7.tex}
\end{auf}
%===============================================================================


%===============================================================================
\begin{auf}\chb\label{lpaplaceAuf8}
\input{../../Aufgabensammlung/hmIIlaplaceA8.tex}
\end{auf}
%===============================================================================


%AUFGABE\\
%Zeige ${\cal L}\{f\ast g(t)\}(s) =
% {\cal L}\{f(t)\}(s) \,\cdot\,  {\cal L}\{g(t)\}(s)$ 
% (Faltungssatz der Laplace-Transformation). Beachte: In dieser Aufgabe
%m\"ussen wir mit zweidimensionalen Integralen umgehen. Man kann in dem %auftrtenden 
%Integral in folgender Weise die Integrationsreihenfolge umdrehen:
%$$\int_0^\infty \int_0^t (\ldots)\, dx dt = \int_0^\infty \int_x^\infty %(\ldots)dt\, dx$$
%LOESUNG
%\begin{eqnarray*}
%&   & \int_0^\infty e^{-st} (f\ast g)(t)\, dt
% =  \int_0^\infty e^{-st} \int_0^t f(t-x)g(x)\, dx\, dt \\
%& = & \int_0^\infty e^{-st} \int_0^t f(t-x)g(x)\, dx\, dt \\
%& = & \int_0^\infty\int_0^t e^{-s(t-x)}\,f(t-x)\, e^{-s x}\, g(x)\, dx\, %dt 
% =  \int_0^\infty\int_x^\infty e^{-s(t-x)}\,f(t-x)\, e^{-s x}\, g(x)\, %dt\, dx \\
%& = & \int_0^\infty\int_x^\infty e^{-s(t-x)}\,f(t-x)\,dt \,\, e^{-s x}\,%g(x)\, dx
% =  \int_0^\infty\int_0^\infty e^{-s(y)}\,f(y)\,dy \,\, e^{-s x}\, g(x)\, %dx\\
%& = & F(s)\,\int_0^\infty \,\, e^{-s x}\, g(x)\, dx = F(s)\, G(s).
%\end{eqnarray*}
%LOESUNGSHAPPEN: 
%$$ \int_0^\infty e^{-st} (f\ast g)(t)\, dt=
%\int_0^\infty\int_0^t e^{-s(t-x)}\,f(t-x)\, e^{-s x}\, g(x)\, dx\, dt 
%=
% \int_0^\infty\int_x^\infty e^{-s(t-x)}\,f(t-x)\,dt \,\, e^{-s x}\, g(x)\, %dx
% = F(s) G(s).$$
%




%AUFGABE:START
%Zeige den D"ampfungssatz, ${\cal L}\{e^{-\alpha t}\, f(t)\}(s) = {\cal %L}\{f(t)\}(s+\alpha)$.
%LOESUNG:
%$${\cal L}\{e^{-\alpha t}\, f(t)\}(s) 
%= \int_0^\infty e^{-st}e^{-\alpha t}\, f(t)\, dt 
%= \int_0^\infty e^{-(s+\alpha)t}\, f(t)\, dt 
%= {\cal L}\{f(t)\}(s+\alpha).$$
%AUFGABE:ENDE


%AUFGABE:START
%Zeige den zweiten Verschiebungssatz: Sei $H(t)$ definiert durch 
%$H(t)=0$ f\"ur $t\leq 0$, und $H(t)=1$ f\"ur $t>0$ ($H(t)$ ist die 
%Heavisdefunktion, benannt nach dem Physiker %Heaviside)\index{Heavisidefunktion}. 
%Dann gilt
%$${\cal L}\{H(t-b)\,f(t-b)\}(s) = e^{-bs}\,\,{\cal L}\{f(t)\}(s).$$
%LOESUNG:
%\begin{eqnarray*} 
%{\cal L}\{H(t-b)\,f(t-b)\}(s)
%&=& \int_0^\infty e^{-st}\,H(t-b)\, f(t-b)\, dt
%= \int_b^\infty e^{-st}\,H(t-b)\, f(t-b)\, dt\\
%&=& \int_b^\infty e^{-s(t'+b)}\,H(t')\, f(t')\, dt'
%= e^{-bs}\,\,{\cal L}\{f(t)\}(s).
%\end{eqnarray*}
%AUFGABE:ENDE


%AUFGABE:START
%Zeige den \"Ahnlichkeitssatz, d.h.\ 
%f\"ur $a>0$, 
%${\cal L}\{f(a\,t)\}(s) = \frac 1 a \,\,\,{\cal L}\{f(t)\}(s/a)$.
%LOESUNG:
%$$ {\cal L}\{f(a\,t)\}(s) 
%= \int_0^\infty e^{-st}\, f(a\,t)\, dt
%= \frac 1 a \,\,\int_0^\infty e^{-(s/a)\, at}\, f(a\,t)\, a\,dt
%= \frac 1 a \,\,\int_0^\infty e^{-(s/a)\, t'}\, f(t')\, dt'
%= \frac 1 a \,\,\,{\cal L}\{f(t)\}(s/a).
%$$
%AUFGABE:ENDE

%AUFGABE:BEGIN
%Finde die Laplace-Transformierte der Funktion
%$$ f(t) = sin(\omega t) e^{- a t}.$$
%LOESUNG:
%Tabelle: 
%${\cal L}\{sin(\omega t)\} = \frac {\omega}{s^2+\omega^2}$.\\
%Verschiebungssatz: ${\cal L}\{e^{a\,t}f(t)\}(s)={\cal L}\{f(t)\}(s+a)$.\\
%Also: ${\cal L}\{sin(\omega t)\, e^{-a t}\}(s) =  \frac {\omega}%{(s+a)^2+\omega^2}$.
%LOESINGSHAPPEN: Tabelle: ${\cal L}\{sin(\omega t)\} = %\omega/(s^2+\omega^2)$
%+ Verschiebungssatz.
%AUFGABE ENDE.


\subsection{Laplace-Transformation und lineare Differentialgleichungen}
Wir sehen uns lineare Differentialgleichungen an, wobei nur die Inhomogenit\"at
von der Zeit abh\"angen darf - die homogene Differentialgleichung muss autonom sein.

\subsubsection{Skalare Gleichung erster Ordnung}
Wenn wir auf 
$$ x'(t) = a \,\, x(t) + b(t),\qquad\quad x(0)=x_0$$
die Laplace-Transformation anwenden, so erhalten wir 
$$ s {\cal L}\{x(t)\}(s) - x(0) = a {\cal L}\{x(t)\}(s) + {\cal L}\{b(t)\}(s).$$
Aufl\"osen nach ${\cal L}\{x(t)\}(s)$:
$$ {\cal L}\{x(t)\}(s) 
= \frac{x_0\,+\,{\cal L}\{b(t)\}(s)}{s-a}
= \frac{x_0}{s-a}
+ \frac{1}{s-a}\,\, {\cal L}\{b(t)\}(s).
$$
R\"ucktransformation (beachte ${\cal L}\{e^{at}\}=1/(s-a)$, 
Linearit\"at der Laplace-Transformation und Faltungssatz)
$$ x(t) 
= x_0\, e^{at} + (e^{a\,(\cdot)}\ast b)(t) 
= x_0\, e^{at} + \int_0^t e^{a(t-\tau)} b(\tau)\, d\tau.$$
Wir finden also (wie erwartet) die Variation-der-Konstanten-Formel wieder.
\par\medskip

\fpbox{ \index{Transferfunktion}
H\"aufig interessiert man sich nur f\"ur das langfristige Verhalten einer
L\"osung. Wenn wir ein Radio einschalten, m\"ochten wir Musik h\"oren. 
Wenn in den ersten Millisekunden noch keine Musik gespielt wird, weil sich
einige Kondensatoren noch aufladen m\"ussen ist uns das egal. 
Wenn unser Rechner erst nach einem kurzen Moment startet, ist das 
egal -- Hauptsache er l\"auft anschlie\ss{}end zuverl\"assig. 
Das sogenannte transiente Verhalten ist oft uninteressant, wir
m\"ochten vor allem das langfristige Verhalten verstehen und kontrollieren.\\
Im Beispiel oben, wird f\"ur $a<0$ der Einflu\ss{} der Anfangsbedingung
exponentiell verschwinden. Daher k\"onnen wir -- wenn uns die erste, sogenannte 
transiente Phase nicht interessiert, die Anfangsbedingung in der Laplace-transformierten
Gleichung einfach weglassen, und betrachten statt 
$${\cal L}\{x(t)\}(s) = \frac{x(0)}{s-a} +  \frac{1}{s-a}\,\,{\cal L}\{b(t)\}(s)$$
die Gleichung
$${\cal L}\{x(t)\}(s) = \frac{1}{s-a}\,\,{\cal L}\{b(t)\}(s).$$
Die Laplace-transformierte des Eingangssignals wird also einfach mit 
$$G(s) = \frac 1{s-a}$$ 
multipliziert. Die Funktion $G(s)$, mit der ${\cal L}\{b(t)\}(s)$ 
multipliziert wird, hei\ss{}t Transferfunktion. \index{Transferfunktion}
Die St\"arke dieser Erkenntnis diskutieren wir unten nochmals etwas detaillierter 
unten, am Ende dieses Kapitels \"uber Laplacetransformation.
}\par\bigskip



\subsubsection{System erster Ordnung}
Wie eben, nur ist jetzt $x(t)\in\R^n$ ein Vektor, $b(t)\in\R^n$ ein Vektor, und statt $a$ schreiben 
wir $A\in\R^{n\times n}$:
\begin{eqnarray*}
\mbox{Gleichung} \quad & x'(t) = & a \,\, x(t) + b(t),\qquad\quad x(0)=x_0\\
\mbox{Laplace-Transformierte} \quad & s {\cal L}\{x(t)\}(s) - x(0)  =&  A {\cal L}\{x(t)\}(s) + {\cal L}\{b(t)\}(s)\\
\mbox{Aufl\"osen nach }{\cal L}\{x(t)\}(s): \quad &
{\cal L}\{x(t)\}(s) 
= & -(A-s I)^{-1}x_0
-(A-s I)^{-1}\, ,{\cal L}\{b(t)\}(s).
\end{eqnarray*}
Die Transferfunktion in diesem Fall ist also gegeben durch $G(s) = -(A-sI)^{-1}$. 
Um die Gleichung zu l\"osen, betrachten wir das lineare Gleichungssystem 
in den Komponenten von ${\cal L}\{x(t)\}(s)$, und l\"osen dieses nach den 
einzelnen Komponenten auf. Dann k\"onnen wir die R\"ucktransformation vornehmen,
und haben die L\"osung gefunden.


\begin{bspX}
Wenn wir das System
\begin{eqnarray*}
x' & = & 3\,y-x,\qquad\quad x(0)=0\\
y' & = & x+y\qquad\qquad y(0)=1
\end{eqnarray*}
per Laplacetransformation l\"osen m\"ochten, so transformieren wir beide Zeilen.
Sei $\hat x =\hat x(s) = s{\cal L}\{x(t)\}(s)$, 
$\hat y =\hat y(s) = s{\cal L}\{y(t)\}(s)$, so
\begin{eqnarray*}
s\,\hat x-0 &=& -\hat x + 3 \hat y\\
s\,\hat y-1 &=& \hat x + \hat y
\end{eqnarray*}
oder -- mit Hilfe des Gau\ss{}-Schemas -- k\"onnen wir schrieben
\begin{eqnarray*}
\,\,\qquad\left(\begin{array}{cc|c}
s+1 & -3 & 0\\
1 & 1-s & -1
\end{array}\right)\\
\Rightarrow\quad\left(\begin{array}{cc|c}
s+1 & -3 & 0\\
0 & 1-s+\frac 3 {1+s} & -1
\end{array}\right)
\end{eqnarray*}
Also
$$ \hat y(s) = \frac {-1}{1-s+\frac 3 {1+s}} 
= \frac{-(1+s)}{(1-s)(1+s)+3}
= \frac{-(1+s)}{4-s^2}
= \frac{1+s}{s^2-4}
$$
und 
$$ \hat x(s) = \frac{3}{s+1}\, \hat y = \frac{3}{s^2-4}.$$
Nun transformieren wir $\hat x(s)$ uns $\hat y(s)$ zur\"uck; dazu ben\"otigen wir
Partialbruchzerlegung, 
\begin{eqnarray*}
\hat x(s) &=& \frac{3}{s^2-4} = \frac{3/4}{s-2}-\frac{3/4}{s+2}\\
\hat y(s) &=& \frac{1+s}{s^2-4} = \frac{3/4}{s-2}+\frac{1/4}{s+2}\\
\end{eqnarray*}
und folglich
$$
x(t) = \frac 3 4 \,\,e^{2\,t}\,\,\,-\,\,\,\frac 3 4 \,\,e^{-2\,t},\quad
y(t) = \frac 3 4 \,\,e^{2\,t}\,\,\,+\,\,\,\frac 1 4 \,\,e^{-2\,t}.
$$
\end{bspX}

\subsubsection{Skalare Gleichung $n$'ter Ordnung}
Sei nun ein lineares Anfangswertproblem
$$ \sum_{i=0}^n\, a_i x^{(i)}(t) = b(t),\qquad x^{(i)}(0)=x_0^i, \qquad i=0,\ldots, n-1$$
gegeben. Wenn wir diese Gleichung Laplacetransformieren, so finden wir wieder eine
algebraische Gleichung in ${\cal L}\{x(t)\}(s)$, die man nach dieser Funktion aufl\"osen
kann. Das Ergebnis kann dann (in vielen F\"allen) zur\"ucktransformiert werden. Wir 
sehen uns das in einem Beispiel an.

\begin{bspX}
$$x'' + x' - 2 x  = 0,\qquad x(0)=1, x'(0)=0.$$
Laplacetransformation dieser Gleichung liefert:
$$ \bigg(-x'(0)-s\, x(0) + s^2 {\cal L}\{x(t)\}(s)\bigg)
+
\bigg(-x(0) + s {\cal L}\{x(t)\}(s)\bigg)
- 2 
{\cal L}\{x(t)\}(s)  =0$$
und wenn wir die Anfangswerte nutzen erhalten wir
$${\cal L}\{x(t)\}(s) 
= \frac{1+s}{s^2+s-2}
= \frac{1+s}{(s-1)\,\,(s+2)}.
$$
Partialbruchzerlegung ergibt
$$
 \frac{s+1}{(s-1)\,\,(s+2)}
= \frac{2/3}{(s-1)}+\frac{1/3}{(s+2)}.
$$
Damit k\"onnen wir die R\"ucktransformation durchf\"uhren,
$$x(t) = \frac 2 3\,\,\,  e^{t}    +\,\frac  1 3\,\,\,   e^{-2\, t}.$$ 
\end{bspX}


%===============================================================================
\begin{auf}\chc\label{lpaplaceAuf11}
\input{../../Aufgabensammlung/hmIIlaplaceA11.tex}
\end{auf}
%===============================================================================


%===============================================================================
\begin{auf}\chb\label{lpaplaceAuf12}
\input{../../Aufgabensammlung/hmIIlaplaceA12.tex}
\end{auf}
%===============================================================================



\subsection{Transferfunktion}


Eine St\"arke der Laplace-Transformation ist die Betrachtung des {\it langfristigen} 
Verhaltens von gr\"o\ss{}eren Systemen. Hier denken wir an z.B.\ 
elektrotechnischen Systemen, bei dem ein Signal $b(t)$ durch verschiedene 
Filter, Verst\"arker etc.\ geleitet wird (Fig.~\ref{systTrans}. 
Jedes dieser Subsysteme kann durch 
lineare Differentialgleichungen dargestellt werden, bekommt als Eingangssignal 
das Ausgangssignal eines anderen (Sub)Systems. 

\begin{figure}[htb]
\begin{center}
\includegraphics[width=12cm]{../figures/transferStruct-eps-converted-to.pdf}
\end{center}
\caption{Wenn ein System aus mehreren Subsystemen aufgebaut ist, kann die 
gesamte Transferfunktionen einfach aus den Transferfunktionen des Subsysteme berechnet werden.}
\label{systTrans}
\end{figure}


Sehen wir uns ein Subsystem ein. Eingangssignal $b(t)$, Ausgangssignal $x(t)$. 
$x(t)$ ist die L\"osung einer linearen Differentialgleichungen, die $b(t)$ 
als Anfangswert besitzt. Laplace-Transformation liefert 
$$
{\cal L}\{x(t)\}(s) 
= G(s)\,\,\bigg(\mbox{Polynom, abh\"angig von  }s\mbox{ und Anfangswerten }\bigg)
+ G(s) {\cal L}\{b(t)\}(s). 
$$
wobei $G(s)$ eine Funktion ist, die von der Differentialgleichung bestimmt wird. 
Die Anfangsbedienungen werden f\"ur die Differentialgleichungen, die wir hier betrachten, 
langfristig (nach einer gewissen zeit) keine Rolle mehr spielen, 
z.B. exponentiell abklingen.
Also gen\"ugt es letztlich, 
$$
{\cal L}\{x(t)\}(s) 
= G(s) {\cal L}\{b(t)\}(s) 
$$
zu betrachten. $G(s)$ hei\ss{}t die Transferfunktion des Subsystems. \par\bigskip

\begin{bspX}
Ein typisches Subsystem ist ein Tiefpassfilter. Der wird durch die Differentialgleichung 
$$x' = - a x + a\,b(t)$$
beschrieben. $a$ gibt die Zeitkonstante des Tiefbass-Filters an. 
Wir werden die Transferfunktion bestimmen, und uns \"uberlegen, wie 
eine Sinus-Signal von dem Tiefpassfilter ver\"andert wird.\par\medskip

Um die Transferfunktion zu finden, wenden wir die 
Laplacetransformation auf beide Seiten an,
\begin{eqnarray*}
s\, {\cal L}\{x(t)\}(s) - x(0) &=& - a {\cal L}\{x(t)\}(s) +a {\cal L}\{b(t)\}(s) \\
{\cal L}\{x(t)\}(s) &=& \frac{x(0)}{s+a} +  \frac{a}{s+a}\,\,{\cal L}\{b(t)\}(s),
\end{eqnarray*}
also
$$ G(s) = \frac a {s+a}.$$
Was sagt uns diese Transferfunktion? Wir k\"onnen ein Sinus-Signal an den Eingang 
legen, $b(t)=\sin(\omega t)$. Dann ist (siehe Tabelle) 
$$ {\cal L}\{b(t)\} ={\cal L}\{\sin(t)\} = \frac{\omega}{s^2+\omega^2}.$$
Das Ausgangssignal wird also (langfristig) durch 
$${\cal L}\{x(t)\}(s) 
=  \frac{\omega\, a}{(s+a)\,(s^2+\omega^2)}.$$
F\"ur die R\"ucktrafo nutzen wir die Partialbruchzerlegung
$$
\frac{\omega\,a}{(s+a)\,(s^2+\omega^2)}
=
\frac A {s+a} + \frac {B\,\omega} {s^2+\omega^2} + \frac {C\, s} {s^2+\omega^2} 
$$
(wir setzen $B\omega$ an, da wir dann sofort 
f\"ur die R\"ucktransformation einen Fall aus der Tabelle erhalten) 
sodass
$$ A(s^2+\omega^2) + B \omega(s+a)+ C s\,(s+a) = \omega\, a$$
und also 
\begin{eqnarray*}
A+C &=& 0\\
B\omega + a C &=& 0\\
A\omega^2 + a\, B\omega &=& \omega\, a.
\end{eqnarray*}
Die L\"osung dieses LGS ist 
$$ A = \frac{\omega\, a}{a^2+\omega^2},\quad B = \frac{a^2}{a^2+\omega^2},\quad C = \frac{-\omega\, a}{a^2+\omega^2}$$
und daher (nochmals: wir fokussieren auf $t\gg 1$)
\begin{eqnarray*}
x(t) &=& 
 \frac{\omega\, a}{a^2+\omega^2}\, e^{-a t}
 + \frac{a^2}{a^2+\omega^2} \sin(\omega t)
 + \frac{-\omega\, a}{a^2+\omega^2} \cos(\omega t)\\
 &\approx& \frac {a}{\sqrt{a^2+\omega^2}}\,\,\left(
 \frac{a}{\sqrt{a^2+\omega^2}} \sin(\omega t)
 - \frac{\omega}{\sqrt{a^2+\omega^2}} \cos(\omega t)\right).
\end{eqnarray*}
 Das Signal erf\"ahrt also einen Frequenz-abh\"angigen Phasen-shift 
 (das ist beim Zusammenspiel 
 mehrerer Subsysteme manchmal wichtig), und eine D\"ampfung um den Faktor 
 $\frac {a}{\sqrt{a^2+\omega^2}}$ (siehe Fig.~\ref{deampf} f\"ur den Graphen 
 der D\"ampfung). Hohe Frequenzen werden
 stark ged\"ampft, niedrige Frequenzen dagegen wenig.  Wenn $\omega=a$ ist, 
 so betr\"agt die D\"ampfung gerade $1/\sqrt{2}$. Der Tiefpass-Filter l\"asst 
 also Frequenzen kleiner $a$ gut durch, Frequenzen gr\"o\ss{}er $a$ hingegen nicht.
 \end{bspX}
 
 \begin{figure}[htb]
 \begin{center}
 \includegraphics[width=10cm]{../figures/tiefLaplace1-eps-converted-to.pdf}
 \end{center}
 \caption{D\"ampfung des Tiefpass-Filters ($a=1$). Wenn die Frequenz gleich $a$, 
       so betr\"agt die D\"ampfung gerade $1/\sqrt{2}$. }\label{deampf}
 \end{figure}


Schalten wir zwei Subsysteme hintereinander, finden wir
\begin{eqnarray*}
{\cal L}\{x_1(t)\}(s) &=& G_1(S) {\cal L}\{b(t)\}(s) \\
{\cal L}\{x_2(t)\}(s) &=& G_2(S) {\cal L}\{x_1(t)\}(s) \\
\Rightarrow\qquad {\cal L}\{x_2(t)\}(s) &=& G_2(s)\, G_1(s)\, {\cal L}\{b(t)\}(s).
\end{eqnarray*}
Beim Hintereinanderschalten zweier Subsysteme m\"ussen wir also, um die Transferfunktion
des Systems zu erhalten, die Transferfunktionen der Subsysteme multiplizieren. \\
Schalten wir die Subsysteme parallel, so m\"ussen wir 
die Transferfunktionen addieren. Das Signal kann auch r\"uckgekoppelt werden 
(typischer Weise in der Form eines negativen Feedbacks), und auch dann finden
wir eine effektive Transferfunktion (\"uberlegen wir hier nicht, das 
ist Stoff z.B.\ der Verfahrenstechnik).


\par\bigskip


%===============================================================================
\begin{auf}\chd\label{lpaplaceAuf9}
\input{../../Aufgabensammlung/hmIIlaplaceA9.tex}
\end{auf}
%===============================================================================

%===============================================================================
\begin{auf}\che\label{lpaplaceAuf10}
\input{../../Aufgabensammlung/hmIIlaplaceA10.tex}
\end{auf}
%===============================================================================
