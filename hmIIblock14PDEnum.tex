


%% ! T E X  root=hmIIneu.tex
% !TEX root=MA9603.WZW.tex
% !TEX program = pdflatex
% !TEX spellcheck = de_DE
%\setcounter{section}{10}\setcounter{page}{112}
\section{Numerik Partieller Differentialgleichungen}



\zbox{
{\bf Ziele}:
\begin{itemize}
\item Finite Differenzenmethode f"ur elliptische PDE's kennen
\item Linienmethode f"ur parabolische PDE's kennen
\item Finite Differenzenmethode f"ur parabolische PDE's kennen
\item Spektralmethoden f"ur elliptische und parabolische PDE's kennen
\end{itemize}}
  

\bigskip
\bigskip
Wie gew"ohnliche Differentialgleichungen, so k"onnen auch 
partielle Differentialgleichungen (abgek"urzt: PDE's) 
h"aufig nicht mehr
analytisch behandelt werden. Es wurden eine Vielzahl von 
Methoden entwickelt, um L"osungen numerisch zu approximieren. 
Wir sehen uns in dieser Lehreinheit einige dieser Methoden an. 
Letzlich k"onnen Computer nur mit endlich vielen Zahlen umgehen. 
Daher m"ussen wir die L"osung durch endlich viele Zahlen 
codieren, oder zumindest approximativ codieren. 

\bigskip

\fpbox{
Wir haben zwei prinzipielle Wege um eine Funktion 
$u(x,y)$ durch endlich viele Zahlen darzustellen: (1) Wir sehen uns
die Funktion an endlich vielen Punkten $(x_i,y_i)$ im Raum 
an, und erhalten
eine Liste von Eintr"agen
$$u(x_1,y_1), u(x_2,y_2), \cdots, u(x_N,y_N).$$
(2) Wir approximieren den Funktionenraum, d.h. wir w"ahlen endlich viele 
Basisfunktionen $\varphi_i(x,y)$ und codieren $u(x,y)$ 
approximativ durch eine Liste von Zahlen
$$ c_1,\cdots, c_N$$
dar, wobei wir aus den $c_i$ die Funktion (oder deren Approximation) $u(x,y)$  durch
$$ u(x,y) = \sum_{i=1}^N c_i \varphi_i(x,y)$$
rekonstruieren. \\
Der erste Ansatz f"uhr auf finite Differenzen-Verfahren, der zweite Ansatz auf Spektralmethoden.
}

%===============================================================================

\begin{figure}[htbp]
 \centering
 \rotatebox{180}{
   \includegraphics[width=10cm]{../figures/pdeNumerikKerze}
   }
   \caption{Aufbau des Testproblems.}
   \label{numPDEtest}
\end{figure}


%===============================================================================

\subsection{Testproblem}
Wir betrachten eine Kerze unter einem Stab der L"ange $L$, 
dessen beiden Enden in 
Eiswasser stecken (Abb.\ref{numPDEtest}). Wir m"ochten die Temperaturverteilung berechnen.
Sei $u(t,x)$ die Temperatur am Ort $x$ zur Zeit $L$, so finden wir
$$ u_t(t,x) = \underbrace{u_{xx}(t,x)}_{Diffusion} + \underbrace{f(x)}_{W"armequelle}$$
(Diffusionskonstante $D$ ist auf 1 gesetzt)
wobei $f(x)$ die W"armezufuhr der Kerze repr"asentiert. Da diese bei $x=L/2$ stehen soll, w"ahlen wir
$$ f(x) = 1\quad\mbox{ falls } x\in (L-0.05, L+0.05)\quad\mbox{und } f(x)=0\quad\mbox{sonst}.$$
Die beiden Enden des Stabes stehen in Eiswasser, also
kennen wir dort die Temperatur,
$$ u(t,0)=0, \quad u(t,L)=0.$$
Wir starten mit einer gegebenen Anfangstemperatur $u_0(x)$,
$$ u(0,x) = u_0(x) = 0.$$
z.B. $u_0(x)=0$.\par\medskip

Falls wir noch die W"armeabstrahlung des Stabes integrieren m"ochten, so
"andert sich die Gleichung leicht
$$ u_t(t,x) = u_{xx}(t,x)\underbrace{-\mu u(t,x)}_{Abstrahlung} + f(x), \quad u(t,0)=u(t,L)=0,\quad u(0,x)=u_0(x).$$
\par\medskip

Wenn wir keine Stab, sondern eine Platte der Form $\Omega$ 
erhitzen (deren R"ander wir ebenfalls k"uhlen), so erhalten wir 
$$ u_t(t,x) = \Delta u(t,x)-\mu u(t,x) + f(x), \quad u|_{\partial\Omega} = 0,\quad u(0,x)=u_0(x)$$
(Erinnerung: $\partial\Omega$ bezeichnet den Rand eines Gebietes).
\par\medskip

Wir werden damit starten, nicht das zeitabh"angige Problem zu betrachten, sondern studieren zun"achst intensiv das station"are Problem. Hier ist 
$u_t=0$, und wir erhalten die elliptische Gleichung
$$ 0 = u''(x)-\mu u(x)+f(x), \quad u(0)=u(L)=0$$
bzw.
$$  u''(x) -\mu u(x) = -f(x), \quad u(0)=u(L)=0$$
Diese Gleichung hei\ss{}t (Poisson-Gleichung).\index{Poisson-Gleichung}
In der zweidimensionalen Situation lautet diese
$$  \Delta u(x) -\mu u(x) = -f(x), \quad u|_{\partial \Omega}=0.$$



\subsection{Finite Differenzenmethode f"ur elliptische Differentialgleichungen}
\index{Differenzenmethode!elliptische Differentialgleichungen}

\subsubsection{Eindimensionale Situation}
Um zu verstehen, wie wir mit dem Problem umgehen k"onnen, 
sehen wir uns erst mal den eindimensionalen Fall an:
$$u''(x) = f(x),\quad u(0)=g_0,\quad u(L)=g_1.$$
Nat"urlich k"onne wir in diesem Fall die analytische L"osung 
sofort angeben, wir m"ussen nur zwei mal integrieren:
$$ 
u(x) = c_1+c_2\, x - \int_0^x\int_0^{x'} f(x'')\, dx''\, dx'$$
wobei $c_1$ und $c_2$ durch die Randbedingungen bestimmt sind. 
Allerdings ist das nicht der Weg, den wir gehen wollen, da diese
Idee nicht leicht auf h"ohere Dimensionen verallgemeinerbar ist. 
Wir gehen den Weg, den wir auch zur Herleitung des Euler-Verfahrens 
f"ur gew"ohnliche Differentialgleichungen gegangen sind: 
Wir approximieren die Ableitung (Differentialquotient) durch einen
Differenzenquotient. 

%===============================================================================

\begin{figure}[htbp]
 \centering
   \includegraphics[width=10cm]{../figures/secondDeriv.pdf}
   \caption{Approximation der zweiten Ableitung durch einen Differenzenquotienten.}
   \label{numPDEdiff}
\end{figure}

%===============================================================================

\fpbox{
Um $u''(x)$ zu approximieren, bemerken wir, dass
$$ u''(x) \approx \frac{u'(x+h/2)-u'(x-h/2)}{h}$$
und 
$$  u'(x-h/2) \approx \frac{u(x)-u(x-h)}{h},\qquad
 u'(x+h/2) \approx \frac{u(x+h)-u(x)}{h}.$$
Setzen wir dies ineinander ein, so finden wir (siehe auch 
Abb.~\ref{numPDEdiff})
$$ u''(x) \approx \frac{u(x-h)-2\, u(x)+u(x+h)}{h^2}.$$
}
\bigskip

Nun kehren wir zur"uck zur Gleichung $u''(x) = f(x)$, $u(0)=u(L)=0$. Wir diskretisieren das Intervall $[0,L]$ mit der Schrittweite 
$h=L/N$, $N\in\N$. Sei $x_i= i\, h$, sodass $x_0=0$, $x_N=L$ 
und $u^i=u(x_i)$. 
Dann k"onnen wir $u''(x) = f(x)$ durch
$$ - f(x_i) = u''(x_i) 
 \approx \frac{u(x_i-h)-2\, u(x_i)+u(x_i+h)}{h^2}
= \frac{u^{i-1}-2\, u^i+u^{i+1}}{h^2} \qquad\mbox{f"ur } 
i=1,\cdots N-1
$$
approximieren. Die Punkte $u^0=u(0)=0$ 
und $u^N=u(L)=0$ sind besonders, da
deren Werte durch die Randbedingungen schon gegeben sind. Wir stellen 
also f"ur $i=1$ und $i=N-1$ die Gleichungen um,
\begin{eqnarray*}
\frac{u^{0}-2\, u^1+u^{2}}{h^2} = -f(x_1) &\Leftrightarrow & 
\frac{-2\, u^1+u^{2}}{h^2} = -f(x_1)\\
\frac{u^{N-2}-2\, u^{N-1}+u^{N}}{h^2} = -f(x^{N-1}) &\Leftrightarrow & 
\frac{u^{N-2}-2\, u^{N-1}}{h^2} = -f(x_{N-1}). 
\end{eqnarray*}
Wir erhalten also ein system linearer Gleichungen f"ur die Unbekannten
$u^1,\cdots,u^{N-1}$. In Matrix-Schreibweise haben wir 
$$ A\vec{u} = \vec{F}$$
mit
$$ 
\vec{u} = \left(\begin{array}{c}u^1\\\vdots\\u^{N-1}\end{array}\right),
\quad
\vec{F} = \left(\begin{array}{l}-f(x_1)\\-f(x_2)\\\vdots\\-f(x_{N-2})\\-f(x_{N-1})\end{array}\right),\quad 
A = h^{-2}\,\left(
\begin{array}{ccccccc}
-2 & 1 & 0 & 0 &\cdots & 0 & 0\\
 1 &-2 & 1 & 0 &\cdots & 0 & 0\\
 0 & 1 &-2 & 1 &\cdots & 0 & 0\\
   &   &   &   &\ldots &   &  \\
 0 & 0 & 0 & 0 &\cdots & 1 & -2
 \end{array}\right). 
$$
Diese Approximation nennt man die Finite-Differnzen-methode zur L"osung
von PDE's. \index{Finite Differenzen}
\fpbox{
Dieses lineare Gleichungssystem m"ussen wir l"osen. 
Wir haben eine Methode gut trainiert: das Gau\ss{}-Verfahren. Aber
auch die Singul"arwert-Zerlegung k"onnen wir dazu nutzen. Haben
wir $A$ erst einmal als
$$A = V \Sigma U$$
dargestellt, wobei $U,V$ orthogonale Matrizen sind, und $\Sigma$ eine
Diagonalmatrix ist, so  nutzen wir, dass $U^TU=V^TV=I$ und die Inverse
einer Diagonalmatrix $\Sigma$ leicht zu berechnen ist (einfach die inversen der Diagonalelemente in die Diagonale schrieben). Wir erhalten
$$ A\vec u = \vec F
\quad\Leftrightarrow\quad
U\Sigma V \vec u = \vec F
\quad\Leftrightarrow\quad
\vec u = V^T \Sigma^{-1} U^T \vec F.$$
F"ur gr"o\ss{}e Matrizen ist (geschickt implementiert) der 
Weg "uber die Singul"arwertzerlegung (oder anderer, "ahnliche 
 Zerlegungen) 
oft effizienter als das Gau\ss-Verfahren. 
}
\par\bigskip

Nun wissen wir, wie wir die Gleichung ohne Abstrahlung diskretisieren. 
Was passiert, wenn wir Abstrahlung noch mit einbeziehen? Der Differentialoperator und die rechte Seite wird exakt genausi wie vorher disretisiert; auf der linkten Seite erscheint allerdings noch ein zus"atzlicher Term.
\begin{eqnarray*}
u''(x) - \mu u(x) &=& f(x)\\
      &\downarrow&\\
 A \vec u - \mu\vec u &=& \vec F
\end{eqnarray*}
Wir haben wirder ein LGS in $\vec u$, das wir mit den uns zur Verf"ugung stehenden Methoden l"osen k"onnen.

\begin{bspX}
Wir w"ahlen nun $L=1$. 
Wir sehen uns verschiedene Diskretisierungen an (Abb.~ \ref{numPDEFinDiffExpl}) und finden
eine gute Konvergenz ab 50 bis 100 Punkte.\\
Analog f"ur die Gleichung 
$$ - u''(x) = -\mu u(x) +f(x)$$
wobei wir $\mu =100$ w"ahlen. Auch hier konvergiert das Verfahren
gut f"ur $N\approx 50$ oder $N\approx 100$.
\end{bspX}

%===============================================================================

\begin{figure}[htbp]
 \centering
   \includegraphics[width=10cm]{../figures/diffFinElem.pdf}
   \caption{Approximation der L"osung con $u''(x) = -f(x)$ (links)
   und $u''(x) = \mu u(x)-f(x)$ (rechts).}
   \label{numPDEFinDiffExpl}
\end{figure}

%===============================================================================

\subsection*{Aufgaben}
\begin{auf}\chb\label{bloc11XA1}
\input{../../Aufgabensammlung/hmIIpde01.tex}
\end{auf}

\begin{auf}\chc\label{bloc11XA2}
\input{../../Aufgabensammlung/hmIIpde02.tex}
\end{auf}

%===============================================================================


\subsubsection{Zweidimensionales Gebiet}

Der Weg f"ur en zweidimensionales Gebiet ist der Gleiche wir f"ur das eindimensionale Gebiet. Nur m"ussen wir diesmal $\Delta u = u_{xx}+u_{yy}$ approximieren. 

\fpbox{
Um $\Delta u = u_{xx}+u_{yy}$ zu approximieren, diskretisieren wir zun"acht
den zweidimensionalen Raum, $(x,y) = (i\, h, j\, h)$ f"ur $i,j\in\Z$. 
Sei weiter $u^{i,j}\approx u(i\, h,j\, h)$, die Approximation 
der Funktion $u$ am Punkt $(x,y) = (i\, h, j\, h)$. \\
Aus den "Uberlegungen von oben entnehmen wir
$$ u_{xx}(i\, h, j\, h) = h^{-2}(u^{i-1,j}-2u^{i,j}+u^{i+1,j})$$
und 
$$ u_{yy}(i\, h, j\, h) = h^{-2}(u^{i,j-1}-2u^{i,j}+u^{i,j+1}).$$
Daher:
$$ \Delta u =  h^{-2}(u^{i-1,j}+u^{i,j-1}-4u^{i,j}+u^{i+1,j}+u^{i,j+1})
.$$
Das ist die Stern-Approximation des Laplace-Operators, da die Punkte "uber und unter und rechts und links von dem Bezugspunkt ben"otigt werden (Abb.~\ref{pdeNumstern}).
}


%===============================================================================

\begin{figure}[htbp]
 \centering
   \includegraphics[width=5cm]{../figures/pdeNumStertn.pdf}
   \caption{Approximation des Laplace-Operators durch den Vier-Punkte-Stern (die Gewichte der Formel sind an den Kreisen annotiert). }
   \label{pdeNumstern}
\end{figure}

%===============================================================================

Damit k"onnen wir analog wie oben die Partielle Differentialgleichung
durch ein lineares Gleichungssystem approximieren, das wir mit unseren Methoden l"osen k"onnen.


\subsection{Spektralmethode f"ur Elliptische Gleichungen}\index{Spektralmethode}
Die Differenzenmethode zielt auf die Approximation des Raumes durch
diskrete Punkte. Die Spektralmethode geht einen v"ollig anderen Weg: diese zielt auf die Approximation von Funktionen durch eine dem Problem besonders
angemessene Funktionenfamilie. Die Wahl dieser Ansatzfunktionen ist eine 
Kunst und bestimmt wesentlich die Qualit"at des Verfahrens. 
Um einen Einblick in die Idee dieser Methode zu erhalten
 betrachten wir wieder das eindimensionale Problem
 $$ u''(x) = -f(x),\qquad u(0)=u(L)=0. $$
Wir wollen Ansatzfunktionen $\varphi_i(x)$ finden, f"ur das dieses
Problem einfach zu handhaben ist. Dazu stellen wir folgende "Uberlegung an: Die zweite Ableitung ist ein Operator. Wir k"onnen 
auch definieren
$$ A u = u''\qquad \mbox{ f"ur Funktionen, f"ur die gilt } u(0)=u(L)=0.$$
$Au$ sieht aus wie das Produkt einer Matrix $A$ mit einem Vektor $u$. 
Und tats"achlich geht die Analogie weit (so sind Matrizen und unser Operator $A$ lineare Abbildungen). Insbesondere haben wir gesehen, 
dass Eigenvektoren f"ur Matrizen besonders wichtig sind. 
Das f"uhrt auf das Problem, f"ur unseren Operator $A$ Eigenfunktionen zu finden

\begin{sdefi} Eine Funktion $\varphi(x)\not = 0$, f"ur die gilt 
$$ A\varphi(x)  = \lambda \varphi(x)$$
und
$$ \varphi(0)=\varphi(L)=0$$
hei"ss{}t Eigenfunktion des Operators $A$
zum Eigenwert $\lambda$. 
\end{sdefi}


 Wir bestimmen die Eigenfunktionen zu $A$:\par\medskip
\fpbox{ Die reellen Eigenfunktionen/Eigenwerte von $A$ lauten
 $$ \varphi_n(x) = \sqrt{2}\, \sin(n\,\pi x/L), \quad
\quad\lambda_n =  \frac{- n^2\pi^2}{L^2}.$$
}
\par\medskip
{\bf Beweis:}
Zun"achst ist $\varphi''(x)=\lambda\varphi(x)$ f"ur gegebenes $\lambda$
eine lineare Differentialgleichung. Wir wissen
$$ \varphi(x) = C_1 e^{\sqrt{\lambda}\, x}+ C_2e^{-\sqrt{\lambda}\, x}.$$
wobei $C_1$, $C_2$ Konstanten sind, um $\varphi(0)=\varphi(L)=0$ zu erf"ullen.\\
$\bullet$ {\it Fall 1:} Falls $\lambda>0$, so haben wir f"ur $x=0$
$$ 0=\varphi(0)= C_1+C_2=0,$$
d.h. $C_2=-C_1$. Bei $x=L$ finden wir daher
$$ 0 = \varphi(L) = C_1 (e^{\sqrt{\lambda}\, L}-e^{-\sqrt{\lambda}\, L})$$
und da f"ur $L>0$, $\lambda>0$ folgt, dass
$e^{\sqrt{\lambda}\, L}-e^{-\sqrt{\lambda}\, L}\not = 0$, so haben wir
sofort $C_1=C_2=0$. Da eine Eigenfunktion (wie ein Eigenvektor) nicht
gleich Null sein darf, finden wir keinen Eigenfunktionen f"ur $\lambda>0$. \\
$\bullet$ {\it Fall 2:} $\lambda =0$. Dann ist $\varphi(x) = C_1+C_2 x $, und
die Randbedingungen ergeben wieder dass $\varphi(x) = 0$.\\
$\bullet$ {\it Fall 3:} $\lambda<0$. Wir schreiben $\lambda = - \omega^2$, und
nutzen das reelle Fundamentalsystem
$$ \varphi(x) = C_1 \sin(\omega x) + C_2 \cos(\omega x).$$
Wegen $\varphi(0) = 0$ finden wir sofort 
$$ 0 = C_1\, 0 + C_2\quad\Rightarrow C_2 = 0.$$
 Nun m"ussen wir noch $\varphi(L)=0$ erf"ullen, d.h. $C_1\sin(\omega L)=0$. Da $C_1$ nicht gleich Null sein darf (sonst w"are $\varphi(x)=0$), muss gelten
 $$ \sin(\omega L) = 0 \quad\Rightarrow \omega =\omega_n= \frac{n\pi}{L},\quad n=1,2,3,\cdots.$$
 Wir haben also ein System von Eigenfunktionen $\tilde \varphi_n(x) = 
 \sin(n\,\pi x/L)$. Aus Gr"unden, die erst sp"ater klar werden, normieren wir diese Funktionen, und w"ahlen als Ansatzfunktionen
 $$ \varphi_n(x) = \sqrt{2/L}\, \sin(n\,\pi x/L), \quad
 \omega_n = \frac{n\pi}{L}, \quad\lambda_n =  \frac{- n^2\pi^2}{L^2}.$$


\begin{sbem}
Man kann zeigen, dass auch in $\C\setminus\R$ keine Eigenwerte liegen. 
\"Ahnlich wie bei einer Matrix, haben wir also spezielle Werte erhalten,
die Eigenwerte sind. Bei einer $\R^{n\times n}$-Matrix bekommen wir h"ohstens $n$ Eigenwerte; da der Raum der Funktionen unendlichdimensional ist, erhalten wir hier unendlich viele Eigenwerte.
\end{sbem}

Definiere auf dem Raum der quadratintegrierbaren Funktionen 
das Skalarprodukt
$$ <f(x),g(x)> = \int_0^L f(x)\, g(x)\, dx.$$
Die Funktionen $\varphi_n(x)$ sind orthonormal, d.h. 
f"ur $n\not = m$ gilt
$$\int_0^L \varphi_n(x)\, \varphi_n(x)\, dx = 1,\quad
\int_0^L \varphi_n(x)\, \varphi_m(x)\, dx = 0.$$
Das ist nur elementares Integrieren, das zeigen wir hier nicht.\\
{\it Bem.: Man kann sogar zeigen, dass i einem geeigneten Sinne diese Funktionen eine Orthonormalbasis des Funktionenraums bilden, d.h. jede Funktion $g(x)$ sich durch 
$$ \sum_{i=0}^\infty c_n \varphi:n(x) = g(x)$$
f"ur geeignete $c_i\in\R$ darstellen l"asst.
}\par\bigskip


Nun m"ochten wir 
$$A u(x) = f(x)$$
approximativ/numerisch l"osen. Wir setzen an
$$ u(x) = \sum_{n=1}^\infty c_n \varphi_n(x).$$
Dann k"onnen wir $Au$ sofort ausrechnen (die $\varphi_n$ sind
ja Eigenfunktionen von $A$),
$$A u(x) 
=  A \sum_{n=1}^\infty c_n \varphi_n(x)
=   \sum_{n=1}^\infty c_n A\varphi_n(x)
=   \sum_{n=1}^\infty c_n  \frac{- n^2\pi^2}{L^2}\, \varphi_n(x)
$$
Also wird aus $Au(x) = f(x)$ die Gleichung
$$ \sum_{n=1}^\infty c_n  \frac{- n^2\pi^2}{L^2}\, \varphi_n(x) = f(x).$$
\fpbox{
Wie k"onnen wir dieses Gleichungssystem l"osen? Um $c_{i_0}$
zu erhalten multiplizieren wir die Gleichung mit $\varphi_{i_0}(x)$
und integrieren zwischen $0$ und $L$, 
$$  \sum_{n=1}^\infty c_n  \frac{- n^2\pi^2}{L^2}\, <\varphi_{i_0}(x)\, \varphi_n(x)> = <\varphi_{i_0}(x)\,,f(x)>.$$
Da nun $<\varphi_{i_0}(x)\, \varphi_n(x)>=0$ f"ur $n\not = i_0$ und 
$<\varphi_{i_0}(x)\, \varphi_n(x)> = 1$ f"ur $n=i_0$, reduziert sich die Gleichung auf
$$ c_{i_0}  \frac{- {i_0}^2\pi^2}{L^2}\,  = <\varphi_{i_0}(x)\,,f(x)>
$$
und also
$$ c_{i_0} =  \frac{-\,L^2\,\, <\varphi_{i_0}(x)\,,f(x)>}{{i_0}^2\pi^2}
.$$
Wir haben also auf der rechten Seite einfach ein Intregral auszuwerten,
bei dem der Integrand aus dem Produkt zweier bekannter Funktionen besteht. Das Problem ist viel einfacher zuu l"osen als eine PDE. \\
Eine numerische Approximation nutzt nun nur eine endliche Zahl von
Basisfunktionen (auch: ''Moden'' genannt\index{Moden}),
$$u(x)\approx \sum_{i=1}^N c_i \varphi_i(x).$$
}




\begin{bspX}
Wie bei den finiten Elementen w"ahlen wir
 $L=1$ und 
  $f(x) = 1$ falls $x\in (L/2-0.05, L/2+0.05)$, Null sonst. 
Wir sehen uns verschiedene Diskretisierungen an (Abb.~ \ref{numPDESpekExpl}) und finden ohne Abstrahlung ($\mu=0$) 
schon eine gute Konvergenz bei 4-8 Moden; mit Abstrahlung ($\mu=100$) 
ben"otigen wir wesentlich mehr Moden ($\approx 40)$. 

\end{bspX}

%===============================================================================

\begin{figure}[htbp]
 \centering
   \includegraphics[width=10cm]{../figures/diffFNumSpektral.pdf}
   \caption{Approximation der L"osung von $u''(x) = -f(x)$ (links)
   und $u''(x) = \mu u(x)-f(x)$ (rechts) durch die Spektralmethode. 
   Typisch ist die recht schnelle Konvergenz, aber auch die Neigung 
   zu r"aumlichen Schwingungen.}
   \label{numPDESpekExpl}
\end{figure}


%===============================================================================

\subsection*{Aufgaben}
\begin{auf}\chc\label{bloc11XA3}
\input{../../Aufgabensammlung/hmIIpde03.tex}
\end{auf}

\begin{auf}\chd\label{bloc11XA4}
\input{../../Aufgabensammlung/hmIIpde04.tex}
\end{auf}

%===============================================================================


\subsection{W"armeleitungsgleichung}

Nun wenden wir uns der W"armeleitungsgleichung 
$$ u_t(t,x) = u_{xx}(t,x)$$
zu.Dabei ist $t\in\R$, $t\geq 0$, die Zeit, und $x\in[0, L]$ der Raum.
Die Funktion ist zur Zeit $t=0$ durch $u_0(x)$ gegeben. Weiter seien die
Randbedingungen 
$$ u(t,0)=0,\qquad u(t,L)=0$$
vorgeschrieben. 
  Wir sehen uns drei numerische Methoden an.

\subsubsection{Linienmethode}

Die rechte Seite sit einfach ein Laplace-Operator. Wir k"onnen 
also den Raum $[0,L]$ wieder wie oben diskretisieren, 
$x_i = i\, h$, und auch die rechte Seite (die zweite Ableitung)
durch einen Differenzenquotient ausdr"ucken.
Sei
$$ u^i(t) = u(t,x_i)$$
so gilt 
$$ \frac{\partial}{\partial t}u(t,x) \approx h^{-2} (u(t,x-h)-2\, u(t,x)+ u(t,x+h))$$
Das k"onnen wir aber als 
System von gew"ohnlicher Differentialgleichungen schrieben
$$ \frac d {dt} u^i(t) = h^{-2} (u^{i-1}(t) - 2 u^i(t) + u^{i+1}(t)).$$
Dieses System von gew"ohnlicher Differentialgleichungen
(wie sehen die Differentialgleichungen f"ur $i=1$ und $i=N-1$ aus?
Hinweis: Randwerte!) 
kann man mit Hilfe z.B. des expliziten oder auch impliziten Euler-Verfahrens numerisch l"osen.

\subsubsection{Finite Differenzen}
Gehen wir weiter und nutzen das explizite Euler-Verfahren um die 
Gew"ohnlichen Differentialgleichungen, die wir eben erhalten haben
auch in der Zeit zu diskretisieren, so finden wir bei einer Zeit-Schrittweite von $\delta$
$$ u^i(t+\delta) = u^i(t)+ \frac{\delta}{h^2}(u^{i-1}(t) - 2 u^i(t) + u^{i+1}(t)).$$
Definieren wir $u^i_n = u^i(n\delta)$, so erhalten wir 
$$ u^i_{n+1} = u^i_n + 
\frac{\delta}{h^2}(u^{i-1}_n - 2 u^i_n + u^{i+1}_n)
.$$
Das ist die finite Differenzenmethode f"ur die W"armeleitungsgleichung.


\subsubsection{Spektralmethode}

Nun erinnern wir uns, dass wir oben Funktionen gefunden haben, f"ur 
die 
$$ A\varphi_n(x) = \varphi_n''(x) =\lambda_n \varphi(x)$$
gilt. Wenn wir definieren
$$\psi_n(t,x) = e^{\lambda_n t}\,\varphi_n(x) $$
so best"atigen wir leicht, dass 
$$ (\psi_n)_t(t,x) = (\psi_n)_{xx}(t,x)$$
d.h., dass diese Funktionen L"osungen der W"armeleitungsgleichung sind, die auch
die Randbedingungen $\psi(t,0)=0=\psi_n(t,L)$ erf"ullen. 
Wenn wir Konstanten $c_i$ finden, sodass
$$  \sum_{n=0}^\infty c_n \psi_n(0,x) = u_0(x)$$
(Erinnerung: $u(0,x)=u_0(x)$), so ist die L"osung
$$ u(t,x) = \sum_{n=0}^\infty c_n \psi_n(t,x)$$
bestimmt. Das Problem, die Koeffizienten $c_n$ zu bestimmen
haben wir aber schon oben, als wir die elliptische PDE
betrachteten, gel"ost -- wenn wir $\psi_n(t,x) = e^{\lambda_n t}\,\varphi_n(x) $nutzen, so lautet das Problem n"amlich
$$  \sum_{n=0}^\infty c_n \varphi_n(x) = u_0(x)$$
und also 
$$ c_{i_0} =  \frac{-\,L^2\,\, <\varphi_{i_0}(x)\,,u_0(x)>}{{i_0}^2\pi^2}
.$$


