 

%% ! T E X  root=hmIIneu.tex
% !TEX root=MA9603.WZW.tex
% !TEX program = pdflatex
% !TEX spellcheck = de_DE



%===============================================================================
\section{Populationsmodelle und Qualitative Theorie}
%===============================================================================
\zbox{
{\bf Ziele}:
\begin{itemize}
\item Einfache Modelle aufstellen und erkl"aren k"onnen
\item Station"are Punkte bestimmen k"onnen
\item Stabilit"at station"arer Punkte bestimmen k"onnen
\item Satz von Hartmann-Grobmann kennen und anwenden k"onnen
\end{itemize}}
%=============================================================================== 
\subsection{Logistisches Wachstum, Chemostat, Kompetitionsmodell}
In diesem Kapitelchen behandeln wir die Grundlagen der Modellbildung. 

\begin{sbem} Grundprinzipien bei der Modellbildung:\\
{\rm (1) Zustand des Systems beschreiben.} Welches sind die absolut n"otigen 
Variablen, um das System zu beschreiben (Konzentrationen, Populationsgr"o"se, 
Ort, Geschwindigkeit, Druck,...)\\
{\rm (2) Dynamik des Systems beschreiben.} Welche Regeln beschreiben die 
Ver"anderung des Systems?\\
{\rm (3)  KISS} (keep it simple and stupid). Nur die wirklich wichtigsten 
Zustandsgr"o"sen bzw. Regeln nutzen!!!
\end{sbem}

\begin{bspX}  {\bf Exponentielles Wachstum}. Ohne Kompetition ist der Zuwachs 
von Bakterien proportional zur Menge der Bakterien. Betrachten
wir Bakterien in einem geschlossenen Reaktionsgef"a"s (``Batch-Culture''). \\
{\bf Zustand:} $x(t)$  Konzentration von Bakterien zur Zeit $t$.\\
{\bf Dynamik:} Die Bakterien teilen sich im Schnitt mit Rate $\beta$.\\
{\bf Gleichung:} 
$$ x' = \beta x.$$
\end{bspX}
{\bf Exkurs: Einheiten und Skala des Systems}\par
Die verschiedenen Gr"o"sen tragen Einheiten. Wir finden 
[x] = Bakterien/Liter, und \"Anderung/Zuwachs von $x$ gegeben durch 
$[\beta x]$=Bakterien/(l s), i.e.\ $[\beta]$=1/s. Da die gleiche Einheit 
auf beiden Seiten des Gleichheitszeichens stehen muss, sehen wir uns noch
die Einheit auf der linken Seite an: $[x'(t)] = [x(t+h)-x(t)/(t+h-t)]$=
Bakterien/(l s), d.h. die Ableitung selbst tr"agt auch eine Einheit (sieht man 
am Differenzenquotienten, oder auch, da die Ableitung die Steigung, 
d.h.\ "Anderung von $x$ pro Zeitintervall angibt). I.e., $[d/dt]$=1/s.\par
Wir m"ussen uns auf die richtigen Skalen f"ur das System einigen. Wenn sich 
die "Anderung der Population nach Stunden zeigt, dann m"ussen wir alle 
Zeiteinheiten in Stunden angeben, um konsistent zu sein. Z.B. $\beta=0.2$/min,
so
$$ x' = \beta x = \frac{0.2}{\mbox{min}}\,\, x 
= \frac{0.2}{h}\,\,\frac h{\mbox{min}}\,\, x
= \frac{0.2}{h}\,\,\frac{60 \,\,\mbox{min}}{\mbox{min}}\,\, x
= 12/h\,\,\, x.$$
Was man macht, wenn verschiedene Prozesse des Systems verschiedene, 
 inkompatible Zeitskalen besitzen, sehen wir uns in der n"achsten 
Lehreinheit an.
\par{\bf Exkurs Ende.}\par\medskip

\begin{bspX}  {\bf Logistisches Wachstum}. Bei exponentiellem Wachstum mit 
 $x(0)=x_0$, erhalten wir die L"osung $x(t) = x_0 e^{\beta t}$, d.h. 
exponentielles Wachstum. das ist nicht m"oglich  - irgendwann sind die Ressourcen ersch"opft. 
Man kann das so formulieren, dass
das Wachstum sich bei gro"ser Population durch Kompetition verlangsamt.\\
{\bf Zustand:} $x(t)$ [Bakterien/Liter] Konzentration von Bakterien zur Zeit $t$\\
{\bf Dynamik:} Die Bakterien teilen sich im Schnitt mit Rate $\beta\, (1-x/K)$ [1/s].\\
 $K$ [Bakterien/Liter] ist die Kapazit"at des Systems.\\
{\bf Gleichung:} 
$$ x' = \beta\, (1-x/K) x.$$
\end{bspX}

\begin{bspX}  {\bf Kompetitions-Modell}. Nun sehen  wir uns zwei Bakterien-Arten an, die beide einem logistischen Wachstum folgen. Weiter fressen sie sich gegenseitig die Nahrung weg. Dieses ``weg-fressen'' formulieren wir wieder in 
einer verminderten Reproduktionsrate, analog zum logistischen Wachstum.\\
{\bf Zustand:} $x(t)$, $y(t)$ [Bakterien/Liter] Konzentration von Bakterien der Art $x$ bzw. $y$.\\
{\bf Dynamik:} Die Bakterien der Art $x$ teilen sich im Schnitt mit Rate $\beta_1\, (1-x/K_1-a y)$ [1/s], 
die der Art $y$ mir Rate  $\beta_2\, (1-y/K_2-b x)$ [1/s].\\
{\bf Gleichung:} 
\begin{eqnarray*}
x' & = & \beta_1 x (1-x/K_1-a y)\\
y' & = & \beta_2 y (1-y/K_2-b x)\\
\end{eqnarray*}
\end{bspX}

\begin{bspX}  {\bf Chemostat}. Der Chemostat ist ein Ger"at zur Durchf"uhrung 
biologischer Experimente. Er findet auch als industrielle Produktionsanlage zur 
Gewinnung von Chemikalien durch Bakterien Anwendung. Im einfachsten Fall besteht 
ein Chemostat aus einem Reaktionsgef"a"s, 
einem Zu- und einem Abfluss. Im Zufluss 
wird N"ahrstoff zugeleitet, das Fl"ussigkeitsvolumen im Reaktionsgef"a"s 
wird dadurch 
konstant gehalten, dass im Abfluss mit der gleichen Rate, mit der 
 Fl"ussigkeit/N"ahrstoff 
eingeleitet wird, auch Medium wieder entnommen wird. Das Medium im Reaktionsger"at 
wird gut ger"uhrt (Sauerstoffversorgung, gleichm"a"sige Verteilung), und es leben 
Bakterien im Reaktionsgef"a"s, die N"ahrstoffe fressen, sich vermehren, und 
herausgesp"ult werden.\\
{\bf Zustand:} $s(t)$ [mol/l] Substratkonzentration, $x(t)$ [Bakterien/l]: Bakterienkonzentration, jeweils im Reaktionsgef"a"s zur Zeit $t$.\\
{\bf Dynamik:} (a) Rate, mit der Fl"ussigkeit zugef"uhrt bzw. abgef"uhrt wird: D [l/h]. Substratkonzentration im Zufluss: $s_0$ [ml/l].\\
(b) Nahrungsaufnahme von Bakterien  : $f(s)x$, [mol/(l s)], wobei $f(s) = \mu_{max}\,s/(K+s)$.\\
(c) Rate, mit der sich die Bakterien vermehren: $\alpha\,f(s)x$, [Bakterien/(l s)],  $\alpha$: Umwandlungsfaktor Nahrung$\rightarrow$Biomasse.\\
{\bf Modellgleichungen:}\\
\begin{eqnarray*}
x' & = & \underbrace{-D x}_{\mbox{\footnotesize Ausschwemmen von Bakterien}} 
+  \underbrace{\alpha f(s) x}_{\mbox{\footnotesize Reproduktion von Bakterien}} \\
s'& =& 
 \underbrace{D s_0}_{\mbox{\footnotesize Einschwemmen von Nahrung}} 
-
\underbrace{D s}_{\mbox{\footnotesize Ausschwemmen von Nahrung}} 
-
 \underbrace{f(s) x}_{\mbox{\footnotesize Aufnahme von Nahrung durch Bakterien}} \\
\end{eqnarray*}
\end{bspX}

%===============================================================================
\subsection{Vektorfeld und Isoklinen}

Die Modelle, die wir oben eingef"uhrt haben, sind alle nicht-lineare 
Differentialgleichungen, entweder eindimensional,
$$ x' = f(x)$$
oder zweidimensional
\begin{eqnarray*}
x' & = & f(x,y)\\
y' &=& g(x,y)
\end{eqnarray*}
Interessiert ist man immer am langfristigen Verhalten. F"ur eine eindimensionale 
Differentialgleichung ist das einfach.
\begin{figure}
\begin{center}
\includegraphics[width=9cm]{../figures/mono.pdf}
\includegraphics[width=5cm]{../figures/hmI27Vector.pdf}
\end{center}
\caption{Links: Die L"osungen $x(t)$ von skalaren Differentialgleichungen erster Ordnung $x'=f(x)$ 
wachsen entweder monoton, oder sie fallen monoton, solange, bis sie auf einen Fixpunkt treffen (oder nach $\pm\infty$ davonlaufen). 
Rechts: Ein Vektorfeld $(f(x,y), g(x,y))$ wei"st jeden Punkt einen Vektor zu. D.h., wir k"onnen an jeden Punkt $(x,y)$ den entsprechenden Vektor heften. Die L"osungskurve $(x(t),y(t))$ der Differentialgleichung $x'=f(x,y)$, 
$y'=g(x,y)$ besitzt genau diese Vektoren als Tangentenvektoren.}\label{odeMono}
\end{figure}

\fpbox{Die L"osung $x(t)$ einer eindimensionalen, autonomen Differentialgleichung $x'=f(x)$ 
ist immer monoton (fallend oder wachsend, 
Abb.~\ref{odeMono}, links). Sei $f(x)$ differenzierbar. 
Das langfristige Verhalten ist 
gegeben durch den Limes
$$ \lim_{t\rightarrow\infty} x(t).$$
F"ur die L"osung einer eindimensionalen, autonomen Differentialgleichung 
divergiert der Limes entweder (``geht gegen $\pm\infty$'') oder existiert 
und geht gegen einen konstanten Wert $x_0 =  \lim_{t\rightarrow\infty} x(t).$
In diesem Fall ist $f(x_0)=0$, d.h. die Konstante $x_0$ ist selbst schon 
L"osung der Differentialgleichung.
}

Wenn wir zwei oder mehr Dimensionen haben, ist die Analyse wesentlich schwieriger. 
Hier ben"otigen wir einige Begriffe und Werkzeuge. 

\begin{sdefi} (a) Ein Vektorfeld ist eine Abbildung 
$F:\R^2\rightarrow\R^2$, $(x,y)\mapsto(f(x,y), g(x,y))$  
(oder allgemeiner, $F:\R^n\rightarrow\R^n$). Ein Vektorfeld definiert eine
Differentialgleichung $x' = F(x)$ (Abb.~\ref{odeMono}, rechts). \\
(b) Die $x$-Isokline (oder $x$-Nullisokline) eines Vektorfelds $(f(x,y), g(x,y))$ 
ist definiert als
$$\{(x,y)\,|\, f(x,y)=0\}$$
die Menge der Punkte, in denen die $x$-Komponente des Vektorfelds Null ist. 
Analog ist die $y$-Isokline durch $g(x,y)=0$ definiert.\\
{\bf Bemerkung:}  Station"are Punkte eines Vektorfelds, gegeben durch $F(x)=0$, 
 sind das genau die  Schnittpunkte der Isoklinen.
\end{sdefi}

Oft kann man aus der Lage der Isoklinen schon sehr viel ablesen. 

\begin{figure}[htb]
\begin{center}
\includegraphics[width=10cm]{../figures/hmIIblock06Chemostat}
\end{center}
\caption{Isoklinen, Vektorfeld und station"are Punkte 
des Chemostats im Fall dass 
$s_0$ gr"o"ser als die L"osung von $f(s)=D/\alpha$ ist.}
\label{chemostat}
\end{figure}
%===============================================================================
\begin{bspX} Station"are Punkte des Chemostats\\
\begin{eqnarray*}
x' & = & -Dx+\alpha f(s) x\\
s' & = & Ds_0 - Ds - f(s) x
\end{eqnarray*}
$$ x'=0\quad\Rightarrow x=0 \mbox{ oder } f(s)=D/\alpha.$$
Fall 1: $x=0$. $y'=0$ f"uhrt auf $(x,s)=(0,s_0)$ als station"aren Punkt (triviale L"osung, keine Bakterien vorhanden).\\
Fall 2: $x\not=0$; dann suche $s^*$, sodass $f(s^*)=D/\alpha$. Aus $s'=0$ 
folgt dann $x=x^*=D(S_0-s^*)/f(s^*).$\par
Damit der nicht-triviale station"are Punkt $(x,s)=(x^*,s^*)$ existiert 
und $x^*>0$, muss $s^*<s_0$ sein.\par
Isokline des Chemostats
\underline{x-Isokline:} $x'=0$, i.e.
$$ 0 = -Dx+\alpha f(s) x
= x[\alpha f(s)-D].$$
Also sind die $x$-Isoklinen gegeben durch $x=0$ und $f(s) = D/\alpha$.\\
\underline{y-Isokline} (besser: $s$-Isokline): $s'=0$, i.e.
$$ 0 = Ds_0 - Ds - f(s) x.$$
Also ist die $y$-Isokline gegeben durch $x = D(s_0-s)/f(s)$.\\
Wir lesen sofort ab: es gibt einen station"aren Punkt mit $x>0$ genau dann, wenn 
die Gleichung $f(s)=D/\alpha$ eine L"osung $s\in(0,s_0)$ besitzt.\par\medskip
\noindent
\underline{Interpretation:} Wenn $\alpha f(s)=D$, dann balanciert sich das 
Ausschemmen der Mikroorganismen (Rate $D$) und das Wachstum der Mikroorganismen 
(Rate $\alpha f(s)$). Die Populationsgr"o"se 
der Mikroorganismen $x$ pendelt sich 
genau auf diesen Wert ein. Ist $s_0$ zu klein, so kann das Wachstum -- selbst 
bei kleinen Populationsgr"o"se $x$ -- das Auswaschen nicht kompensieren, und die 
Population stirbt aus. Es gibt nur den station"aren Punkt $x=0$, $s=s_0$. 
\par\medskip
\noindent
\underline{Langfristiges Verhalten:} 
Wir k"onnen noch mehr ablesen: wir kennen dir Richtung des Vektorfelds auf den 
Isoklinen, und k"onnen daher auch die Richtung des Vektorfelds in den Fl"achen
zwischen den Isoklinen qualitativ angeben (sowas wie ``x w"achst, s f"allt''). 
Damit ist es oft schon m"oglich, sich einen groben "uberblick "uber das langfristige 
Verhalten zu verschaffen. In unserem Fall k"onnen
wir erraten, dass die L"osungen gegen den station"aren Punkt mit $x>0$ 
laufen werden (Fig.~\ref{chemostat}).
\end{bspX}

\begin{auf}\chd\label{block6A1}
\input{../../Aufgabensammlung/hm702.tex}
\end{auf}

%===============================================================================
\subsection{Station"are Punkte, Satz von Hartman-Grobman und Stabilit"at}
Station"are Punkte spielen also eine wichtige Rolle; einerseits sind sie
gute Kandidaten f"ur das langfristige Verhalten einer L"osung, andererseits
 strukturieren sie die Dynamik. Und, man kann station"are Punkte 
in der Regel noch recht 
gut ausrechnen. Um die Dynamik in der N"ahe der station"aren Punkte besser
zu verstehen, macht man folgende, formale Rechnung.

{\bf  Dynamik in der N"ahe der station"aren Punkte.} Gegeben sei die  
\begin{eqnarray*}
x' = f(x,y)\\
y' = g(x,y)
\end{eqnarray*}
und
$$ f(x_0,y_0)=g(x_0,y_0)=0.$$
Um die L"osung nahe des station"aren Punktes zu untersuchen, nutzen wir diesen als 
Bezugspunkt, und r"ucken ihn in den Nullpunkt des Koordinatensystems. Wir definieren 
neue Koordinaten $u$, $v$ durch:
$$ x=x_0+u,\qquad y = y_0+v.$$
Dann finden wir via Taylor-Entwicklung 
$$ u' = (x_0+u)' = x' = f(x,y) = f(x_0+u,y_0+v)
= f(x_0,y_0)+a u+ bv + \mbox{ quadratische Terme}
$$
wobei 
$$ a=\frac{\partial f(x_0,y_0)}{\partial x},\quad b=\frac{\partial f(x_0,y_0)}{\partial y}.$$
Nun ist $f(x_0,y_0)=0$ (da $(x_0,y_0)$ station"arer Punkt), und die quadratischen 
Terme sind f"ur $u$, $v$ klein vernachl"assigbar. D.h., wir haben
approximativ
$$ u' \approx \frac{\partial f(x_0,y_0)}{\partial x}\, u+
\frac{\partial f(x_0,y_0)}{\partial y} v.$$
Analog erhalten wir approximativ
$$ v' \approx \frac{\partial g(x_0,y_0)}{\partial x}\, u+
\frac{\partial g(x_0,y_0)}{\partial y} v.$$
D.h.,
$$\left(\begin{array}{c}
u\\v
\end{array}
\right)'
= A \left(\begin{array}{c}
u\\v
\end{array}
\right)$$
wobei
$$ A = 
\left(\begin{array}{cc}
 \frac{\partial f(x_0,y_0)}{\partial x}\,&
\frac{\partial f(x_0,y_0)}{\partial y}\\
\frac{\partial g(x_0,y_0)}{\partial x}\, &
\frac{\partial g(x_0,y_0)}{\partial y} 
\end{array}
\right) =: \frac{\partial f}{\partial (x,y)}$$
die \emph{Jacobi-Matrix} bedeutet. D.h., nahe des station"aren Punktes kann man das 
nichtlineare Vektorfeld sehr gut durch ein lineares Vektorfeld ersetzen. 
Die L"osungen linearer Vektorfelder sind uns sehr gut durch den ``Zoo'' bekannt. 

%===============================================================================
\begin{satz} \label{hartGrob}(Hartman-Grobman) Gegeben sei in nicht-lineares Vektorfeld 
$F:\R^n\rightarrow \R^n$, und ein station"arer Punkt $x_0\in\R^n$, $F(x_0)=0$. 
Sei $A$ die Jacobi-Matrix an $x_0$, d.h.
$$ ((A))_{i,j} = \frac{\partial (F)_i}{\partial x_j}(x_0).$$
Gelte weiter, dass alle Eigenwerte von $A$ einen Realteil ungleich Null besitzen. 
Dann sehen sich die L"osungen des nichtlinearen Systems nahe $x_0$ 
denen des linearen Systems $y'=Ay$ bei $y=0$ "ahnlich.
\end{satz}

Insbesondere gilt die wichtige Folgerung aus dem Satz.
\begin{satz} Seien die Voraussetzungen wie im Satz von Hartman-Grobman. 
Falls der Realteil aller Eigenwerte der Jacobi-Matrix strikt kleiner Null
sind, so ist der Limes von L"osungen, die nahe genug bei dem station"aren Punkt 
starten, der station"are Punkt selbst.
\end{satz}
Dies ist eine direkt Folgerung aus der Tatsache, dass die L"osung von $x'=Ax$
gegen Null geht wenn die Realteile aller Eigenwerte von $A$ kleiner Null sind. 
Diese Eigenschaft gibt Anlass zu der folgenden Definition.

\begin{sdefi} Sei $x'=F(x)$ gegeben, $F(x_0)=0$ ein station"arer Punkt. Dieser
Punkt hei"st lokal asymptotisch stabil, falls aus $x(t_0)-x_0$ klein genug folgt,
dass
$$ \lim_{t\rightarrow\infty} x(t) = x_0.$$
Sonst hei"st der Punkt instabil.\\
Der Punkt hei"st linear stabil, falls die Jacobi-Matrix nur Eigenwerte mit 
negativen Realteilen besitzt.  
\end{sdefi}
Also folgt aus ``linear stabil'', dass ein Punkt lokal asymptotisch stabil ist. 

%===============================================================================
\begin{bspX} 
Falls $s_0$ gro"s genug ist, finden wir 
zwei station"are Punkte im Modell des Chemostats. Wir sehen uns die Stabilit"at der station"aren Punktes an. 

\noindent{\bf Jacobi-Matrix.} Wir leiten erst $f(x,s) = -Dx+\alpha f(s) x$ 
nach $x$, dann nach $s$ ab, um die erste Zeile der Jacobi-Matrix zu bestimmen. 
Dann leiten wir $g(x,y) = Ds_0-Ds-f(s)x$ wieder erst nach $x$ dann nach $s$ ab, 
um die zweite Zeile der Jacobi-Matrix zu bestimmen. Wir erhalten
$$ A(x,s) = \left(\begin{array}{cc}
-D+\alpha f(s) & \alpha f'(s)x\\
-f(s) & -D-f'(s)x
\end{array}\right)$$

\noindent{\bf Erster station"arer Punkt.} Jetzt setzen wir  
$(x,s)=(0,s_0)$ in die Jacobi-Matrix ein: 
$$A(0,s_0) = 
\left(\begin{array}{cc}
-D+\alpha f(s_0) & 0\\
-f(s_0) & -D
\end{array}\right)
$$
Das ist eine Dreiecksmatrix. Wir wissen, dass wir die Eigenwerte in diesem Fall 
direkt von der Diagonalen ablesen k"onnen. D.h., wir finden die Eigenwerte
$$ \lambda_1 = -D<0,\quad \lambda_2 = -D+\alpha f(s_0).$$
Da wir momentan voraussetzen, dass ein station"arer Punkt mit der Eigenschaft
$f(s^*)=D/\alpha$, $s^*<s_0$ existiert, und $f(s)$ monoton steigt, 
folgern wir $\lambda_2>0$. Also ist dieser station"are Punkt instabil. 
Wenn wir wenige Mikroorganismen in den Chemostat einbringen, 
werden sich die Mikroorganismen ausbreiten. 

\noindent{\bf Zweiter station"arer Punkt.}
Jetzt setzen wir $(x,s)=(x^*,s^*)$ ein, wobei $f(s^*)=D/\alpha$ und $x^*
>0$.
$$ A(x^*,s^*) = \left(\begin{array}{cc}
-D+\alpha f(s^*) & \alpha f'(s^*)x^*\\
-f(s^*) & -D-f'(s^*)x^*
\end{array}\right)
=
\left(\begin{array}{cc}
0 & \alpha f'(s^*)x^*\\
-f(s^*) & -D-f'(s^*)x^*
\end{array}\right).
$$
Wir finden
$$ \mbox{sp}(A)=-D-f'(s^*)x^*<0,\qquad
\mbox{det}(A) = -f(s^*) \alpha f'(s^*)x^*<0.$$
Daher (siehe Kapitel 2.3.2) besitzen alle Eigenwerte negativen Realteil, und
 die nicht-triviale station"are L"osung ist lokal asymptotisch stabil.
\end{bspX}
Wir haben also unsere Vermutung von oben, dass sich das System langfristig auf 
den station"aren Punkt einspielt, best"arkt. Wir wissen zumindest, dass wir -- 
wenn wir in der N"ahe des station"aren Punktes starten -- uns auf den station"aren 
Punkt wieder setzen werden. Wir k"onnen nicht ausschlie"sen (das kann man beweisen, 
aber man ben"otigt daf"ur weitere Hilfsmittel), 
dass jeder Anfangswert mit $x(0)>0$ 
und $s(0)\geq 0$ sich auf diesen station"aren Zustand einspielt. 
%===============================================================================
\subsection*{Aufgaben}
\begin{auf}\chc\label{block6A2}
\input{../../Aufgabensammlung/hm703.tex}
\end{auf}

\begin{auf}\chc\label{block6A3}
\input{../../Aufgabensammlung/hm704.tex}
\end{auf}

\begin{auf}\chc\label{block6A4}
\input{../../Aufgabensammlung/hm705.tex}
\end{auf}

\begin{auf}\chd\label{block6A5}
\input{../../Aufgabensammlung/hm706.tex}
\end{auf}

\begin{auf}\chc\label{block6A6}
\input{../../Aufgabensammlung/hm707.tex}
\end{auf}

\begin{auf}\chb\label{block6A7}
\input{../../Aufgabensammlung/hm745.tex}
\end{auf}

