 

%% ! T E X  root=hmIIneu.tex
% !TEX root=MA9603.WZW.tex
% !TEX program = pdflatex
% !TEX spellcheck = de_DE


%===============================================================================
\section{Die Navier-Stokes-Gleichungen - partielle Differentialgleichungen
der Str"omungsmechanik}
%===============================================================================

\zbox{
{\bf Ziele}:
\begin{itemize}
\item Herleitung der Navier-Stokes-Gleichung erkennen
\item Laminarer Fluss in einer R"ohre berechnen k"onnen
\end{itemize}}

Die Navier-Stokes-Gleichungen sind f"ur die Fluidmechanik was das Newtonsche
Grundgesetz  (Kraft gleich Masse mal Beschleunigung) f"ur die klassische 
Mechanik ist. Tats"achlich beruhen die Navier-Stokes-Gleichungen im Wesentlichen 
genau auf diesem Gesetz, das f"ur Fl"ussigkeiten entsprechend adaptiert wird. 
Entwickelt wurden die Gleichungen von 
Claude Lewis M.\ H.\ Navier (1785-1836) und Sir Georg Gabriel Stokes 
(1819-1903). Zus"atzlich zum Grundgesetz der Mechanik muss vor allem noch der
Masse-Erhalt formuliert werden. Wir betrachten den wichtigsten Spezialfall: die 
inkompressiblen Fl"ussigkeiten. 
%===============================================================================
\subsection{Eulersche und Lagrangesche Koordinaten}
%===============================================================================
In diesem und den folgenden 
Paragraphen entwickeln wir einige Grundlagen, die wir f"ur die 
 Navier-Stokes-Gleichungen ben"otigen. Die Erste betrifft zwei verschiedene 
Bezugssysteme: wir k"onnen in ein festes Koordinatensystem gehen (Eulersche 
 Koordinaten), oder aber ein Koordinatensystem w"ahlen, das mit dem Fluss 
 mitbewegt wird (Lagrangsche Koordinaten). 
%===============================================================================
Nehmen wir einen kleinen, dreidimensionalen Kasten. 

In der \underline{Euler-Koordinaten} sitzt dieser Kasten
an einem festen Punkt im Raum. Sein Inhalt "andert sich durch die hineinflie"sende 
und die hinausflie"sende Fl"ussigkeit. Zus"atzlich m"ogen im Kasten noch physikalische 
oder chemische Vorg"ange die Fl"ussigkeit "andern
(``intrinsische "Anderungen'').

In den \underline{Lagrange-Koordinaten} wird der Kasten mit der Fl"ussigkeit mitbewegt.
 D.h., es wird keine Fl"ussigkeit hineinflie"sen oder  hinausflie"sen, 
sondern wir haben ausschlie"slich die intrinsischen "Anderungen 
zu betrachten. 
%===============================================================================
 \index{Lagrange--Koordinaten}  \index{Euler--Koordinaten Koordinaten} 
Wir "ubersetzen die Idee in Mathematik. 

\fpbox{
Nehmen wir an, dass wir 
das Geschwindigkeitsfeld $u(x,t)$ schon kennen (es ist eigentlich das Ziel,
genau dieses Geschwindigkeitsfeld zu berechnen). Wir wissen also, dass ein
Partikel am Ort $x\in\R^3$ zur Zeit $t$ sich in Richtung $u(x,t)\in\R^3$ 
bewegt. Nehmen wir weiter an, dass sich dieses Partikel zur Zeit $t=0$ am Ort $X_0\in\R^3$ 
 befand. Dann wissen wir, dass die Trajektorie (Bahn) des Partikels gegeben 
ist durch die Differentialgleichung
$$ \dot x = u(x,t),\qquad x(0) = X_0.$$
Sei nun 
$$ \Phi(t;X_0) = x(t)$$
der Ort eines Partikels zur Zeit $t$, welches sich zur Zeit $t=0$ am Ort $X_0$ 
befand. Dann wissen wir (s.o.), dass
 $$\frac d {dt} \Phi(t;X_0) = u(\Phi(t;X_0), t).$$
Damit k"onne wir also einen Ort im dreidimensionalen Raum zur Zeit $t$ dadurch 
kennzeichnen, dass wir ihn direkt angeben (Euler'sches Bild, $(x,t)$) 
oder aber sagen, wo der Ort zur Zeit $t=0$ war (Lagrange'sches Bild, 
$(X_0, t)$). Die beiden Bilder h"angen zusammen durch die Funktion $\Phi(t;X_0)$, 
$$ (x,t) = (\Phi(t;X_0),t).$$}
\par\medskip
Der n"achste Schritt besteht darin, eine Funktion in Euler- bzw. Lagrange-Koordinaten 
auszudr"ucken. \par\medskip
%===============================================================================
\fpbox{
Nun betrachten wir eine Funktion $f$ von Ort und Zeit (z.B. die Partikeldichte), 
d.h. $f:\R^3\times \R_+\rightarrow\R$, $(x,t)\mapsto f(x,t)$. Hier geben wir
Euler-Koordinaten an, d.h.\ wir fixieren einen Punkt in einem absoluten 
Koordinatensystem $(x,t)$ und geben daf"ur den Wert der Funktion an.\\
Wir k"onnen auch Lagrange-Koordinaten nutzen. In diesem Fall haben wir eine
andere Darstellung der  (gleichen(!)) Funktion;  als Argument nutzen wir nun den 
Ort $X_0$ zur Zeit $t=0$ und die Zeit $t$, $\tilde f:\R^3\times \R_+\rightarrow\R$, 
$(X_0,t)\mapsto \tilde f(X_0,t)$. 

Die Beziehung zwischen diesen beiden Darstellungen der Funktion lautet
$$ \tilde f(X_0,t) = f(\Phi(X_0,t), t).$$
Wichtig wird nun die Beziehung der zeitlichen Ableitung der Funktion in den 
beiden Bildern:
\parbox{\textwidth}{
$$ \frac {d}{dt}\tilde f(X_0, t) = \frac {d}{dt} f(\Phi(t;X_0), t) = \frac{\partial}{\partial t}f(\Phi(t; X_0),t) + \nabla\bigg( f(\Phi(t;X_0),t) \, u(\Phi(t,X_0), t)\bigg).$$
}
}
%===============================================================================
\subsection{Erhaltung der Masse} 
{\bf Massenerhalt in Euler-Koordinaten.} Betrachte ein Gebiet $V$ im 
dreidimensionalem Raum mit glatten Rand $\partial V$. 
Sei  $\rho(x,t)$ die Massendichte der Fl"ussigkeit am Ort $x$ und Zeit $t$
Wir werden uns auf inkompressible Fl"ussigkeiten beschr"anken, da ist $\rho(x,t)$ 
konstant; aber lassen wir diese Vereinfachung zun"achst mal au"ser Acht, und sehen, 
welche Gleichung uns der Massenerhalt im Allgemeinen f"ur $\rho(x,t)$ liefert.  
Die Masse, die im Volumen $V$ enthalten ist, betr"agt
$$ m(t) = \int_V \rho(x,t)\, dt.$$
Da wir Massenerhalt haben, kann sich die Masse in $V$ nur durch den Fluss von
Partikeln durch die Oberfl"ache  $\partial V$ des Volumens "andern. 
Die Massendichte bewegt sich nun entlang des Geschwindigkeitsfeld $u(x,t)$, 
d.h. der Massenfluss ist gegeben durch 
$$ \mbox{Fluss} = j(x,t) = \rho(x,t)\,u(x,t). $$
Sei $n(x,t)$ die "au"sere Normale in $x\in\partial V$. dann folgt also 
mit Hilfe des Gau"sschen Satzes
\begin{eqnarray*}
\frac{d}{dt} m(t) = \int_V\frac{\partial}{\partial t}\rho(x,t)\, dx = - \int_{\partial V} \rho(x,t) u(x,t)\,n(x,t) do 
= -\int_V\nabla\,j(x,t)\, dx
= -\int_V\nabla(\rho u)\, dx.
\end{eqnarray*}
Da das Volumen $V$ beliebig gew"ahlt war, m"ussen schon die Integranten gleich sein, 
und wir erhalten die Gleichung f"ur den Massenerhalt im Euler-Koordinaten (vergleiche auch mit dem ersten Fickschen Gesetz
 auf Seite~\pageref{fickschesXXGesetz})\\
\fpbox{
$$ \frac{\partial}{\partial t}\rho(x,t)  + \nabla(\rho(x,t) u(x,t)) = 0. $$
}

%===============================================================================
{\bf Massenerhalt in Lagrange-Koordinaten.} Um die Lagrange-Form des Massenerhalts 
herzuleiten, sei zun"achst bemerkt, dass 
wir in die Euler-Form $x=\Phi(X,t)$ setzen d"urfen. Also:
$$
 \frac{\partial}{\partial t}\rho(\Phi(X,t),t)  
+ \nabla(\rho(\Phi(X,t),t) u(\Phi(X,t),t),t) = 0. 
$$
Definieren wir $\tilde\rho(\Phi(X,t),t)$ 
als Massendichte in Lagrange-Koordinaten, so 
finden wir (siehe oben) 
\begin{eqnarray*}
d/dt\,\ \tilde \rho(X,t) & = & \frac{\partial}{\partial t}\rho(\Phi(t;X_0),t)+
\nabla \rho(\Phi(t;X_0),t) \, u(\Phi(t,X_0), t).
\end{eqnarray*}
Die rechte Seite ist Null (siehe die Gleichung zuvor), sodass wir die
Langrange-Form des Massenerhalts erhalten\par
\fpbox{
$$ \frac d {dt}\tilde\rho(X,t) = 0.$$
}
Das ist auch klar, so, da -- wenn keine Partikel erzeugt oder vernichtet werden 
sich nicht "andert -- die Partikeldichte am mitbewegten Punkt konstant bleiben muss.
%===============================================================================
\begin{sbem} Zusammenfassend haben wir die Euer-Form des Massenerhalts
$$ \frac{\partial}{\partial t}\rho(x,t)  + \nabla(\rho(x,t) u(x,t)) = 0. $$
und die Lagrange-Form des Massenerhalts
$$ \frac d {dt}\tilde\rho(X,t) = 0.$$
Eine inkompressible Fl"ussigkeit hat eine konstante Dichte, 
d.h.\ $\rho_t=0$ und 
$\nabla(\rho(x,t) u(x,t)) = \rho\nabla u(x,t)$. Aus der Eulerschen Form folgt 
dann, dass Massenerhalt "aquivalent mit
$$ \nabla u(x,t) = 0.$$
ist.
\end{sbem}
%===============================================================================
\subsection{Newtons Grundgleichung f"ur Str"omungsmechanik} 

Wieder betrachten wieder ein kleines Volumen $V$. Verschiedene Kr"afte
wirken auf die Masse im Volumens. Einige wirken auf die Masse durch die
Oberfl"ache (wie den Druck), andere wirken auf jedes Partikel in dem Volumen
(wie Gravitation oder elekromagnetische Kr"afte). Nach dem Grundgesetz der 
Mechanik balancieren sich diese Kr"afte mit der Massentr"agheit, d.h.\ 
beschleunigen die Masse. Das wiederum bedeutet eine "Anderung des 
Geschwindigkeitsfelds. In einer eher physikalisch als 
mathematisch inspirierten Schreibweise finden wir

$$ dF = dm \, \frac {du}{dt} = \rho\, dV\, \frac {du}{dt} \qquad \Rightarrow\qquad 
\frac{dF}{dV} = \rho  \frac {du}{dt}
$$
und mit $du/dt = \partial u/\partial t + (u\cdot \nabla) u$ erhalten wir
$$
\frac{dF}{dV} = \rho  \left(\frac{\partial u}{\partial t} + (u\cdot \nabla) u\right). 
$$
Sei $f = \partial F/\partial m$ die spezifischen Kr"afte (Kraft pro Masseneinheit). 
Dann, 
$$ \frac{dF}{dV} = f \frac{dm}{dV} = f \rho =  \rho 
  \left(\frac{\partial u}{\partial t} + (u\cdot \nabla) u\right). 
\qquad\Rightarrow\qquad
 \frac{\partial u}{\partial t} + (u\cdot \nabla) u = f.
 $$
Der Vektor $f$ summiert die intrinsischen als auch die extrinsischen Kr"afte.
Um die Gleichung zu vervollst"andigen, betrachten wir die wichtigsten Kr"afte.

{\bf (a)  Druck.} \index{Druck} 
$p(x,t)$ bezeichne der Druck. D.h., die Kraft, die auf das Volumen 
von Au"sen aufgrund der Fl"ussigkeit in den Nachbarvolumina wirkt. 
Diese Kraft ist gegeben durch ($n$ ist wieder die "au"sere Normale auf $\partial V$)
$$ \mbox{Kraft durch Druck auf }V =  \int_{\partial V} p \, n\, do = \int_V \nabla p\, dx
.$$
Die spezifische Kraft (Kraft pro Massedichte) lautet daher 
 $(\nabla p)/\rho$. Wenn dies die einzige zu ber"ucksichtigende Kraft ist, so 
 erhalten wir die sogenannten \underline{inkompressiblen Eulergleichungen} 
\index{Euler Gleichungen, inkompressiblen}\index{inkompressiblen Euler Gleichungen}\\
\fpbox{
$$  \frac{\partial u }{\partial t} + (u\,\cdot\,\nabla)\, u = \frac 1 \rho \nabla p,\qquad \nabla u = 0.$$
}\par\medskip
%===============================================================================
{\bf (b) Viskosit"at.} 
Die Molek"ule der Fl"ussigkeit folgen nicht wirklich genau dem Geschwindigkeitsfeld, 
sondern diese Bewegung wird "uberlagert durch eine
W"armebewegung (Brownsche Bewegung). Dieser Effekt ``transportiert'' 
oder ``verschmiert'' Geschwindigkeit.. Molek"ule, die von benachbarten Trajektorien 
stammen, vermischen sich. Dieses ``Vermischen'' wird gut durch 
Terme der W"armeleitungsgleichung. Die spezifische Kraft, die durch die Viskosit"at 
ausge"ubt wird, ist gegeben durch
$$ f = \nu \Delta u.$$
Alles in allem erhalten wir die Navier-Stokes Gleichung \\
\fpbox{
$$  \frac{\partial u }{\partial t} + (u\,\cdot\,\nabla)\, u+\nu\Delta u - \frac 1 \rho \nabla p = f,$$
}
wobei $f$ weitere Kr"afte (falls vorhanden) repr"asentiert. Diese 
Kr"afte k"onnen z.B. durch Gravitation oder Elektromagnetismus 
bewirkt werden. 

Zusammen mit der Gleichung
$$ \nabla u=0$$
f"ur den Massenerhalt inkompressibler Fl"ussigkeit bilden diese beiden Gleichungen 
die Grundgleichungen der Str"omungsmechanik.

%===============================================================================
\fpbox{{\bf Bemerkung.} Man kann die Navier-Stokes-Gleichungen auch "uber die 
Botzmann-Gleichungen herleiten. das ist insofern interessant, als hier
mikroskopische Eigenschaften (Bewegung einzelner Molek"ule) zu einer 
Markoskopischen Beschreibung wird 8navier-Stokes-Gleichung). Eine Andeutung des
Weges ist im Appendix~\ref{boltzxx} gegeben.}
%===============================================================================
\subsection{Einfaches Beispiel: Station"arer, laminarer Fluss durch eine  R"ohre}\index{Laminarer Fluss}
Wir\footnote{Dieser Abschnitt kann in der Vorlesung vermutlich nur verk"urzt besprochen werden} betrachten ein rundes Rohr in dem eine Fl"ussigkeit langsam flie"st. 
Diese Fl"ussigkeit sei inkompressibel. Wir verlangen, dass die Geschwindigkeit
am Rand der R"ohre Null ist (``no slip Randbedingungen''). \par

Wir sehen uns die einfachste L"osung
der Navier-Stokes Gleichung an; es gibt auch noch komplexere L"osungen -- 
insbesondere turbulente Fl"usse sind nicht mehr analytisch zu handhaben. 
In diesem Fall ist man auf numerische Verfahren angewiesen. Im Fall von langsamen 
Str"omungen findet man allerdings laminare Fl"usse. 

\noindent{\bf Schritt 1: Geometrie und Konsequenz der Symmetrie.} Die R"ohre sei gegeben 
durch
$$\Omega = \{(x,y,z)\,|\, x^2+y^2\leq R\},$$
d.h.\ das Rohr hat Radius $R$ und l"auft in Richtung der $z$-Achse. 
Wir sind an einem station"aren Fluss interessiert, d.h.\ einen zeitlich konstanten 
Fluss. Alle Gr"o"sen (Druck wie Geschwindigkeit) h"angen nicht von der Zeit ab. \\
Wir erwarten, dass der Fluss sich entlang der Symmetrie-Achse der R"ohre bewegt. 
Also,
$$ u(x,y,z) = 
\left(\begin{array}{c}
u_1(x,y,z)\\u_2(x,y,z)\\u_3(x,y,z)\\
\end{array}\right)
= 
\left(\begin{array}{c}
0\\0\\u_3(x,y,z)\\
\end{array}\right).
$$
Weiter erwarten wir (immer noch aus Symmetriegr"unden), dass die Geschwindigkeit 
unabh"angig von $z$ ist, und nur vom Abstand zum Mittelpunkt
der R"ohre abh"angt, d.h.
$$ u_3(x,y,z) = u_3(r),\qquad r^2 = x^2+y^2.$$
Insbesondere folgern wir 
 $$(u_3)_z=0.$$
Der Druck hingegen wird nur von $z$ abh"angen,
$$ p(x,y,z) = p(z).$$
\fpbox{
Bemerkungen: Diese Annahmen sind von der Physik (Symmetrie der Situation) her 
plausibel. Letztlich kommt die Berechtigung der Annahmen aber daher, dass
wir gleich eine L"osung der Navier-Stokes-Gleichungen finden werden, die 
diese Annahmen erf"ullen. Daher gibt es eine Str"omung, die diese Eigenschaften 
tats"achlich besitzt. }

\noindent{\bf Schritt 2: Navier-Stokes Gleichung.} Mit diesen Annahmen geht die 
Navier-Stokes-Gleichung (die ja eigentlich 
einen Vektor beschriebt, d.h.\ drei Gleichungen sind)
$$ \frac{\partial u }{\partial t} + (u\,\cdot\,\nabla)\, u+\nu\Delta u - \frac 1 \rho \nabla p = f=0
$$
"uber in eine Gleichung f"ur die Geschwindigkeitskomponente $u_3$,
$$ 
u_3(u_3)_z + \nu \Delta u_3 -  \frac 1 \rho p_z(z) =
 \nu \Delta u_3(r) -  \frac 1 \rho p_z(z) 
=0.
$$
Das hei"st also,
$$ \Delta u_3(r) = \frac 1{\nu\rho} p_z(z).$$
Das Produkt $\nu\rho$ ist konstant, da wir nur inkompressible Fl"ussigkeiten 
betrachten. Da die linke Seite nur von $r = \sqrt{x^2+y^2}$, die rechte Seite 
nur von $z$ abh"angt, kann diese Gleichung nur erf"ullt sein, wenn beide Seiten 
konstant sind:
\begin{eqnarray*}
\Delta u_3(r)  & = & c\\
p_z(z) & = & \nu\rho c.
\end{eqnarray*}
\noindent{\bf Schritt 3: Druck.} Die Gleichung f"ur den Druck ist recht einfach: da die 
Ableitung nach $z$ konstant ist, muss der Druck eine lineare Funktion von $z$ 
sein. Integration ergibt
$$ p(z) = p(z_0) + (z-z_0) \nu\rho c.$$
Oder, wenn wir den Druck an zwei Punkten $z_0$ und $z_1$ kennen ($p(z_0)$ und 
$p(z_1)$), muss gelten
$$ p(z) = p(z_0) + \frac{p(z_1)-p(z_0)}{z_1-z_0} (z-z_0).$$
Daraus k"onnen wir die Konstante $c$ bestimmen, 
$$ c =  \frac{p(z_1)-p(z_0)}{\nu\rho(z_1-z_0)}.$$
{\bf Schritt 4: Geschwindigkeit.} Die Geschwindigkeit 
ist schwerer zu erhalten. 
Dazu ben"otigen wir zun"achst die Darstellung des Laplace-Operators $\Delta$ auf 
eine radialsymmetrische Funktion.
%===============================================================================
\fpbox{{\bf Exkurs: Wirkung des Laplace auf eine radialsymmetrische 
Funktion in zwei Dimensionen.} Zun"achst sei bemerkt, dass
$$\frac{\partial}{\partial x}r 
= \frac{\partial}{\partial x} \sqrt{x^2+y^2}
= \frac{x}{ \sqrt{x^2+y^2}} = \frac x r,\qquad
\frac{\partial}{\partial y}r = \frac y r,
$$
und die zweiten Ableitungen sind gegeben durch
$$ \frac{\partial^2}{\partial x^2}r 
= \frac{\partial}{\partial x}\left(\frac{\partial}{\partial x}r \right)
= \frac{\partial}{\partial x}   \frac x r = \frac 1 r - \frac {x}{r^2}
\frac{\partial}{\partial x}r  = \frac 1 r - \frac {x^2}{r^3},\qquad
\frac{\partial^2}{\partial y^2}r = \frac 1 r - \frac {y^2}{r^3}
.$$

Sei $f(r) = f(\sqrt{x^2+y^2})$, so folgt
\begin{eqnarray*}
\frac{\partial}{\partial x} f(r)
& = & 
f'(\sqrt{x^2+y^2}) \frac{\partial}{\partial x}r  = 
f'(r)  \frac x r.\\
\frac{\partial^2}{\partial x^2} f(r)
& = & 
 \frac{\partial}{\partial x}\left(\frac{\partial}{\partial x} f(\sqrt{x^2+y^2}) \right)
=  \frac{\partial}{\partial x}\left(f'(r)  \frac x r\right)
=  f''(r)  \frac {x^2}{ r^2} + f'(r)\left(\frac 1 r - \frac {x^2}{r^3}\right)\\
\frac{\partial^2}{\partial y^2} f(r)
& = & 
 f''(r)  \frac {y^2}{ r^2} + f'(r)\left(\frac 1 r - \frac {y^2}{r^3}\right)
\end{eqnarray*}
und daher
\begin{eqnarray*}
\Delta f(r) & = &
 f''(r)  \frac {x^2}{ r^2} + f'(r)\left(\frac 1 r - \frac {x^2}{r^3}\right)
+ 
 f''(r)  \frac {y^2}{ r^2} + f'(r)\left(\frac 1 r - \frac {y^2}{r^3}\right)\\
&=& f''(r) \underbrace{\frac {x^2+y^2}{r^2}}_{=1} 
+ f'(r)
\left(\frac 2 r - \underbrace{\frac {x^2+y^2}{r^3}}_{=1/r}\right)
= f'(r) \,\,\frac 1 r +  f''(r)  
=  \frac 1 r \frac d {dr}\left(r \frac {d f(r)}{dr}\right)
\end{eqnarray*}
}
\par\medskip
Nutzen wir diese kompakte Darstellung des Laplace-Operators, so lautet unsere 
Gleichung f"ur $u_3$
$$ \Delta u_3(\sqrt{x^2+y^2}) = (\partial_x^2+\partial_y^2) u_3(\sqrt{x^2+y^2}) 
= \frac 1 r \frac d {dr}\left(r \frac {d u_3(r)}{dr}\right) = c.$$
Also haben wir wieder eine gew"ohnliche Differentialgleichung gefunden.
Die k"onnen wir l"osen:
\begin{eqnarray*}
\frac 1 r \frac d {dr}\left(r \frac {d u_3(r)}{dr}\right) & = & c\\
\Rightarrow\quad
\frac d {dr}\left(r \frac {d u_3(r)}{dr}\right) & = & cr\\
\Rightarrow\quad
r \frac {d u_3(r)}{dr} & = &\frac 1 2 cr^2+A\\
\Rightarrow\quad
\frac {d u_3(r)}{dr} & = & \frac 1 2cr+\frac A r\\
\Rightarrow\quad
u_3(r) & = & \frac 1 4 cr^2+\ln(r) A  + B\\
\end{eqnarray*}
Eine nat"urliche Bedingung ist nun, dass die Geschwindigkeiten endlich bleiben sollten. 
Der Logarithmus hat aber eine Singularit"at bei $r=0$. Daher m"ussen
wir $A=0$ w"ahlen. Das Geschwindigkeitsprofil lautet also
$$u_3(r) = \frac 1 4 cr^2+ B.$$
Nun m"ussen wir noch $B$ w"ahlen. Wir haben no-slip-Randbedingungen, d.h. die 
Geschwindigkeit an der Wand des Rohres muss Null sein. Der Radius des Rohres ist $R$, also $u_3(R)=0$:
$$ u_3(r) = \frac {-1} 4 c(r^2-R^2-r^2) = 
 \frac{p(z_1)-p(z_0)}{2 \nu\rho(z_1-z_0)}\,\, (-R^2+r^2)
 =
 \frac{p(z_0)-p(z_1)}{2 \nu\rho(z_1-z_0)}\,\, (R^2-r^2)
$$
Wenn die R"ohre die L"ange $L$ besitzt, der Druck an eine Ende $p_0$ und am andere 
Ende $p_1$ ist, so folgt  das Ergebnis
$$ u_3(r) = 
 \frac{p_0-p_1}{2 \nu\rho L}\,\, (R^2-r^2).
$$

Bemerkung: Unsere L"osung erf"ullt auch $\nabla u=0$, d.h. den Massenerhalt. 

Nun fragen wir, wieviel Fl"ussigkeit pro Zeiteinheit durch das Rohr transportiert 
wird. Wir integrieren den Fluss entlang der $z$-Achse
$$ u_3(r) \rho$$
Dazu bemerken wir, dass (siehe auch Beispiel~\ref{volIntegralBsp}), dass
das Integral einer radialsymmetrischen Funktion "uber einen Kreis 
in ein eindimensionales Integral verwandelt werden kann, 
$$ \int_{x^2+y^2<R} f(r) dxdy = \sum \int_0^R f(r) (2\pi r)\, dr.$$
Nutzen wir diese Formel, so finden wir

\begin{eqnarray*}
Q & = & \int_0^R \rho\,\,(2\pi r)u_3(r)\, dr\\
& = & \rho\,\,\int_0^R 2\pi r \frac{p_0-p_1}{4\nu\rho L}\,\, (R^2-r^2)\, dr\\
 & = & \rho\,\, \frac{\pi(p_0-p_1)}{2\nu\rho L} \int_0^R  r \, (R^2-r^2)\, dr\\
  & = &  \rho\,\,\frac{\pi(p_0-p_1)}{2\nu\rho L} \left(R^2 R^2/2-R^4/4\right)\\
    & = & \rho\,\, \frac{\pi(p_0-p_1)}{2\nu\rho L} \left(R^4/4\right)\\
        & = &  \frac{\rho\,\,\pi(p_0-p_1) R^4}{8\nu\rho L}
\end{eqnarray*}
\index{Poiseulliesche Gesetz}
%===============================================================================
\fpbox{Dies ist ein in der Str"omungsmechanik ber"uhmtes Gesetz, das 
Poiseulliesche Gesetz.
das besagt, dass der
Fluss durch ein Rohr proportional ist zur
\begin{enumerate}[(1)]
  \item vierten Potenz des Radius,
  \item Druckdifferenz an den beiden Enden,
  \item Inversen der L"ange.
\end{enumerate}}
\par\medskip
%===============================================================================
Insbesondere muss eine auch nur kleine Verringerung des Radius mit einer enormen 
Steigerung des Drucks bezahlt werden, will man den gleichen 
Durchfluss erhalten.
%===============================================================================

\begin{auf}\chd\label{block11A1}
\input{../../Aufgabensammlung/hm710.tex}
\end{auf}
