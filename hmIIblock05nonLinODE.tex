
%% ! T E X  root=hmIIneu.tex
% !TEX root=MA9603.WZW.tex
% !TEX program = pdflatex
% !TEX spellcheck = de_DE

%===============================================================================
\section{Nichtlineare Differentialgleichungen}
%===============================================================================
\zbox{
{\bf Ziele}:
\begin{itemize}
\item Differentialgleichungen mit getrennten Variablen l"osen k"onnen
\item nichtlineare, homogene Differentialgleichungen l"osen k"onnen
\item Eulerverfahren mit der Hand rechnen k"onnen
\end{itemize}}
%===============================================================================
%%%%% \subsection{Nicht-lineare gew"ohnlicher Differentialgleichungen}
%===============================================================================
\par\medskip
Die meisten Beispiele in der Physik -- Schwingungsgleichungen, Elektrotechnik -- sind lineare Gleichungen. Modelle aus Biologie und Chemie hingegen
 bestehen aus nicht-linearen Differentialgleichungen. 
Beispiele sind:

\begin{bspX}
Homogene, nicht-lineare Differentialgleichung erster Ordnung
$ x' = 5x(1-x)$.
\end{bspX}
%===============================================================================
\begin{bspX}
Inhomogene, nicht-lineare Differentialgleichung zweiter Ordnung
$ x'' - x'x + 5x = \sin(t)$.
\end{bspX}
%===============================================================================
Solche Differentialgleichungen sind in der Regel recht schwierig zu l"osen. In
der Praxis wird man einfache Differentialgleichungen handhaben k"onnen 
(z.B. die Typen, die wir unten behandeln), f"ur schwierigere Differentialgleichungen 
spezialisierte Literatur oder/und  Computer-Algebra-Programme zu Rate ziehen, und 
am aller h"aufigsten numerische Verfahren (siehe unten) anwenden.
%===============================================================================
\subsection{Einige Typen von  Differentialgleichungen und L"osungsverfahren}
%===============================================================================
\subsubsection{Differentialgleichungen mit getrennten Variablen}
%===============================================================================
\begin{sdefi} Eine Differentialgleichung der Form
$$ x' = f(t) g(x)$$
hei\""{}t Differentialgleichungen mit getrennten Variablen.
\end{sdefi}
Hier wendet man die Methode der Trennung der Variablen an, und rechnet
$$ \frac {d x}{dt} = f(t) g(x),\quad x(t_0) = x_0$$
aufl"osen nach $x$ ({\it formal} trennen wir auch $dx$ und $dt$)
$$ \frac {1}{g(x)}\, dx = f(t) dt$$
und nun integrieren wir,
$$ \int_{x_0}^x\frac {1}{g(y)}\, dy = \int_{t_0}^tf(\tau) d\tau.$$
Wenn $G(y)$ die Stammfunktion von $1/g(y)$ und $F(t)$ die Stammfunktion von $f(t)$ ist,
so finden wir
$$ G(x)-G(x_0) = F(t)-F(t_0).$$
Diese Gleichung m"ussen wir nach $x$ aufl"osen, und wir haben die L"osung.
%===============================================================================
\begin{bspX}
Logistische Differentialgleichung.
$$ x' = x(1-x),\qquad x(0) = x_0.$$
Dann rechnen wir
\begin{eqnarray*}
\frac {dx}{dt} & = & x (1-x)\\
\frac {1}{x(1-x)}dx & = & dt\\
\int_{x_0}^x  \frac {1}{y(1-y)}dy & = & \int_0^td\tau\\
\int_{x_0}^x  \left(\frac {1}{1-y} +\frac{1}{y}\right) dy & = & \int_0^td\tau\\
\log(\frac{y}{1-y})\bigg|_{x_0}^x   & = & t
\qquad\Rightarrow\qquad
\frac{x(1-x_0)}{x_0(1-x)} = e^t\\
x(1-x_0) &=& x_0(1-x)e^t = x_0 e^t- x_0\,\, x\,\, e^t\\
x\left(1-x_0+x_0\,\, e^t\right) &=& x_0 e^t
\\
x(t)
 &=& \frac{x_0 e^t}{1-x_0+x_0e^t} 
 = \frac{x_0 e^t}{1+x_0(e^t-1)} 
\end{eqnarray*}
\end{bspX}
%===============================================================================
\begin{auf}\chc\label{block5A1}
\input{../../Aufgabensammlung/hm697.tex}
\end{auf}
%===============================================================================
\subsubsection{Homogene Differentialgleichung}
%===============================================================================
\begin{sdefi} Eine Differentialgleichung der Form
$$ x' = f(x/t)$$
hei\ss{}t homogene Differentialgleichung.
\end{sdefi}

\fpbox{{\bf Achtung!} Wir haben nun zwei verschiedene Bedeutungen von ``homogene Differentialgleichung''.
Aufpassen!}\par\medskip

Wir transformieren, indem wir die Funktion
$$ u(t) = y(t)/t$$
einf"uhren. Dann finden wir $x(t) = t\,u(t)$ und
$$ x'(t) = u(t) + t u'(t)$$
Da $x'(t) = f(x(t)/t) = f(u(t)$, so haben wir
$$ u'(t) = (f(u(t))-u(t))\,\,\frac 1 t$$
eine Differentialgleichung mit getrennten Variablen. Die k"onnen wir mit den Methoden des
vorherigen Kapitels behandeln.
\begin{bspX}
$$x' = \sqrt{\frac x t},\qquad x(1)=2$$
Transformation
$$ $$

\begin{eqnarray*}
u=x/t,\quad u'&=&\sqrt{u}/t-u/t,\quad u(1)=x(1)/1=2\\
\Rightarrow\quad \int_{2}^u \frac 1 {\tilde u^{1/2}-\tilde u}\,d\tilde u 
& = & \int_1^t\,\tilde t^{-1} d\tilde t\\
\end{eqnarray*}
Wir finden
\begin{eqnarray*}
\int_{2}^u \frac 1 {\tilde u^{1/2}-\tilde u}\,d\tilde u 
& = & \int_{2}^u \frac 1 {\sqrt{\tilde u}(1-\sqrt{\tilde u})}\,d\tilde u 
 =  \int_{2}^u \left(
  \frac 1 {\sqrt{\tilde u}}
+ \frac 1 {1-\sqrt{\tilde u}}\right)
\,d\tilde u \\
& = & 2\sqrt{\tilde u}\bigg|_2^u -2\sqrt{\tilde u}
                -2\ln(\sqrt{\tilde u}-1)\bigg|_2^u
= -2\ln(\sqrt{\tilde u}-1)\bigg|_2^u
= -2\ln(\sqrt{u}-1)+2\ln(\sqrt{2}-1)
\end{eqnarray*}
Damit, und mit $\int_1^t \tilde t^{-1}\,d\tilde t = \ln(t)$, erhalten wir
\begin{eqnarray*}
-2\ln(\sqrt{u}-1)+2\ln(\sqrt{2}-1)& = & \ln(t)\\
\ln(\sqrt{u}-1)-\ln(\sqrt{2}-1)& = & -\ln(t)/2 = \ln(1/\sqrt{t})\\
\ln(\sqrt{u}-1) & = & 
\ln(1/\sqrt{t})+ \ln(\sqrt{2}-1) )= \ln((\sqrt{2}-1)/\sqrt{t})\\
\sqrt{u}- 1 & = & \frac{(\sqrt{2}-1)}{\sqrt{t}}\qquad \Rightarrow\qquad
u(t)  = \left(1+\frac{(\sqrt{2}-1)}{\sqrt{t}}\right)^2
\end{eqnarray*}
Mit $x=u\, t$ folgt
$$ x(t) = t\,\, \left(1+\frac{(\sqrt{2}-1)}{\sqrt{t}}\right)^2.$$
\end{bspX}
%===============================================================================
\subsection{Numerische Verfahren}
Wir sehen uns hier an, wie wir eine Differentialgleichung des Typs 
$$  x'=f(x,t), \qquad x(0) = x_0$$ 
numerisch l"osen k"onnen (Anfangswertproblem). D.h., wir w"ahlen Zeitpunkte 
$t_i$, und suchen Werte $x_i$, die $x(t_i)$  approximieren. Gesucht wird 
nun ein Algorithmus bzw.\ eine Funktion $F(.)$, sodass
$$ x_{i+1}=F(x_i)\quad\mbox{ und }\quad  x_i\approx x(t_i).$$
Wir n"ahern uns der numerischen Approximation der L"osung 
"uber zwei (letztlich gleichwertige) Ans"atze: eher intuitiv, "uber die Interpretation einer Differentialgleichung, und eher formal, 
mittels Taylor-Entwicklung.
%===============================================================================
\subsubsection{Explizites Euler-Verfahren}

%===============================================================================
\fpbox{
{\bf Weg "uber die Definition der Ableitung.} Letztlich ist die Interpretation 
der Differentialgleichung, dass wir einen Zustand $x(t)$ haben, und 
f"ur diesen die Zustands\"anderungsrate $x'(t)$ als Funktion des Zustandes (und eventuell der Zeit) 
auffassen, d.h. $x'(t) = f(x(t),t)$.

Die Ableitung ist der Limes des Differenzenquotienten. Ersetzen wir wieder
die Ableitung durch einen Differenzenquotienten in einem kleinen, aber
positiven Zeitintervall $\Delta t$ zur"uckgehen, erhalten wir
$$ x' = f(x,t) 
\qquad \rightarrow \qquad
\frac{x(t+\Delta t)-x(t)}{\Delta t} \approx f(x(t),t) 
\qquad \rightarrow \qquad
x(t+\Delta t) \approx x(t) + \Delta t \,\, f(x(t), t).$$
Da wir uns also nur f"ur die Zeitpunkte $t_i = i\Delta t$ interessieren, so bekommen wir eine Rekursionsformel f"ur $y_i := x(t_i)$:
$$ y_0 = x_0 \mbox{ gegeben, }\quad
y_{i+1} = y_i + \Delta t f(y_i, t_i).$$
Diese Rekursionsformel kann leicht implementiert werden, und so die L"osung berechnet werden.
Fragen der Numerik sind dann: wie gro\ss{} ist der Fehler (Abweichung dieser approximativen L"osung von der ``echten'' L"osung
an den Stellen $t_i$? Optimierung des Verfahrens, i.e. funde Verfahren die einen m"oglichst kleinen Fehler haben, und
trotzdem nicht rechen-aufwendig sind.
Dies ist das {\bf Euler-Verfahren}. \index{Euler-Verfahren}
}
%===============================================================================

%===============================================================================
\fpbox{{\bf Weg "uber Taylorentwicklung.} Nun sehen wir uns das Problem aus einem 
leicht anderen Blickwinkel an: Wir entwickeln die L"osung von $x'=f(x,t)$ nach der Zeit in eine Taylor-Reihe,
$$ x(t+\Delta t) = x(t) + \Delta t x'(t) + \Delta^2 x''(t)+\cdots.$$
Wir wissen wohl, dass 
$$ x'(t) = f(x(t),t)$$
d.h.
$$ x(t+\Delta t) = x(t) + \Delta t f(x(t), t) +  \Delta t^2 x''(t)+\cdots.$$
Vernachl"assigen wir Terme zweiter Ordnung ($\Delta t^2$), so erhalten wir exakt das gleiche Verfahren wie vorhin. 
}
%===============================================================================
\begin{bspX}
Betrachte die Differentialgleichung 
$$ x' = a\, x,\qquad x(0)=x_0$$
i.e.\ $x' = f(x,t)$ mit $f(x,t)= a x.$
Die analytische L"osung kennen wir, $x(t)=e^{at}\, x_0$. 
Die N"aherungsL"osung des expliziten Euler--Verfahrens zu den Zeitpunkten $n\,\Delta t$ nennen wir
 $x_n$,
$$ x_{n+1} = x_n + \Delta t \,\,(a\, x_n) = (1+a\Delta t) x_n.$$
Nun halten wir einen Zeitpunkt $t$ fest, erh"ohen $n$ und verkleinern dabei
gleichzeitig $\Delta t$, sodass immer  $t=n\Delta t$  wahr ist. Dann folgt
$$ x_n =(1+a\Delta t)^n x_0 
= (1+at/n)^{n}\, x(0).$$
Erinnerung: Eine M"oglichkeit, $e^x$ zu definieren, lautete
$$ e^x = \lim_{n\rightarrow\infty} (1+x/n)^n.$$
Wenn wir diese beiden Formeln vergleichen, so finden wir
$$ \lim_{n\rightarrow n} x_n = x_0e^{at}.$$
D.h., wenn wir $t$ festhalten, $\Delta t$ immer kleiner machen, 
so approximiert
die N"aherungsL"osung tats"achlich die analytische L"osung.
\end{bspX}
%===============================================================================
\subsubsection{Stabilit"at und Positivit"atserhalung 
des numerischen Verfahrens}\par
\index{Stabilit"at numerischer Verfahren}
Es hat sich herausgestellt, dass man die Qualit"at eines numerischen Verfahrens
u.a.\ daran beurteilen kann, ob es gewisse Testgleichungen richtig 
l"osen kann, wenn die Zeit $t$ gegen unendlich geht; das ist nat"urlich schwierig, weil man da unendlich viele Approximationsschritte machen muss, und
sich der Fehler Aufschaukeln kann. Die einfachste Testgleichung ist
$$ x' = -\lambda x,\qquad x(0)=x_0,\qquad\mbox{ mit } \lambda>0.$$
Dann wissen wir, dass $x(t)\rightarrow 0$ f"ur $t\rightarrow\infty$. 
%===============================================================================
\begin{sdefi}
Sei $\lambda\in\R$, $\lambda\geq 0$. 
Das Stabilit"atsgebiet zu $\lambda$ 
eines numerischen Verfahrens ist die Menge aller 
$\Delta t\in\C$,  sodass die Approximation von 
$$ x' = -\lambda x,\qquad x(0)=x_0,\qquad\mbox{ mit } \lambda>0.$$
gegen Null geht, wenn die Zahl der Schritte gegen unendlich geht.
\end{sdefi}
%===============================================================================
\begin{bspX}
Wie gro\ss{} ist der Stabilit"atsgebiet des expliziten Euler-Verfahrens?\\
Da  $x_n =(1+a\Delta t)^n x_0$ und $a=-\lambda$ muss
$$ |1-\lambda\Delta t |<1$$
gelten, d.h.  $\lambda\Delta t$ liegen im Kreis mit Radius eins und Mittelpunkt 1.
D.h., wenn $\lambda$ gro\ss{} ist, muss man die Zeitschrittweite 
sehr klein machen.
\end{bspX}
%===============================================================================
Eine weitere, wichtige Eigenschaft (insbesondere f"ur die Simulation von Modellen 
in der Biochemie) ist die Positivit"at.
%===============================================================================
\begin{sdefi}
Der positivit"atserhaltende Bereich (Positivi"atsbereich) 
eines numerischen Verfahrens ist die Menge aller 
$\Delta t\in\R$,  sodass die Approximation von 
$$ x' = -\lambda x,\qquad x(0)=x_0,\qquad\mbox{ mit } \lambda>0.$$
positiv bleibt, wenn $x_0$ positiv ist.
\end{sdefi}
%===============================================================================
\begin{bspX}
Wie gro\ss{} ist der positivit"atserhaltende Bereich des expliziten Euler--Verfahrens?\\
Da  $x_n =(1+a\Delta t)^n x_0$ und $a=-\lambda$ muss
$$ 1-\lambda\Delta t >0$$
gelten, d.h.  
$$\Delta t < 1/\lambda.$$
D.h., wenn $\lambda$ gro\ss{} ist, muss man die Zeitschrittweite 
sehr klein machen.
\end{bspX}
Kombinieren wir den positiven Bereich mit den stabilen Bereich, so finden wir 
$$ 0<\Delta t < 1/\lambda.$$
D.h., wenn $\lambda$ gro\ss{} ist, m"ussen wir $\Delta t$ sehr klein machen. 
Das ist schlecht, da wir viele Schritte ben"otigen, bis wir zu einem gegebenen 
Zeitpunkt $t$ kommen.\par\medskip
%===============================================================================
\fpbox{
Bemerkung: Wenn man mit numerischen Paketen wie MATLAB arbeitet, so sollte man 
unbedingt die Schrittweitenkontrolle entsprechend einstellen, falls man auf 
Positivit"atserhaltung wert legt!  L"asst man MATLAB die Schrittweite automatisch 
w"ahlen, so wird i.d.R. die Positivit"at gebrochen!}
\par\medskip
Wir sehen uns zwei M"oglichkeiten an, die es erlauben, 
gr"o\ss{}ere Zeitschrittweiten 
zu w"ahlen: implizite Euler--Verfahren und Verfahren h"oherer Ordnung. 
%===============================================================================
\subsubsection{Implizites Euler--Verfahren }
\fpbox{Das explizite Euler--Verfahren haben wir letztlich aus 
$$ x' = f(x,t) 
\qquad \rightarrow \qquad
\frac{x(t+\Delta t)-x(t)}{\Delta t} \approx f(x(t),t) 
\qquad \rightarrow \qquad
x(t+\Delta t) = x(t) + \Delta t \,\, f(x(t), t)$$
gewonnen. Wir k"onnen aber genauso gut sagen, dass 
$$ \frac{x(t+\Delta t)-x(t)}{\Delta t} \approx f(x(t+\Delta t),t+\Delta t)$$
 und somit 
$$ x(t+\Delta t) \approx  x(t)+ \Delta t\,\, f(x(t+\Delta t),t+\Delta t).$$
Setzen wir wie vorher $x_n=x(n\Delta t)$,und $t_n =  n\Delta t$ so haben wir
$$ x_{n+1} = x_n + \Delta t\,\, f(x_{n+1},t_{n+1}).$$
Dies ist eine implizite Gleichung f"ur $x_{b+1}$ (d.h. die Gleichung muss erst ach $x_{n+1}$ aufgel"ost werden.
Das Verfahren hei\ss{}t daher {\bf Implizites Euler--Verfahren}.}
%===============================================================================
\begin{bspX} Wir nehmen uns wieder unsere Testgleichung her
$$ x' = -\lambda x,\quad x(0) = x_0$$
und leiten eine Formel f"ur das implizite Euler--Verfahren her:
$$ x_{n+1} = x_n-\Delta t \lambda x_{n+1}
\quad \Rightarrow\quad
x_{n+1}+ \Delta t \lambda x_{n+1} = x_n
\quad \Rightarrow\quad
x_{n+1} = \frac 1 {1+\lambda\, \Delta t} x_n.$$
\end{bspX}
%===============================================================================

\begin{bspX} Stabilit"atsbereich und Positivit"atsbereich f"ur das implizite Eulerverfahren.\par
\begin{itemize}
\item Der Stabilit"atsbereich ist gegeben durch 
alle komplexen $\Delta t$ mit (wir setzen $a=-\lambda$)
$$ 
\left |\frac 1 {1+\lambda \Delta t}\right| <1
\quad \Leftrightarrow\quad
\left |1+\lambda \Delta t\right| >1.
$$
D.h. der Stabilit"atsbereich besteht aus der komplexen Ebene {\rm ohne} den Kreis 
mit Radius $1/\lambda$ um -1.
\item Der Positivit"atsbereich ist gegeben durch alle reellen $\Delta t$ mit
$$ \frac 1 {1+\lambda \Delta t}>0
\quad \Leftrightarrow\quad
\Delta t > -1/\lambda.
$$
\end{itemize}
\end{bspX}
Kombinieren wir wieder den Stabilit"ats- und Positivit"atsbereich, so sehen, wir 
dass wir im Prinzip alle nicht-negativen Schrittweiten $\Delta t$ zulassen d"urfen. 
D.h., jede positive Schrittweite $\Delta t$ wird bei positiver Anfangsbedingung 
eine positive L"osung generieren, die f"ur $n\rightarrow\infty$ gegen Null geht. 
Damit erf"ullt sie zwei wichtige Forderungen. Damit die Approximation aber genau wird, d"urfen wir dennoch die Schrittweite nicht zu gro\ss{} w"ahlen. Typischerweise w"ahlt man $\Delta t \approx 0.01$. 
%===============================================================================
\subsubsection{Verfahren h"oherer Ordnung }
Ausgehend von der Taylorformel, kann man auch Verfahren herleiten, die einen 
kleineren Fehler haben. Wir k"onnen z. B. in der Taylorentwicklung
$$ x(t+\Delta t) = x(t) + \Delta t x'(t) + \Delta^2 x''(t)+ \Delta^3\cdots.$$
nicht nur $x'$ durch $f(x,t)$ ersetzen, sondern auch $x''$ einsetzen, geben durch
$$ x''(t) = \frac d {dt} f(x,t) = f_x x'+f_t = f_x(x,t) f(x,t)+f_t(x,t)$$
d.h.
$$ x(t+\Delta t) = x(t) + \Delta t f(x(t),t) + \Delta^2 [f_x(x,t) f(x,t)+f_t(x,t)]+ \Delta^3\cdots.$$
W"ahrend der (lokale) Fehler-Term in dem Euler-Verfahren mit einem Quadrat in der 
Zeit-Schrittweite $\Delta t$
startet, startet der Fehlerterm in dem Verfahren
$$ x_{i+1} = x_i + \Delta t f(x_i,t_i) + \Delta t^2 [f_x(x_i,t_i) f(x_i,t_i)+f_t(x_i,t_i)]$$
mit $\Delta t^3$. Da $\Delta t$ klein ist, sollte der Fehler wesentlich kleiner 
sein. Man kann die Ordnung des Fehlers
definieren, und Verfahren mit gegebener Ordnung suchen und z.B.\ dabei einen
m"oglichst gro\ss{}en Stabilit"atsbereich garantieren. Die Theorie der Numerik 
gew"ohnlicher Differentialgleichungen hat auf diese Weise sehr raffinierte 
Verfahren entwickelt.
\par\medskip
\fpbox{\underline{Worauf kommt es an?}\par
(1) Wir m"ussen die Schrittweite aus dem Stabilit"atsbereich w"ahlen.\\
(2) Wir m"ussen die Schrittweite klein genug w"ahlen, um den Fehler klein zu halten.\\
(3) Falls wir Positivit"atserhaltung ben"otigen, sollten wir  ein implizites
Verfahren w"ahlen.}
%===============================================================================
\subsection*{Aufgaben}
\begin{auf}\chc\label{block5A2}
\input{../../Aufgabensammlung/hm695.tex}
\end{auf}
\begin{auf}\chc\label{block5A3}
\input{../../Aufgabensammlung/hm744.tex}
\end{auf}
\begin{auf}\chc\label{block5A4}
\input{../../Aufgabensammlung/hm693.tex}
\end{auf}
\begin{auf}\chb\label{block5A5}
\input{../../Aufgabensammlung/hm696.tex}
\end{auf}
\begin{auf}\cha\label{block5A6}
\input{../../Aufgabensammlung/hm261.tex}
\end{auf}
%===============================================================================





