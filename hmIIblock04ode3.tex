 

%% ! T E X  root=hmIIneu.tex
% !TEX root=MA9603.WZW.tex
% !TEX program = pdflatex
% !TEX spellcheck = de_DE

%==============================================================================
\section{Lineare Differentialgleichungen 3}
%==============================================================================
\zbox{
{\bf Ziele}:
\begin{itemize}
\item  Partikul"are (spezielle) L"osungen f"ur folgende St"orfunktionen finden k"onnen: 
            Polynom, exponentiell, trigonometrisch
\item L"osung eines (auch inhomogenen) Anfangswertes oder Randwertes bestimmen k"onnen.
\end{itemize}}
%============================================================================== 
\subsection{Inhomogene Gleichung:  Partikul"are L"osung}
%==============================================================================
Eine M"oglichkeit eine partikul"are (spezielle) L"osung in die Hand zu bekommen, besteht darin,
die Variation-der-Konstanten Formel auf Systeme mit konstanten Koeffizienten zu erweitern.
Wenn wir die Differentialgleichung h"oherer Ordnung als System schreiben, so erhalten wir
$$ y' = A y+ b$$
wobei $y$ und $b$ nun Vektoren, $A$ eine Matrix ist. Nun k"onnen wir die Methoden von Systemen
(s.u.) anwenden, um  eine partikul"are L"osung zu erhalten. Dies ist mathematisch am befriedigendsten -- allerdings nicht sehr praktikabel.\\
F"ur spezielle (die wichtigsten) Inhomogenit"aten finden wir ``Rezepte'', d.h.\ spezielle Ansatzfunktionen,
die zu einer partikul"aren L"osung f"uhren. Die wichtigsten F"alle klappern wir jetzt ab. \index{St"orfunktion}

\subsection{Polynom als St"orfunktion}
Betrachte f"ur eine lineare Differentialgleichung $n$'ter Ordnung ein Polynom
als St"orfunktion, d. h. \index{St"orfunktion!Polynom} 
$$ \sum_{i=0}^n a_i x^{(i)}(t) = p(t)$$
mit dem Polynom $ p(t) = \sum_{i=0}^k c_i t^i.$ vom Grad $k$. Wenn 
$a_0=\cdots a_{l-1}=0$, $a_{l}\not = 0$, so setze an
$$ x(t) = \sum_{i=0}^{k+l} b_i t^i.$$
Einsetzen dieser Ansatzfunktion und Koeffizientenvergleich f"uhrt zum Ziel.\par\medskip
%==============================================================================
\begin{bspX}
Finde spezielle L"osung f"ur 
$$ x'' + 3 x'   = 9 t^2 + 8 t + 8.$$
Dann ist $a_0=0$ und das Polynom auf der rechten Seite hat Grad zwei. Damit 
setzen wir f"ur die L"osung ein allgemeines Polynom vom Grad drei an, und finden
\begin{eqnarray*}
x(t) & = &  b_0 + b_1 t + b_2 t^2 + b_3 t^3\\
x'(t) & = &   b_1  + 2\,b_2 t + 3\,b_3 t^2\\
x''(t) & = &  2\,b_2  + 6 b_3 t
\end{eqnarray*}
Einsetzen ergibt
\begin{eqnarray*}
3\,(b_1  + 2\,b_2 t + 3\,b_3 t^2) + (2\,b_2  + 6 b_3 t)  & = & 9 t^2 + 8 t + 8\\
9 b_3  t^2 + ( 6 b_2+6 b_3)t + (3 b_1 +  2 b_2) & = & 9 t^2 + 8 t + 8
\end{eqnarray*}
Wir erhalten also ein lineares Gleichungssystem f"ur die Koeffizienten 
$( b_1, b_2, b_3)$; $b_0$ kommt nicht vor:
\begin{eqnarray*}
\left(\begin{array}{ccc|c}
3 & 2 & 0 & 8\\
0 & 6 & 6 &8\\
0 & 0 & 9 &9\\
\end{array}
\right)
\end{eqnarray*}
Dieses Gleichungssystem wird von $b_3=1$, 
$b_2=(8-6 b_3)/6 = 2/6=1/3$ und $b_1=(8-2b_2)/3 = (8-2/3)/3 = 22/9$ erf"ullt, 
d.h. jede Funktion
$$ x(t) = b_0 +\frac{22}{9}t+\frac{1}{3}t^2+t^3$$
f"ur irgendein $b_0\in\R$ ist eine spezielle L"osung der inhomogenen Gleichung 
(es gibt nat"urlich noch viel, viel mehr!).
\end{bspX}

%==============================================================================
\begin{auf}\chb\label{block4A1}
\input{../../Aufgabensammlung/hm104.tex}
\end{auf}
%==============================================================================
\subsection{Exponentialfunktion als St"orfunktion}
Betrachte \index{St"orfunktion!Exponentialfunktion} 
$$ \sum_{i=0}^n a_i x^{(i)}(t) = f(t)$$
und
$$ f(t) = a\,e^{bt}.$$
Wir m"ussen zwei F"alle unterschieden: $b$ ist / ist nicht Nullstelle des 
charakteristischen Polynoms $p(\lambda)$.\par\medskip
\fpbox{{\it Fall 1:} $p(b)\not=0$.\par
Ansatz:
$$x(t) = c\, e^{bt}.$$
Einsetzen ergibt
$$ p(b) \,c\,e^{bt} = a\,e^{bt}$$
d.h. $c= a/p(b)$ und 
$$ x(t) = \frac a {p(b)} e^{bt}$$
ist spezielle L"osung.}\par\medskip
\fpbox{{\it Fall 2:}  $p(b)=0$, $b$ ist $m$-fache Nullstelle.\\ Also ist 
$p(\lambda) = q	(\lambda) (\lambda-b)^m$.
Dann m"ussen wir ansetzen
$$ x(t) = c\, t^{m}e^{bt}.$$
Einsetzen und Koeffizientenvergleich f"uhrt zum Ziel. Davon k"onnen wir uns 
"uberzeugen: Die Rechnung
in Lehr\-ein\-heit 3, Kapitel 3.2 (A), Fall~2 dieses Abschnittes zeigt
$$p(d/dt) c t^m e^{bt} 
= q(d/dt) (d/dt-b)^mc t^m e^{bt} 
= q(d/dt) m!\, c\, e^{bt} 
= m!\, c\, q(b)\,e^{bt} \not = 0$$
Also
$$ p(d/dt)c t^m e^{bt}  = a\,e^{bt}$$
f"uhrt auf
$$ c  = \frac{a}{m! q(b)}.$$
(Bemerkung: Diese Gleichung ist nicht f"ur den praktischen Gebrauch - siehe 
Beispiel - sondern nur dazu da, sich klar zu machen, dass der Ansatz zum Ziel 
f"uhrt).}

\begin{bspX}
Finde eine spezielle L"osung von $ x''-4x' = 3\,e^{4t}$. Zur L"osung:
Das charakteristische Polynom lautet $p(\lambda) = \lambda^2-4\lambda = \lambda(\lambda-4)$
d.h. der Exponent der Inhomogenit"at (der ist vier) ist Nullstelle des 
charakteristischen Polynoms. (man kann es auch so ausdr"ucken: $e^{4t}$ ist 
L"osung der homogenen Gleichung $x''-4x'=0$).\par
Die Nullstelle ist einfach. Wir setzen daher an
$ x(t) = c\, t \, e^{4t}$ und erhalten
$$ 
x' = c e^{4t}+4cte^{4t},\qquad
x'' = 4c e^{4t}+4ce^{4t}+16ce^{4t}.$$
Einsetzen ergibt
$$ x''-4x'
 = (4c e^{4t}+4ce^{4t}+16ce^{4t}) - 4(c e^{4t}+4cte^{4t})
= 4 c e^{4t}.
$$
Diese Funktion soll gleich $3 e^{4t}$ sein, d.h.
$ c = 3/4$ und $ x(t) = \frac 3 4 \,\, t\,\,e^{4t}.$
\end{bspX}
%==============================================================================
\begin{auf}\chb\label{block4A2}
\input{../../Aufgabensammlung/hm127.tex}
\end{auf}
%==============================================================================
\subsection{Trigonometrische Funktionen als St"orfunktion}
Betrachte \index{St"orfunktion!Trigonometrische Funktion} 
$$ \sum_{i=0}^n a_i x^{(i)}(t) = f(t)$$
und
$$ f(t) = a\,\sin(bt),\qquad\mbox{ oder }\quad f(t) = a\,\cos(bt).$$
Dann machen wir folgenden Trick: Wir ersetzen die rechte
Seite durch 
$$\tilde f(t) = a e^{{\rm i} b t}.$$
Dann sind wir im letzten Fall (St"orfunktion ist Exponentialfunktion). Den k"onnen 
wir behandeln und erhalten eine komplexe, spezielle L"osung $x(t)$. Realteil 
dieser L"osung ist die spezielle L"osung f"ur $f(t) = a \cos(bt)$, Imagin"arteil die
spezielle L"osung f"ur $f(t) = b\sin(t)$.\\
Man beachte, dass auch $a\,e^{rt}\sin(bt)$ als Imagin"arteil von 
$a\,e^{(r+\iii b)t}$ geschrieben werden kann (analog f"ur 
$a\,e^{rt}\cos(bt)$ mit dem Realteil dieser Funktion), und so auch diese St"orfunktion auf die gleiche Art abgehandelt werden kann.\par

\begin{bspX} $x''+x=\sin(\omega\,t)$\par
{\it Genereller Ansatz:} Wir rechnen komplex, $y''+y=e^{\iii\omega\,t}$.\par
Charakteristisches Polynom: $p(\lambda)=\lambda^2+1$.\\
{\it Fall 1:} $\omega\not=\pm 1$. Dann ist $p(\iii\omega)\not = 0$. 
Ansatz: $y=c\, e^{\iii\omega\,t}$. 
$$ -\omega^2 c\, e^{\iii\omega\,t}+c\,e^{\iii\omega\,t} = e^{\iii\omega\,t}.$$
also $c(-\omega^2+1)=1$, d.h.\ $c=1/(1-\omega^2)$. \\
Spezielle L"osung: 
$x(t) 
= \Im\left(e^{\iii\omega\,t}/(1-\omega^2)\right)
= \sin(\omega t)/(1-\omega^2).$\\
{\it Fall 2:} $\omega = \pm1$. Dann ist $p(\iii\omega) = 0$ 
(einfache Nullstelle). 
Ansatz: $y=c\, \,t\,e^{\iii\omega\,t}$. 
$$ \left(-\omega^2 c\, t\, e^{\iii\omega\,t}
 + 2\iii\omega e^{\iii\omega\,t}\right)_{\omega=\pm 1}
+\left(c\,t\,e^{\iii\omega\,t}\right)_{\omega=\pm 1}
= \pm 2\iii e^{\pm\iii\,t}
= e^{\pm\iii\,t}$$
also $c = 1/(\pm 2 \iii)=\mp\iii/2$. 
Spezielle L"osung:  
$x(t) 
= \Im\left(\mp\iii\,t\,e^{\iii\omega\,t}/2\right)
= t\cos(\omega t)/2.$\\
{\it Interpretation:} Die Gleichung $x''+x=0$ beschreibt eine 
unged"ampfte Schwingung (z.B.\ eines Federpendels) 
mit Eigenfrequenz $1$. Die Inhomogenit"at speigelt eine Anregung des
Systems mit Frequenz (besser: Kreisfrequenz) $\omega$. 
Falls $\omega\pm 1$, so wird das System nicht mit der Eigenfrequenz angeregt, es stellt sich eine stabile Schwingung ein. Falls $\omega=\pm 1$, so trifft die Anregung die Eigenschwingung. 
Das System akkumuliert Energie, und die Amplitude der Schwingung steigt linear in der Zeit.
\end{bspX}

%==============================================================================
\begin{auf}\cha\label{block4A3}
\input{../../Aufgabensammlung/hm701.tex}
\end{auf}
%==============================================================================
\subsection{Produkt von Polynom und Exponential- (Trigonometrischer)  Funktionen als St"orfunktion}
{\it Betrachte\footnote{Dieser Abschnitt wird in der Vorlesung nicht behandelt}
\index{St"orfunktion!Produkt von Polynom und Trigonometrische Funktion} 
\index{St"orfunktion!Produkt von Polynom und Exponentialfunktion Funktion}
$$ \sum_{i=0}^n a_i x^{(i)}(t) = f(t),$$
und $q(t)$ ein Polynom des Grades $j$ und
$$f(t) = q(t) e^{bt}.$$
Der ``richtige'' Ansatz ist dann ein Produkt aus einem Polynom mit $e^{bt}$, d.h.
$$ x(t) = \sum_{i=0}^j b_i t^i\,\,\, e^{bt}.$$
Etwas genauer: 
\begin{itemize}
\item[Fall 1:] Sei $b$ nicht Nullstelle des Charakteristischen Polynoms.\\
Dann ist der richtige Ansatz
$$ x(t) = \sum_{i=0}^{j} b_i t^i\,\,\, e^{bt}.$$
\item[Fall 2:] Sei $b$ $m$-fache Nullstelle des charakteristischen Polynoms.\\
Dann ist der richtige Ansatz
$$ x(t) = t^m\,\,\sum_{i=0}^{j} b_i t^i\,\,\, e^{bt}.$$ 
\end{itemize}}
%==============================================================================
\subsection{St"orfunktion: Summen von bisherigen St"orfunktionen}
 Betrachte \index{St"orfunktion!Summen von Funktion} 
$$ \sum_{i=0}^n a_i x^{(i)}(t) = f(t)$$
und
$$ f(t) = f_1(t) + f_2(t).$$
Dann finde partikul"are L"osungen $y_1$ f"ur
$$ \sum_{i=0}^n a_i y_1^{(i)}(t) = f_1(t)$$
und $y_2$ f"ur 
$$ \sum_{i=0}^n a_i y_2^{(i)}(t) = f_2(t).$$
Die Summe dieser beiden partikul"aren L"osungen $x=y_1+y_2$ tut's:
\begin{eqnarray*}
\sum_{i=0}^n a_i x^{(i)}(t)
 & = & 
\sum_{i=0}^n a_i (y_1(t)+y_2(t))^{(i)}\\
 & = & 
\sum_{i=0}^n a_i (y_1^{(i)}(t)+y_2^{(i)}(t))\\
 & = &  
 \sum_{i=0}^n a_i y_1^{(i)}(t) \quad + \quad
 \sum_{i=0}^n a_i y_2^{(i)}(t)\\
  & = & 
  f_1(t)+f_2(t)
\end{eqnarray*}
Der Trick funktioniert nat"urlich auch, wenn wir nicht nur $f_1$ und $f_2$, 
sondern $k$ Summanden $f_1+f_2+\cdots+f_k$ auf der rechten Seite haben.
%==============================================================================
\subsection{Volles Problem}
%==============================================================================
Im Prinzip wissen wir jetzt, wie wir ein inhomogenes Anfangswertproblem angehen m"ussen:  
\begin{enumerate}
\item Finde eine spezielle L"osung, und eliminiere damit die Inhomogenit"at
\item F"ur geeignete Anfangswert (in der Regel nicht die Original-Anfangswerte!) 
bestimme die L"osung der homogenen Gleichung
\item Addiere beide L"osungen. Fertig!
\end{enumerate}

Das f"uhren wir jetzt an einem Beispiel durch.\par\medskip
%==============================================================================
\begin{bspX}
L"osungswegs eines inhomogenen, linearen Anfangswertproblems: suche
die L"osung von 
$$ 3 x'' + 12 x' + 12 x + 12 = -24\, t,\qquad
\mbox{ mit }\quad x(0) = 2,\quad x'(0) = -1.$$
\noindent{\it Spezielle L"osung.} Die Gleichung kann man schreiben als
$$ 3 x'' + 12 x' + 12 x =f(t) $$
mit $ f(t) = -12-24 t$.\\
Die St"orfunktion ist also ein Polynom  ersten Gerades. 
Da $a_0=12\not=0$ setzen wir die spezielle L"osung
$y(t)$ ebenfalls als Polynom ersten Grades an, und finden
$$ y(t) = a + bt,\qquad y'(t) = b,\qquad y''(t) = 0.$$
Einsetzen in die Gleichung ergibt
$$ -12-24 t = 3y''+12y'+12y = 12\, b + 12\, (a+bt) = 12 (a+b) + 12 b t.$$
Damit folgt $b=-2$ und $a = 1$. Die (besser: eine) spezielle L"osung lautet also $y(t) = 1-2t$.\\
%==============================================================================
\noindent{\it Anfangswerte.} Wir schreiben die L"osung $x(t)$ als Summe von der 
schon berechneten speziellen L"osung $y(t)$ und einer unbekannten Funktion $z(t)$. 
Die Funktion $z(t)$ folgt dann
$ 3 z'' + 12 z' + 12 z = 0$
mit Anfangsbedingungen
$ z(0) = x(0)-y(0) = 2-1=1,\qquad z'(0) = x'(0)-y'(0) = -1-(-2)=1.$\\
Das charakteristische Polynom lautet
$ p(\lambda) = 3\lambda^2+12\lambda+12 = 3(\lambda^2+4\lambda+4) = 3(\lambda+2)^2$
d.h.\ es besitzt nur eine, doppelte, Nullstelle $\lambda=-2$. \\
Das (ein) Fundamentalsystem lautet daher
$ e^{-2t}$ und $t\,e^{-2t}.$
Wir k"onnen $z(t)$ also darstellen als
$ z(t) = c_1 \,\, e^{-2t} \,\,+ c_2\,\, t\,e^{-2t}.$
Die Anfangswerte bestimmen uns $c_1$, $c_2$. Die Bedingung $z(0) = 1$ ergibt
$ c_1 = 1$ und mit $z'(t) = c_1(-2 e^{-2t}) \,\,+ c_2\,\,(e^{-2t}-2t \,e^{-2t})$ folgt aus der Bedingung $z'(0) = 1$
dass
$ -2c_1+c_2 =1$ und daraus $c_2 = 1+2 c_1 = 3.$
Damit finden wir
$ z(t) =  e^{-2t} \,\,+ 3\,\, t\,e^{-2t}.$\\
\noindent{\it L"osung des Problems} Die L"osung des Problems ist gleich der Summe 
von $y$ und $z$, $ x(t) =  e^{-2t} \,\,+ 3\,\, t\,e^{-2t} + 1-2t.$
\end{bspX}
%==============================================================================
{\it Bemerkung:} Als die allgemeine L"osung eines inhomogenen Anfangswertproblems 
\index{allgemeine L"osung}
bezeichnet man die Summe der speziellen L"osung mit allen Funktionen des Fundamentalsystems (die
mit Konstanten multipliziert werden). Man kann durch Wahl der Konstanten dann alle m"oglichen
L"osungen erhalten. In unserem Beispiel w"urde die allgemeine L"osung lauten
$$ x(t) = c_1 e^{-2t} \,\,+ c_2\,\, t\,e^{-2t} + 1-2t.$$
%==============================================================================
\subsection*{Aufgaben}

\begin{auf}\chb\label{block4A5}
\input{../../Aufgabensammlung/hm185.tex}
\end{auf}
\begin{auf}\cha\label{block4A6}
\input{../../Aufgabensammlung/hm071.tex}
\end{auf}
\begin{auf}\chb\label{block4A7}
\input{../../Aufgabensammlung/hm190.tex}
\end{auf}
%==============================================================================






















