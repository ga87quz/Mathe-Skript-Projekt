 


%% ! T E X  root=hmIIneu.tex
% !TEX root=MA9603.WZW.tex
% !TEX program = pdflatex
% !TEX spellcheck = de_DE



%===============================================================================
\section{Zeitskalenanalyse, Chemische Reaktionen  und Enzymkinetik}
%===============================================================================
\zbox{{\bf Ziele}:
\begin{itemize}
\item Modelle chemischer Reaktionen aufstellen k"onnen
\item Die Prinzipien der Zeitskalenanalyse erkl"aren k"onnen
\item Michaelis-Menten Kinetik aufstellen k"onnen 
\end{itemize}}
%===============================================================================
\subsection{Chemische Reaktionen - Massenwirkungsgesetz}

Wir betrachten drei Stoffe, A, B, C. Die Konzentrationen dieser Stoffe sei gegeben durch $[A]$,
$[B]$ und $[C]$. Die Stoffe $A$ und $B$ sollen zu $C$ reagieren. D.h., $[A]$ und $[B]$ wird kleiner
und $[C]$ gr"o"ser. Wir haben die Verbrauchs- bzw.\ Bildungsgeschwindigkeit $v_A$, $v_B$, $v_C$ gegeben durch
$$ 
\frac d{dt}[A] = -v_A,\quad 
\frac d{dt}[B] = -v_B,\quad 
\frac d{dt}[C] = v_C.$$
Wenn genau ein Molek"ul A und ein Molek"ul B zu einem Molek"ul C reagiert, 
$$ A+B\rightarrow C$$
so muss die Verbrauchsgeschwindigkeit von A gleich der von B, und die Verbrauchsgeschwindigkeiten von A,B gleich der
Bildungsgeschwindigkeit von C sein,
$$ v_A=v_B=v_C.$$
I.a. geht man davon aus, dass die Reaktanten A und B eine Zufallsbewegung (Brownsche Bewegung) folgen, und
manchmal zuf"allig so heftig gegeneinander prallen, dass die Aktivierungsenergie f"ur die Reaktion 
gegeben ist, und sie dann zu C reagieren. 
Je mehr von A vorhanden ist, desto "ofters sollten Zusammenst"o"se erfolgen. Analog, sollte die Zahl
der Zusammenst"o"se pro Zeiteinheit ebenfalls proportional zur Konzentration von B sein. Insgesamt 
ist die Rate der Zusammenst"o"se also proportional zum Produkt der Konzentrationen von A und B. 
Wenn diese Annahme korrekt ist, so wird auch die Verbrauchsgeschwindigkeit 
von A und B proportional zu dem Produkt ihrer Konzentration sein,
$$ v_A=v_B = v_C = k [A][B]$$
wobei $k$ die Reaktionskonstante bezeichnet. Wir erhalten also die Gleichungen
$$ \frac d{dt}[A] = -k [A][B],\qquad \frac d{dt}[A] = -k [A][B],\qquad \frac d{dt}[C] = k [A][B].$$
Man nennt die Modellvorstellung, die besagt dass die Reaktionsgeschwindigkeit proportional
zu den Konzentrationen der Reaktanten ist, auch Massenwirkungsgesetz. 
Dieses System besitzt zwei Erhaltungsgr"o"sen,
$$ \frac d {dt} ([A]+[C])=\frac d {dt} ([B]+[C])=0.$$
I.e., die Zahl der Molek"ule von A plus der Zahl der Molek"ule C ist zeitlich konstant (genauso wie B und C). Dies sind st"ochiometrische Erhaltungss"atze, die bei der Vereinfachung von Reaktionssystemen h"aufig 
eine zentrale Rolle spielen.\par
Ein anderer Fall liegt vor, wenn ein Molek"ul in zwei Molek"ule spontan zerf"allt 
(das ist bei Komplexen h"aufiger 
der Fall). Dieser Vorgang "ahnelt dem radioaktiven Zerfall,
$$ C \rightarrow A+B$$
wird dann beschrieben durch 
$$ \frac d{dt}[A] = k [C],\qquad \frac d{dt}[A] = k [C],\qquad \frac d{dt}[C] = -k [C].$$
Wenn  mehre chemische Reaktionen gleichzeitig aktiv sind, so addiert man die entsprechenden Terme auf der rechten Seite der gew"ohnlichen Differentialgleichung. So wird aus 
\begin{eqnarray*}
A + B    & \mathop{\rightleftharpoons}\limits_{k_{-1}}^{k_1}  C  
\end{eqnarray*}
das System
\begin{center}
\begin{tabular}{cl|c|c}
 && $A+B\rightarrow C$ & $C\rightarrow A+B$\\
\hline
$\frac d {dt}$[A] &=& $- k_1\,[A]\,[B]$&$+k_2\,[C]$\\
$\frac d {dt}$[B] &=& $- k_1\,[A]\,[B]$&$+k_2\,[C]$\\
$\frac d {dt}$[C] &=& $\,\, k_1\,[A]\,[B]$&$-k_2\,[C]$
\end{tabular}
\end{center}
Oder aus 
\begin{eqnarray*}
S + E    & \mathop{\rightleftharpoons}\limits_{k_{-1}}^{k_1}  C  \mathop{\rightarrow}\limits^{k_2} P
\end{eqnarray*}
wird untere Annahme des Massenwirkungsgesetzes
\begin{eqnarray*}
\quad [S]' & = & -k_1 [S][E]+ k_{-1} [C]\\
 \quad[E]' & = &  -k_1 [S][E]+ k_{-1} [C]\\
\quad [C]' & = &  \quad k_1 [S][E]- k_{-1} [C] - k_2 [C]\\
\quad [P]' & = & \qquad\qquad\qquad\qquad\qquad k_2 [C].
\end{eqnarray*}
Bei gr"o"seren Reaktions-Systemen treten verschiedene Probleme auf: Die Reaktionen sind nicht alle bekannt,
die Reaktionsratenkonstanten sind nicht bekannt, verschiedene Reaktionen laufen auf v"ollig verschiedene
Zeitskalen ab -- manche sind sehr schnell, andere sehr langsam. 
Um mit der letzten Eigenschaft umgehen zu k"onnen, ben"otigt man die Methoden der Zeitskalenanalyse.
%===============================================================================
\subsection{Zeitskalen}
Man stelle sich vor, dass man "uber 3 Monate hinweg ein Feld wachsender 
Sonnenblumen alle 30 Minuten photographiert, und dann daraus einen Film
erstellt. Wenn man den Film langsam ablaufen l"asst, so kann man sehen, wie 
die Sonnenblumen (solange sie noch nicht bl"uhen) ihre K"opfe jeden Tag mit der 
Sonne drehen. Man wird dabei aber kaum ein Wachstum bemerken. Man kann 
den Film aber auch schnell ablaufen lassen - jetzt verschwimmt die Bewegung der K"opfe,
aber man kann gut verfolgen wie die Sonnenblumen gr"o"ser werden.

Wir haben also zwei Prozesse (Drehen der K"opfe und Wachsen), und durch die
Geschwindigkeit des Abspielens des Films k"onnen wir entscheiden, welchen der beiden
Prozesse wir studieren wollen. Das ist die Grundidee der Zeitskalenanalyse. 

\fpbox{{\bf Strukur.}
Gegeben sei eine Differentialgleichung des Typs 
\begin{eqnarray*}
\frac d {dt}x &=& \epsilon f(x,y)\\
 \frac d{dt}y &=&  g(x,y)
\end{eqnarray*}
$\varepsilon$ sei klein. Dann ist $\varepsilon f(x,y)$ 
klein, d.h. $x$ "andert sich nur langsam. Daher hei"st $x$ 
die langsame Variable oder langsame Prozess. In Bezug auf $x$ "andert sich
$y$ schnell, hei"st also schnelle Variable bzw. schneller Prozess. 
}
\par\medskip
Um die Idee von oben, die richtige Zeitskala f"ur die Analyse des jeweiligen Prozesses
auszunutzen, m"ussen wir wissen, wie wir die Zeitskala wechseln (den Film schneller/langsamer ablaufen lassen).
Wir k"onnen die Zeit umskalieren: sei $\tau$ gegeben durch
$$\tau = \varepsilon\, t.$$
D.h., wenn $t$ die Zeit in Stunden misst, so misst $\tau$ die Zeit in $\varepsilon$ Stunden,
wobei $\varepsilon$ klein sein muss (typischerweise wenigstens 0.01), d.h. $\tau$ misst die Zeit
in Sekunden. Der Prozess auf der langsamen Zeitskala ist dann 
$$
\tilde x(\tau) = \tilde x(\varepsilon t) = x(t) = x(\tau/\varepsilon),
\qquad
\tilde y(\tau) = \tilde y(\varepsilon t) = y(t) = y(\tau/\varepsilon).$$
Wir erhalten als Differentialgleichung f"ur $\tilde x$
$$ \frac d{d\tau}\tilde x(\tau) 
= \frac d{d\tau}x(\tau/\varepsilon)
= \frac d{d}x(t)\big|_{t=\tau/\varepsilon}\,\,\,\frac d {d\tau}\left(\tau/\varepsilon\right)
= \varepsilon f(x(t), y(t))\big|_{t=\tau/\varepsilon}\,\,\,\frac 1 \varepsilon
= f(x(\tau/\varepsilon),y(\tau/\varepsilon))
= f(\tilde x(\tau),\tilde y(\tau)).
$$
bzw.\ f"ur $\tilde y$
$$ \frac d{d\tau}\tilde y(\tau) 
= \frac d{d\tau}y(\tau/\varepsilon)
= \frac d{d}y(t)\big|_{t=\tau/\varepsilon}\,\,\,\frac d {d\tau}\left(\tau/\varepsilon\right)
= g(x(t), y(t))\big|_{t=\tau/\varepsilon}\,\,\,\frac 1 \varepsilon
= g(x(\tau/\varepsilon),y(\tau/\varepsilon)) \frac 1 \varepsilon
= \frac 1 \varepsilon\,\,g(\tilde x(\tau),\tilde y(\tau)).
$$

Zusammenfassung:\\
\fpbox{
\begin{eqnarray*}
\frac d {d\tau}\tilde x &=&  f(\tilde x,\tilde y)\\
\epsilon\,\frac d{d\tau}\tilde y &=& g(\tilde x,\tilde y)
\end{eqnarray*}}
Wir haben also zwei verschiedene Zeitskalen zur Verf"ugung:\\
$\bullet$ die schnelle Zeitskala $t$, die f"ur den schnellen Prozess $y$ nat"urlich ist\\
$\bullet$ die langsame Zeitskala $\tau$, die f"ur den langsamen Prozess $x$ nat"urlich ist.\\
Entsprechend hei"st das System mit $t$ als Zeit das schnelle System, das mit $\tau$ als Zeit das langsame System.

Nun k"onnen wir die Zeitskalen auf die Spitze treiben, und (formal) $\varepsilon$ gegen Null laufen lassen - siehe
Tabelle \ref{epsnulltabelleeins}.
\begin{table}[h]
\caption{\label{epsnulltabelleeins} Zur Unterscheidung zwischen schnellem und langsamen System.}
\begin{center}
\begin{tabular}{ccl|ccl}
\multicolumn{3}{c|}{Langsames System}&
\multicolumn{3}{c}{Schnelles System}\\
\hline
$\quad$& & $\quad$ & & \\
$\frac d {d\tau} \tilde x(\tau)$ & =  & $f(\tilde x,\tilde y)$ $\qquad$&$\qquad$
   $ \frac d {d t}  x(t)$ & = & $\epsilon f(x,y)$\\
$\epsilon \frac d {d\tau} \tilde y(\tau)$ & = & $ g(\tilde x, \tilde y)$&$\qquad$
        $ \frac d {d t}y $ & =  & $g(x,y)$\\
\multicolumn{6}{c}{$\lim \epsilon\rightarrow 0$} \\
$\quad$& & $\quad$ & & \\
$\frac d {d \tau} \tilde x(t)$ & =  & $f(\tilde x,\tilde y)$ $\qquad$&$\qquad$
   $ \frac d {d t} x$ & = & $0$\\
$ 0$ & = & $ g(\tilde x,\tilde y)$&$\qquad$
        $ \frac d {dt} y $ & =  & $g(x, y)$
\end{tabular}
\end{center}
\end{table}
Nun sehen wir uns hintereinander das schnelle und dann das langsame System an.

\paragraph{\bf (a) Schnelles System.} Hier "andert sich die langsame Variable $x$ 
nicht mehr ($dx/dt=0$). Wir haben ein Differentialgleichung in $y(t)$ alleine, 
in der $x$ die Rolle eines festen Parameters spielt. Wir nehmen an
(und das ist auch h"aufig so), dass $y(t)$ sich langfristig auf einen station"aren 
Zustand setzt,
$$ \lim_{t\rightarrow\infty} y(t) = y_0.$$
Dann ist $y_0$ ein station"arer Punkt, d.h.\ erf"ullt $f(x,y_0)=0$. Wir sehen, 
dass in der Regel 
der Wert $y_0$ von $x$ abh"angen wird; wir erhalten 
eine Funktion $\Phi$ von $x$, 
die aus station"aren Punkten gebildet wird,
$$ y_0 = \Phi(x), \qquad g(x,\Phi(x))=0.$$
In der schnellen Zeitskala wird also $x$ sich nicht ver"andern, 
aber $y(t)$ wird (schnell) zu $\Phi(x)$ streben.
Danach ist das schnelle System im Gleichgewicht, d.h.\ es passiert hier nichts mehr. Man nennt
die Menge $\{(x,\Phi(x))\}$ auch ``langsame Mannigfaltigkeit'' (mathematisch) bzw.\ Quasi-Gleichgewicht
(in den Anwendungen).
%===============================================================================
\begin{figure}[htbp] %  figure placement: here, top, bottom, or page
   \centering
   \includegraphics[width=10cm]{../figures/hmIIblock7zeitskala.pdf}
   \caption{Zeitskalenanalyse. Das schnelle System setzt sich auf die langsame Mannigfaltigkeit.
Hier kann die Dynamik durch das langsame, reduzierte System beschrieben werden.}
   \label{rk:mm:sing}
\end{figure}
%===============================================================================
\paragraph{\bf (b) Langsames System.} Nachdem das schnelle System im Gleichgewicht 
ist, "ubernimmt die langsame Zeitskala. D.h., wir wissen schon, dass 
$\tilde y=\Phi(\tilde x)$ (man beachte, dass die Tilde nur die Zeitskalen
andeutet, und die sind in station"aren Punkten egal). Also ist die Forderung des 
langsamen Systems
$$ 0=g(\tilde x,\tilde y)$$
schon erf"ullt. Wir finden eine Differentialgleichung, diesmal f"ur $\tilde x$ alleine,
$$ \frac d {d\tau}\tilde x = f(\tilde x, \tilde y) = f(\tilde x, \Phi(\tilde x)).$$
Das ist das langsame, reduzierte System.
H"aufig interessiert man sich nicht f"ur das schnelle System -- das ist so schnell im Gleichgewicht,
dass es oft schwer zu beobachten ist -- sondern nur f"ur das langsame System.

%===============================================================================
\subsection{ Michaelis-Menten Kinetik - Zeitskalenargumente}

Wir betrachten eine enzymatische Reaktion: ein Enzym bildet mit einem
\index{Michaelis-Menten-Kinetik}
Substrat einen Komplex. Dieser Komplex kann entweder wieder zerfallen, oder aber
das Substrat wird in ein Produkt verwandelt, und danach spaltet sich das Enzym wieder ab.
Zustand des Systems:\\
$s(t)$ Substrat (zur Zeit $t$)\\
$e(t)$ Substrat (zur Zeit $t$)\\
$c(t)$ Komplex aus Enzym und Substrat (zur Zeit $t$)\\
$p(t)$ Substrat (zur Zeit $t$)\\
%===============================================================================
{\bf Dynamik / Differentialgleichung.} Wir betrachten die chemischen Reaktionen
\begin{eqnarray*}
S + E    & \mathop{\rightleftharpoons}\limits_{k_{-1}}^{k_1} & [SE]\\
 {[SE]}  &  \mathop{\rightarrow}\limits^{k_2} & P+E
\end{eqnarray*}
Unter Annahme des Massenwirkungsgesetztes knnen wir die Reaktionen sofort umsetzten in 
 gew"ohnliche Differentialgleichungen:
\begin{eqnarray*}
\dot s & = & -k_1 s e + k_{-1} c\\
\dot e & = & -k_1 s e + k_{-1} c + k_2 c\\
\dot c & = & k_1 s e - k_{-1} c - k_2 c\\
\dot p & = & k_2 c
\end{eqnarray*}
mit Anfangsbedingungen $s(0) = s_0$, $e(0) = e_0$, $c(0) = 0$ und $p(0) = 0$.
Um die Dimension des Gleichungssystems (die Zahl der Gleichungen) zu erniedrigen
nutzen wir den Massenerhalt: die Masse von freiem Enzym und Enzym das  an das
Substrat gebunden ist, bleibt konstant,
$$ \frac d {dt} (e+c) = 0.$$
d.h.\ $ e+ c = e_0$. Also,
\begin{eqnarray*}
\dot s & = & -k_1 s (e_0-c) + k_{-1} c
        =  -k_1 s e_0 + (k_1 s + k_{-1}) c,\qquad\qquad\qquad s(0) = s_0\\
\dot c & = & k_1 s (e_0-c) - k_{-1} c - k_2 c
        =  k_1 s e_0 - (k_1 s + k_{-1} + k_2) c,\quad\qquad c(0) = 0.\\
\end{eqnarray*}
Nun f"uhren wir Zeitskalen ein. Wir nehmen an, dass die Komplexbildung sehr viel schneller
vonstatten geht als die Reaktion (die G"ultigkeit der Annahme kann man formal beweisen, wenn
die Menge des Enzyms wesentlich keiner ist als die des Substrates, siehe Kapitel~\ref{mmRescal}).
Also schreiben wir $\varepsilon$ vor die Ableitung von $c$, 
\begin{eqnarray*}
\dot s & = & -k_1 s (e_0-c) + k_{-1} c
        =  -k_1 s e_0 + (k_1 s + k_{-1}) c,\qquad\qquad\qquad s(0) = s_0\\
\varepsilon \dot c & = & k_1 s (e_0-c) - k_{-1} c - k_2 c
        =  k_1 s e_0 - (k_1 s + k_{-1} + k_2) c,\quad\qquad c(0) = 0.\\
\end{eqnarray*}
\underline{Strategie:} \\
Wir haben uns vorher "uberlegt, dass wir so ein System am besten verstehen, 
wenn wir zun"achst auf die schnelle Zeitskala gehen, und untersuchen, wie
das System auf der schnellen Zeitskala in ein (Quasi)-Gleichgewicht l"auft. Anschlie"send wechseln wir auf die langsame Zeitskala, und berechnen dann, wie
sich dann das (Quasi)-Gleichgewicht ver"andert.\par\medskip
\underline{Schnelles System:} \\
Wir m"ussen die zeit umskalieren (das System ist oben gegeben in der langsamen Zeitskala). Wir erhalten in der schnellen Skala
\begin{eqnarray*}
s' & = & \varepsilon(k_1 s (e_0-c) + k_{-1} c)
        =  \varepsilon(-k_1 s e_0 + (k_1 s + k_{-1}) c),\qquad\qquad\qquad s(0) = s_0\\
c' & = & k_1 s (e_0-c) - k_{-1} c - k_2 c
        =  k_1 s e_0 - (k_1 s + k_{-1} + k_2) c,\quad\qquad c(0) = 0.\\
\end{eqnarray*}
Mit $\varepsilon\rightarrow 0$ folgt, dass $s'=0$, i.e.\ $s$ konstant ist. 
F"ur $c$ haben wir also eine lineare Differentialgleichung,
$$
c'  =  k_1 s (e_0-c) - k_{-1} c - k_2 c
        =  k_1 s e_0 - (k_1 s + k_{-1} + k_2) c,\quad\qquad c(0) = 0.\\
$$
f"ur die wir die L"osung ausrechnen k"onnen (siehe die Lehreinheit "uber lineare Differentialgleichungen)
$$ c(t) = e^{-(k_1 s + k_{-1} + k_2) t} c_0 
+ (1-e^{-(k_1 s + k_{-1} + k_2)t}) \,\, 
\frac{s k_1e_0}{ k_1s + (k_{-1} + k_2)} .$$
Wir finden also
$$ \lim_{t\rightarrow\infty}c(t) = \frac{s k_1 e_0}{ s + (k_{-1} + k_2)}  =
\frac{s e_0}{ s + (k_{-1} + k_2)/k_1} .$$
Man definiert die Michaelis-Menten Konstante 
$K_m = (k_{-1} + k_2)/k_1$, 
(eigentlich: Briggs-Haldene-Konstante, 
da Michaelis und Menten eine spezielle Situation, n"amlich $k_{-1}\ll k_2$ betrachtet haben,
und folgerichtig $K_m=k_2/k_1$ approximiert haben)
so k"onnen wir schreiben
$$ c(t)\rightarrow C(s) = \frac{s e_0}{ s + K_m}.$$
\underline{Langsames System:} \\
Lassen wir jetzt $\varepsilon\rightarrow 0$ in der langsamen Zeitskala, 
so finden wir 
\begin{eqnarray*}
\dot s & = & -k_1 s (e_0-c) + k_{-1} c
        =  -k_1 s e_0 + (k_1 s + k_{-1}) c,\qquad\qquad\qquad s(0) = s_0\\
0 & = & k_1 s (e_0-c) - k_{-1} c - k_2 c
        =  k_1 s e_0 - (k_1 s + k_{-1} + k_2) c.
\end{eqnarray*}
Insbesondere setzen wir also das Quasi-Gleichgewicht 
f"ur $c$ voraus, $c=C(s)$.
Wenn wir damit in die Gleichung f"ur $s'$ gehen (um das langsame System im 
quasi-station"aren Zustand zu studieren), so finden wir
\begin{eqnarray*} \dot s 
&= &
  -k_1 s e_0 + (k_1 s + k_{-1}) C(s)
 =          -k_1 s e_0 + (k_1 s + k_{-1}) \frac{s e_0}{ s + K_m}\\
&= &\frac{-k_1 s e_0(s + K_m)+ (k_1 s + k_{-1})s e_0}{s + K_m}
= \frac{(k_1-k_{-1}K_m)s e_0}{s + K_m}
= \frac{(k_{-1}-k_{1}K_m)s e_0}{s + K_m}\\
&=& \frac{(k_{-1}-(k_{-1} + k_2))s e_0}{s + K_m}
= \frac{- k_2 s e_0}{s + K_m}
\end{eqnarray*}
Das ist die ber"uhmte Michaelis-Menten Kinetik: Wenn viel Substrat vorhanden ist 
($s\gg K_m$), so wird mit einer ann"ahernd konstanten Geschwindigkeit $\approx k_2$ 
das Substrat abgebaut; das Enzym arbeitet in S"attigung. 

Wenn allerdings das Substrat wenig wird ($s\ll K_m$), so ist die Abbaugeschwindigkeit 
ann"ahernd proportional zu der Enzym-Menge, $\approx (k_2/K_m) s$. Die H"alfte
der maximalen Abbaugeschwindigkeit wird genau bei $s=K_m$ erreicht.

Wenn wir annehmen, dass wir wenige Enzym im System haben, muss die Bildungsgeschwindigkeit der Produkts
der Abbaugeschwindigkeit des Substrates einigermassen gleichen, d.h.
$$ \frac d {dt} s = -\frac{k_2 e_0 s}{K_m+s},\qquad \frac d {dt} p = \frac{k_2 e_0 s}{K_m+s}.$$
Wir haben also die urspr"unglich vier Gleichungen (f"ur S, E, [ES], P) auf zwei Gleichungen gedr"uckt.

%===============================================================================
 \subsubsection{Michaelis-Menten-Plot und Lineweaver-Burks-Plot}

Um die Parameter der Michaelis-Menten-Kinetik aus Daten zu 
gewinnen, wurden zahlreiche graphische Methoden entwickelt. Wir sehen uns hier zwei 
Methoden an.

\noindent{\bf Michaelis-Menten-Plot.} 
Es ist m"oglich die Abbaugeschwindigkeit $V$ f"ur verschiedene Substrat-Konzentrationen 
zu messen. Wenn wir $V$ "uber $s$ auftragen, so erhalten wir 
$$ V = \frac{V_{max} s}{s+K_m}$$
(wobei wir $k_2$ in $V_{max}$ umgetauft haben, den Gepflogenheiten der Enzymchemie folgend)
und den Michaelis-Menten-Plot.\\
Wir m"ussen dann versuchen, die Asymptotik von $V$ 
f"ur $s\rightarrow\infty$ zu sch"atzen; 
das ergibt die maximale Abbaugeschwindigkeit $V_{max}$. Dann k"onnen
wir $K_m$ durch $V(K_m)=V_{max}$ ablesen.
%===============================================================================
\begin{figure}[htbp] %  figure placement: here, top, bottom, or page
   \centering
   \includegraphics[width=17cm]{../figures/hmIIblock7mm.pdf}
   \caption{Michaelis-Menten-Plot und Lineweaver-Burks-Plot.}
   \label{rk:mm:sing}
\end{figure}
%===============================================================================
Der Michaelis-Menten-Plot ist sozusagen der Funktionsgraph; der Lineweaver-Burks-Plot
linearisiert durch eine Koordinatentransformation den Funktionsgraphen und erm"oglicht
durch die Linearisierung die Anwendung der linearen Regression zur Bestimmung
der Parameter $K_m$ und $V_{max}$.
%===============================================================================

\noindent{\bf Lineweaver-Burks-Plot.} Wenn wir den Kehrwert der Gleichung
$$ V = \frac{V_{max} s}{s+K_m}$$
nehmen, erhalten wir
$$ \frac 1 V = \frac{s+K_m}{V_{max} s} = \frac 1 {V_{max}} + \frac{K_m}{V_{max}}\,\, \frac 1 s.$$
D.h., $1/V$ und $1/s$ erf"ullen eine lineare Beziehung; tragen wir Daten f"ur $1/V$ gegen $1/s$ auf, so sollten
diese Datenpunkte ann"ahernd auf einer Geraden liegen. Das ist der 
 Lineweaver-Burks-Plot. 
Lineare Regression erlaubt Achsenabschnitt und
Steigung zu sch"atzen. Der Kehrwert des Achsenabschnitts ist dann gerade $V_{max}$. Multipliziert man die Steigung
mit $V_{max}$, so erh"alt man $K_m$.

%===============================================================================
\subsection*{Aufgaben}
\begin{auf}\cha\label{block7A1}
\input{../../Aufgabensammlung/hm708.tex}
\end{auf}

\begin{auf}\che\label{block7A2}
\input{../../Aufgabensammlung/hm709.tex}
\end{auf}


%===============================================================================




\subsection{Appendix: Zeitskalen in Michaelis-Menten II} 
\label{mmRescal}
{\it Um\footnote{Dieser Abschnitt wird in der Vorlesung nicht behandelt} zu verstehen, dass die unterschiedlichen Zeitskalen eine Konsequenz der unterschiedlichen 
Konzentrationen von Enzym und Substrat sind, m"ussen wir reskalieren. Die Idee hierbei ist,
dass wir die Einheiten der Gr"o"sen w"ahlen k"onnen. Wir m"ussen die Zeit nicht
in Sekunden messen, sondern k"onnen eine Zeiteinheit auch auf einen anderen Wert festlegen
(z.B. $1.245 $ h). Genauso m"ussen die Dichten der Substanzen nicht in Mol angegeben werden,
 sondern k"onnen auch in Einheiten von $2.45 \mu$ Mol o."a.\ beschrieben werden.\par
 Dieses Umskalieren liefert Freiheitsgrade (zus"atzliche Konstanten, die wir einschmuggeln).
 Diese Konstanten k"onnen geschickt gew"ahlt werden, sodass das resultierende System 
 einfacher wird. Wir w"ahlen folgende Einheiten/Reskalierungen
\begin{eqnarray*}
u(\tau) = \beta s(\alpha\tau),\qquad 
v(\tau) = \gamma c(\alpha \tau).
\end{eqnarray*}
Die Konstanten $\alpha$, $\beta$ und $\gamma$ werden wir sp"ater so w"ahlen, dass das System m"oglichst einfach wird.
Wir erhalten
\begin{eqnarray*}
\frac d {d\tau}u(\tau) & = & \alpha \beta s'(\alpha\tau)\\
& = & \alpha\beta [ -k_1 e_0 s(\alpha\tau) + (k_1 s(\alpha\tau)+k_{-1}) c(\alpha\tau) ]\\
& = & [ -\alpha k_1 e_0 \beta s(\alpha\tau) + (k_1 \beta s(\alpha\tau)+k_{-1}\beta ) \frac{\alpha}{\gamma} \gamma c(\alpha\tau) ]\\
& = & -\alpha k_1 e_0 u(\tau) + (k_1 u(\tau)+k_{-1}\beta ) \frac{\alpha}{\gamma} v(\tau) \\
& = &  -\alpha k_1 e_0 u(\tau) + (k_1 u(\tau)+k_{-1}\beta ) \frac{\alpha}{\gamma} v(\tau) \\
& = &  -\alpha k_1 e_0 u(\tau) + (u(\tau)+\frac{k_{-1}\beta}{k_1} ) \frac{k_1 \alpha}{\gamma} v(\tau) 
\end{eqnarray*}
Aus Gr"unden, die gleich klar werden, betrachten wir die mit $e_0/s_0$ multiplizierte Ableitung von $v(\tau)$:
\begin{eqnarray*}
\frac{e_0}{s_0}\frac d {d\tau}v(\tau) & = & \frac{e_0}{s_0}\alpha \gamma c'(\alpha\tau)\\
& = &\frac{e_0}{s_0} (\alpha \gamma k_1  e_0  \frac {\beta}{\beta}s(\alpha\tau) - (k_1 \frac {\beta}{\beta}s(\alpha\tau) + k_{-1} + k_2) \alpha \gamma c(\alpha\tau) )\\
& = &\frac{e_0}{s_0} ( \frac {\alpha \gamma k_1  e_0 }{\beta}u(\tau) - (\frac {k_1}{\beta}u(\tau) + k_{-1} + k_2) \alpha v(\tau) )\\
& = & \frac{e_0}{s_0}( \frac {\alpha \gamma k_1  e_0 }{\beta}u(\tau) - (u(\tau) + \frac{(k_{-1} + k_2)\beta}{k_1}) \alpha v(\tau) )\\
& = & \frac {\alpha \gamma k_1  e_0^2 }{s_0\beta}u(\tau) - \left(u(\tau) + \frac{(k_{-1} + k_2)\beta}{k_1}\right) \frac{k_1\, e_0}{s_0\beta}\alpha v(\tau) \\
\end{eqnarray*}
Nun setzen wir
\begin{eqnarray*}
\alpha = 1/(k_1 e_0),\qquad \beta = 1/s_0,\qquad \gamma = 1/e_0,\\
\lambda = \frac{k_2}{k_1 s_0},\qquad
K = \frac{k_{-1}+k_2}{k_1 s_0} =: \frac{K_m}{s_0},\qquad
\epsilon = \frac{e_0}{s_0}.
\end{eqnarray*}
Damit erhalten wir
\begin{eqnarray*}
         \dot u & = & -u + (u+K-\lambda) v,\qquad u(0) = 1\\
\epsilon \dot v & = & u - (u+K) v,\qquad\qquad\quad v(0) = 0.
\end{eqnarray*}
%===============================================================================
\begin{figure}[htbp] %  figure placement: here, top, bottom, or page
   \centering
   \includegraphics[width=6cm]{../figures/rk_mm_sing.pdf}
   \caption{Michalis-Menten: Langsame und schnelle Dynamik.}
   \label{rk:mm:sing}
\end{figure}
%===============================================================================
Die Hauptannahmen nun ist, dass es wenig Enzym im Verh"altnis zum Substrat
gibt, d.h.
$$ \epsilon = e_0/s_0\ll 1.$$
Nun sind wir in der Lage, die Analyse von eben nochmals durchzuf"uhren (das ist tats"achlich die \underline{gleich} Analyse wie oben). Wir haben  einen schnellen Prozess ($u$) und einen langsamen Prozess~($v$).\par\medskip
Wir k"onnen nun aber noch einmal die Zeit reskalieren, um uns auf die Zeitskala (Zeiteinheit) des schnellen Prozesses zu setzten. Definieren wir
$$\tau = \epsilon \sigma$$
und 
$\tilde u(\sigma) = u(\sigma\,\epsilon)$, $\tilde v(\sigma) = v(\sigma\,\epsilon)$, so erhalten wir
\begin{eqnarray*}
         \tilde u' & = & \epsilon (-\tilde u + (\tilde u+K-\lambda) \tilde v),\qquad u(0) = 1\\
\tilde v' & = & \tilde u - (\tilde u+K) \tilde v,\qquad\qquad\quad v(0) = 0.
\end{eqnarray*}
\index{Zeitskalen}
In dieser Zeitskala ist also die Ableitung von $\tilde u$ klein, w"ahrend die von $\tilde v$ sich
im ``normalen'' Ramen bewegt. In beiden Zeitskalierungen bleibt der Schluss, dass $u$
sich langsamer ver"andert als~$v$.\par
Es ist also nat"urlich, im $\tau$-System ($u$, $v$) den Prozess $u$ anzusehen, w"ahrend das
$\sigma$-System ($\tilde u$, $\tilde v$) f"ur den $v$-Prozess zust"andig ist.
Um dies zu bewerkstelligen, lassen wir $\epsilon$ in den jeweiligen Systemen 
gegen Null gehen - siehe Tabelle \ref{epsnulltabelle}.
\begin{table}[h]
\begin{center}
\caption{\label{epsnulltabelle} Zur Unterscheidung zwischen schnellem und langsamen
System bei der Zeitskalenanalyse der Michaelis-Menten-Kinetik (zweiter Teil).}
\begin{tabular}{ccl|ccl}
\multicolumn{3}{c|}{Langsames System}&
\multicolumn{3}{c}{Schnelles System}\\
\hline
$\quad$& & $\quad$ & & \\
$\frac d {d\tau} u(\tau)$ & =  & $-u + (u+K-\lambda) v$ $\qquad$&$\qquad$
   $ \frac d {d\sigma} \tilde u(\sigma)$ & = & $\epsilon(-\tilde u + (\tilde u+K-\lambda) \tilde v)$\\
$\epsilon \frac d {d\tau} v(\tau)$ & = & $ u-(u+K) v$&$\qquad$
        $ \frac d {d\sigma}\tilde v $ & =  & $\tilde u-(\tilde u+K) \tilde v$\\
\multicolumn{6}{c}{$\lim \epsilon\rightarrow 0$} \\
$\quad$& & $\quad$ & & \\
$\frac d {d\tau} u(\tau)$ & =  & $-u + (u+K-\lambda) v$ $\qquad$&$\qquad$
   $ \frac d {d\sigma} \tilde u$ & = & $0$\\
$ 0$ & = & $  u-( u+K) v$&$\qquad$
        $ \frac d {d\sigma}\tilde v $ & =  & $\tilde u-(\tilde u+K) \tilde v$
\end{tabular}
\end{center}
\end{table}

\paragraph{\bf (A) Schnelles System.} $\tilde u$ "andert sich nicht (Ableitung ist 
Null). F"ur $v$ finden wir die lineare Differentialgleichung
$$\frac d {d\sigma}\tilde v =   \tilde u-(\tilde u+K) \tilde v, \qquad v(0) = v_0.$$
Diese lineare Differentialgleichung kann man l"osen und findet
$$ \tilde v(\sigma) = v_0 e^{-(\tilde u + K)\sigma} + \frac{\tilde u} {\tilde u + K} (1-e^{-(\tilde u + K)\sigma})
$$
F"ur $\sigma\rightarrow\infty$ gilt also
$$
\tilde v(\sigma)\rightarrow 
\frac{\tilde u} {\tilde u + K}
$$
Das $\tilde u$-System setzt sich also (schnell) auf die Kurve
$$
\tilde v = 
\frac{\tilde u} {\tilde u + K}
$$
Die Interpretation ist, dass die Komplexbildung von Enzym und Substrat schnell in sein
Gleichgewicht geht (ohne viel an der Masse des Substrates zu ver"andern, i.e.\ ohne 
(in dieser kurzen Zeit bis das Gleichgewicht sich einstellt) viel Produkt zu bilden.
Danach bestimmt das langsame System die Dynamik.

\paragraph{\bf (B)Langsames System (Quasi-Gleichgewicht).} Nun nehmen wir an, dass 
das schnelle System im Gleichgewicht ist, d.h.
$$ 
 u-(u+K) v =0\qquad 
\Leftrightarrow\qquad
v = \frac{u}{u+K}\qquad
$$
Dann ist die Gleichung f"ur $v$ erf"ullt. Wir erhalten die Differentialgleichung f"ur $u$
$$ \dot u 
= -u + (u+K-\lambda)\frac{u}{u+K}
= -u + u -\frac{\lambda}{u+K} u = -\frac{\lambda u}{u+K}.
$$
}