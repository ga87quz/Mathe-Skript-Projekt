

%% ! T E X  root=hmIIneu.tex
% !TEX root=MA9603.WZW.tex
% !TEX program = pdflatex
% !TEX spellcheck = de_DE


%=============================================================================== 
\section{Kurvenintegrale, Integralsatz von Gau"s und Erhaltung von Masse}
%=============================================================================== 

\zbox{
{\bf Ziele}:
\begin{itemize}
\item Definition des Kurvenintegrals verstehen
\item Kurvenintegrale berechnen k"onnen
\item Gau"sscher Integralsatz kennen und anwenden k"onnen
\item Erstes Ficksches Gesetz kennen und interpretieren k"onnen
\end{itemize}}
  
%=============================================================================== 
\subsection{Kurven und ihre L"ange im $\R^n$}
\label{kurvexxx}
%=============================================================================== 
{\it Kurve und Tangentialvektor (Geschwindigkeitsvektor)}\par

Eine \emph{Kurve} im $\R^n$ k"onnen wir uns letztlich als eine (krumme) Linie im 
$\R^n$ vorstellen. 

\begin{sdefi} 
Eine Kurve eine Abbildung eines Intervalls $I=[a,b]\subseteq \R$ nach $\R^n$,
$$ K: I\rightarrow \R^n.$$
\end{sdefi}
{\bf Interpretation.} Wenn das Argument $t$ das Intervall $I$ durchl"auft, 
durchl"auft entsprechend $K(t)\in\R^n$ die Kurve.  Die Kurve kann als Beschreibung der Bewegung eines
Punktes im $n$-dimensionalen Raum interpretiert werden. Man kann sich eine Kurve 
gut vorstellen, als den Weg, den ein Auto in einem bestimmten Zeitintervall 
zur"uckgelegt hat. Das Auto startet zur Zeit $t_1$ und beendet den Weg zur Zeit 
$t_2$, d.h. ist im Zeitintervall $I=[t_1, t_2]$ unterwegs. Zu jedem Zeitpunkt $t\in I$
gibt $K(t)$ den Ort des Autos an.

Das Argument $t\in I$ spielt also die Rolle der Zeit, und unser Punkt befindet 
sich zur Zeit $t\in I$ am Ort $K(t)\in\R^n$. Dann kann man
den \emph{Geschwindigkeitsvektor} (der wird auch \emph{Tangentialvektor} genannt) 
berechnen; dieser ist gerade die Ableitung ("Anderung des Ortes pro Zeit)
$$ v(t) = \dot K(t).$$
Damit haben an jedem Punkt $t\in I$ einen  Vektor $v(t)$ ausgezeichnet (solange $v\not = 0$).

%=============================================================================== 
\begin{sbem} Man kann einen Weg mit dem Auto zweimal abfahren, aber mit v"ollig verschiedenen Geschwindigkeiten.
Das hei"st, man kann eine Kurve umparametrisieren. Wenn man eine (monoton wachsende) Abbildung 
vom Interval $I_1$ in das Intervall $I_2$ definiert, $f:I_1\rightarrow I_2$, so wird
$$ K_1: I_1 \rightarrow \R^n,\qquad t\mapsto K_1(t)$$
und
$$ K_2: I_2 \rightarrow \R^n,\qquad t\mapsto K_1(f(t))$$
die gleichen Punkte ``abfahren'', aber i.d.R.\ mit unterschiedlicher Geschwindigkeit. 
Also sind $K_1$ und $K_2$ zwei verschiedene Kurven, aber manche Aspekte sollten gleich bleiben. 
Z.B.\ h"angt die L"ange des Weges (der Kurve) nicht von der Geschwindigkeit (der Parametrisierung) 
ab.
\end{sbem}
%===============================================================================
\subsubsection{L"ange einer Kurve}
\index{Kurve!L"ange}
Im Allgemeinen gilt, dass bei gegebener Geschwindigkeit (oder besser, Betrag der Geschwindigkeit) 
$\|v(t)\|$ die in der Zeit von $t_1$ bis zu der Zeit $t_2$ zur"uckgelegt wird, genau
$$ \int _{t_1}^{t_2}\|v(t)\|\, dt$$
ist. Daher ist die L"ange der Kurve $K(t)$ gegeben durch
$ L(K) = \int _{a}^{b}\|v(t)\|\, dt = \int_a^b \|\dot K(t)\|\, dt$. Dies ist eine Definition.

\begin{sdefi} Gegeben eine differenzierbare Kurve $ K: I\rightarrow \R^n.$ Die L"ange dieser Kurve ist gegeben durch 
$$ L(K) = \int _{a}^{b}\|v(t)\|\, dt = \int_a^b \|\dot K(t)\|\, dt$$
\end{sdefi}

%===============================================================================
\begin{sbem}
Eigentlich haben wir hier das Pferd von hinten aufgez"aumt: Warum ist die Strecke,
die man zur"ucklegt, wenn man mit Geschwindigkeitsvektor $v(t)$ f"uhrt, gleich
$\int_{t_1}^{t_2}\|v(t)\|\, dt$? Nun, weil die Kurve $K(t)$ mit $\dot K(t)=v(t)$ gerade 
die in der Definition gegebene L"ange besitzt (Geschwindigkeit = L"ange/Zeit). 
Die heuristische Begr"undung, die wir oben angef"uhrt haben, ist also plausibel, bei"st sich aber
in den Schwanz. Wir ben"otigen ein besseres Argument. 
\end{sbem}
%===============================================================================

\begin{figure}[htbp] %  figure placement: here, top, bottom, or page
   \centering
   \includegraphics[width=10cm]{../figures/kurvenlaenge.pdf} 
   \caption{Approximation einer Kurve mittels Polygonzug, um dessen L"ange zu approximieren.}
   \label{fig:kl}
\end{figure}
%===============================================================================
\fpbox{
{\it Definition der Kurvenl"ange, die Zweite.} 
Nun wollen wir nochmal uns der Kurvenl"ange zu n"ahern. Diesmal direkt.
Sei wieder $K:I=[a,b]\rightarrow\R^n$ eine Kurve. 
Wir diskretisieren die Kurve, und approximieren sie durch einen Polygonzug (st"uckweise gerade Strecken; siehe Abb.~\ref{fig:kl}). Wir betrachten
also die Zeitpunkte $t_i = a+h\,i$, $i=0,1,2,..$ und $h=(b-a)/n$ (f"ur $n\in\N$, gro"s). Die L"ange der Kurve sollte dann in
etwa der L"ange des Polygonzugs entsprechen.  
Die L"ange eines St"uckchens ist $ \|K(t_{i+1})-K(t_i)\|$, d.h.
$$ L \approx \sum_{i=0}^{n-1}\|K(t_{i+1})-K(t_i)\|.$$ 
Nun machen wir es uns
zu Nutze, dass wir von $t_i$ nach $t_{i+1}$ nur ein kleines St"uckchen weit springen, und deshalb praktisch
linear rechnen k"onnen,
$$ K(t_i+h) = K(t_i) + h K'(t_i) + \mbox{Kleiner Fehler} = K(t_i)+h \, v(t_i)+\mbox{kleiner Fehler}.$$
Daher folgt, dass 
$$ L 
\approx  \sum_{i=0}^{n-1}\|K(t_{i+1})-K(t_i)\|
 \approx \sum_{i=0}^{n-1}\|v(t_i) h\|
= \sum_{i=0}^{n-1}\|v(t_i)\| h
\rightarrow \int_a^b \|v(t)\|\, dt.
$$ 
Wir finden wieder die gleiche Definition der L"ange.
}
%===============================================================================
\begin{bspX}
Betrachte die Kurve
$$ K: I=[0,2\pi]\rightarrow\R^3,\qquad t\mapsto 
\left(\begin{array}{c} \sin(t)\\ \cos(t)\\ t\end{array}\right).$$
Dann ist 
$$ v(t) = \left(\begin{array}{c} \cos(t)\\ -\sin(t)\\ 1\end{array}\right)
\qquad\mbox{{\rm und}}\qquad
\|v(t)\| = \sqrt{1+\sin^2(t)+\cos^2(t)} = \sqrt{2}.$$
Also folgt, dass
$$ L = \int_0^{2\pi}\|v(t)\|\, dt = 2^{3/2}\pi.$$
\end{bspX}
%===============================================================================
\begin{auf}\chb\label{block9A1}
\input{../../Aufgabensammlung/hm721.tex}
\end{auf}
%===============================================================================
\subsubsection{Kurvenintegral "uber eine skalare Funktion}
\index{Kurvenintegral} \index{Kurvenintegral!skalare Funktion}
%===============================================================================
Neben einer Kurve $K:I=[a,b]\rightarrow\R^n$ betrachten wir nun gleichzeitig 
noch eine Funktion $g:\R^n\rightarrow \R$. 

Anschaulich: man stelle sich vor, dass wir die Kurve aus einem Material herstellen, 
dessen Dichte variiert, d. h. die Funktion $g$ ist das Gewicht pro L"angeneinheit an einem 
Punkt der Kurve. Ziel ist nun, die Gesamtmasse der Kurve zu berechnen, d.h. diese Dichte
aufzuintegrieren. Das Kurvenintegral $ \int_{K} g(K) dK$ ist also das Integral 
dieser Funktion "uber der Kurve $K(t)$. Um dieses Integral zu definieren sehen wir uns eine spezielle
Parametrisierung an: Nehmen wir an, dass $\|\dot K(t)\|=\|v(t)\|\equiv 1$, d.h.\ nehmen wir an, dass wir die Kurve mit konstanter Geschwindigkeit, 
n"amlich eine L"angeneinheit pro Zeiteinheit, durchlaufen. Wenn wir $f:I\rightarrow\R$, $t\mapsto g(K(t))$, so ist es klar, wie man die Funktion $f$ 
"uber $I$ integriert. Wir erhalten also in diesem Spezialfall das Kurvenintegral: 
\begin{sdefi} Sei $ K: I\rightarrow \R^n$ eine differenzierbare Kurve, $g:\R^n\rightarrow \R$, und $\|\dot K(t)\|=1$. Dann ist das Kurvenintegral gegeben durch
$$ \int_{K} g(K) dK = \int_a^b f(t)\, dt = \int_a^bg(K(t))\, dt
$$
\end{sdefi}
Nun m"ussen wir noch verstehe, was passiert, wenn eine Parametrisierung 
$\|\dot K(\tau)\|=1$ nicht erf"ullt. Das Gewicht der Kurve im obigen Beispiel 
darf nat"urlich (wie die L"ange der Kurve) nicht von der Parametrisierung abh"angen.
Die richtige Art, das Integral Geschwindigkeitsunabh"angig zu machen, haben wir 
bei der L"angenberechnung gesehen: wir m"ussen den Intendant mit dem Betrag der 
Geschwindigkeit multiplizieren.
\begin{sdefi} Sei $ K: I\rightarrow \R^n$ eine differenzierbare Kurve, und  $g:\R^n\rightarrow \R$. Dann ist das Kurvenintegral 
von $g$ "uber $K$ definiert als 
$$ \int_{K} g(K) dK  = \int_a^bg(K(t))\,\|\dot v(t)\|\, dt
$$
\end{sdefi}
%===============================================================================
\begin{bspX}
Betrachte die Kurven
$$ K_1: I_1=[0,2\pi]\rightarrow\R^3,\qquad t\mapsto 
\left(\begin{array}{c} \sin(t)\\ \cos(t)\\ t\end{array}\right)
$$
und deren Umparametrisierung
$$ K_2: I_2=[0,\pi]\rightarrow\R^3,\qquad t\mapsto 
\left(\begin{array}{c} \sin(2t)\\ \cos(2t)\\ 2t\end{array}\right)
$$
Weiter, definiere die Funktion $g:\R^3\rightarrow \R$, $(x,y,z)^T\mapsto z^2$.
Dann ist 
$$ v_1(t) = \left(\begin{array}{c} \cos(t)\\ -\sin(t)\\ 1\end{array}\right)
\qquad\mbox{\rm und}\qquad
\|v_1(t)\| = \sqrt{1^2+\sin^2(t)+\cos^2(t)} = \sqrt{2}$$
bzw.\
$$ v_2(t) = \left(\begin{array}{c} 2\cos(2t)\\ -2\sin(2t)\\ 2\end{array}\right)
\qquad\mbox{{\rm und}}\qquad
\|v_2(t)\| = \sqrt{2^2 +4\sin^2(2t)+4\cos^2(2t)} = \sqrt{8} = 2\sqrt{2}.
$$
Damit erhalten wir die Kurvenintegrale
\begin{eqnarray*}
\int_{K_1}g(K) dK & = & \int_0^{2\pi} t^2 \sqrt{2}\, dt = \frac {\sqrt{2}}3 (2\pi)^3\\
\int_{K_2}g(K) dK & = & \int_0^{\pi} (2t)^2 \,2\,\sqrt{2}\, dt = \frac {\sqrt{2}\, \cdot\,2\,\cdot\, 2^2}3 (\pi)^3 = \int_{K_1}g(K) dK. 
\end{eqnarray*}
Der Wert des Integrals ist also tats"achlich f"ur die beiden Parametrisierungen gleich.
\end{bspX}
%===============================================================================
\subsubsection{Fluss "uber eine Kurve}
Eine der Hauptanwendungen in diesem Skript ist die Berechnung von
 Massen"anderungen in einem Gebiet durch einen Fluss von Partikeln. Wir haben
eine Dichte $u$ von Partikeln gegeben (man stelle sich Sandk"orner auf einem
Tablett vor), die sich mit Geschwindigkeit $v_D$ bewegen (das Tablett wird 
verschoben). Man hat eine Kurve gegeben (man spannt einen faden "uber das 
 Tablett), und fragt sich, wie viel Masse pro Zeiteinheit "uber die Kurve 
flie"st 
(wie viele Sandk"orner pro Zeiteinheit laufen unter dem Faden durch, siehe
Abb.~\ref{fig:flss}).
%===============================================================================

\begin{figure}[htbp] %  figure placement: here, top, bottom, or page
   \centering
   \includegraphics[width=14cm]{../figures/fluss.pdf} 
   \caption{Wie viele Sandk"orner rutschen unter dem Faden durch, wenn man das Tablett mit dem Sand bewegt? Die grau markierte Fl"ache (d.h.\ die Menge der Sandk"orner), die in dem gegebenen Zeitintervall 
unter dem Faden durchtaucht,  h"angt nur von 
der Geschwindigkeitskomponente normal zur Kurve $K$ ab.}
   \label{fig:flss}
\end{figure}
%===============================================================================

\par\medskip
Wir starten mit dem einfachsten Fall: die Massendichte $u$ ist konstant (in Raum und Zeit), die Geschwindigkeit mit der sich die Partikel bewegen ist konstant, und die Kurve ist gegeben durch die Strecke
$$ K:[0,s_0]\rightarrow\R^2,\quad s\mapsto 
\left(\begin{array}{c}a\\b
\end{array}\right)s,\qquad a,b\geq 0,\quad \sqrt{a^2+b^2}=1.$$
Wir haben die Geschwindigkeit der Kurve hier also schon auf eins normiert. 
Wie gro"s ist der Massenfluss? Wenn wir das Tablett entlang der Kurve verschieben (d.h. $v_D$ in parallel zur Tangente $(a,b)^T$ steht), so wechselt
kein Sandkorn die Seite der Kurve. Die Bewegung in tangentialer Richtung ist 
also uninteressant, interessant ist allein die Bewegung in Richtung der normalen der Kurve. Sei
$$ n = \left(\begin{array}{c}-b\\a
\end{array}\right)$$
die Normale.
{\it Achtung:} wir k"onnen zwei normale angeben, das gew"ahlte $n$, 
wie auch $-n$ stehen senkrecht auf dem Tangentenvektor. Das hat eine Bedeutung: Alle Teilchen (Sandk"orner), die in Richtung von $n$ die Seite 
der Kurve wechseln werden positiv gez"ahlt, alle Sandk"orner, die in Richtung $-n$ wechseln, negativ. Wir bilanzieren nur den Netto-Massenfluss, d.h.\ wir
fragen, wie viele Sandk"orner nach einem kurzen Zeitintervall mehr auf der
Seite der Kurve sind, in die $n$ zeigt.\par
Da die Geschwindigkeit in Richtung $n$ gerade durch die Projektion von $v_D$
auf $n$, d.h. dem Skalarprodukt $v_D n = V_D^Tn = <v_D,n>$ gegeben ist, 
so finden wir pro Zeiteinheit an einem Punkt der Kurve den Massenfluss 
\begin{eqnarray*}
 &&\mbox{Massenfluss an einem Punkt der Kurve}\\
& =& \mbox{Massendichte x Geschwindigkeit in Normalenrichtung} = u v_D\, n.
\end{eqnarray*}
\fpbox{{\bf Fluss:}\ 
Wir bezeichnen als Fluss (oder Massenfluss) $J$ das Produkt von Massendichte 
und Geschwindigkeit; wenn Dichte $u(x,t)$ von Ort und zeit abh"angt, und Geschwindigkeit $v(x,t)$ von Ort und Zeit, so definieren wir
$$ J(x,t) = u(x,t) v(x,t).$$
Da $v(x,t)$ ein (Geschwindigkeits-)Vektor ist, so ist auch $J(x,t)$ ein Vektor.
}
\par\medskip
Kehren wir zu unserm einfachen Fall ($u$ und $v_D$ konstant) zur"uck, so 
ist $j= u v_D$ ebenfalls konstant, und der Massenfluss "uber die Kurve an einem Punkt ist gerade $j\, n$. Um den Massenfluss "uber die gesamte Kurve
 zu erhalten, m"ussen wir "uber die Kurve Integrieren,
$$ \mbox{Massenfluss "uber K} = \int_K n^T\,J\,\,dK.$$

\fpbox{{\bf Massen"anderung in einem Gebiet} Sei $\Omega\subset\R^2$ ein Gebiet. 
Die Kurve, die das Gebiet umrandet, nennen wir $\partial\Omega$. 
Wir geben eine Dichte $u(x,t)$ und ein Geschwindigkeitsfeld $u(x,t)$
vor. Es soll keine Masse erzeugt oder vernichtet werden, die Partikel sollen 
 sich nur bewegen.\par
Die Masse $m(t)$ im Gebiet ist gegeben durch $m(t)=\int_{\Omega} u(x,t)$. 
Wie "andert sich die Masse im Gebiet? Da weder Teilchen erzeugt noch vernichtet
werden, kann sich die Masse nur dadurch ver"andern, dass Teilchen  "uber den
 Rand des Gebietes flie"sen. Sei $n$ die "au"sere Normale auf die Kurve
 $\partial\Omega$ (also die Normale, die aus dem Gebiet heraus zeigt), so 
bedeutet ein positiver Nettofluss, dass Masse aus dem Gebiet heraus flie"st,  
d.h.\ $m'(t)<0$ ist. Wir erhalten
$$ \frac d {dt} m(t) = \int_{\partial \Omega}  n^T\,J\,\,do$$
Bemerkung: Eigentlich m"ussten wir $\int_{K}  n^T\,J\,\,dK$ f"ur die Kurve $K=\partial\Omega$ schreiben, um die gleiche Notation wie oben zu nutzen.
Man findet in der Literatur aber eher diese Schreibweise ($\partial\Omega$ statt
$K$ und $do$ statt $dK$).
}\par\medskip

\fpbox{{\bf Massen"anderung in einem dreidimensionalen Gebiet.} 
Man kann die Formel f"ur die Massen"anderung sofort auf drei Dimensionen verallgemeinern.
Sei $\Omega\subset\R^3$ ein Gebiet, $\partial\Omega$ die Oberfl"ache des Gebietes, $n$ die "au"sere Normale auf diese Oberfl"ache, $u(x,t)$ Materiedichte,
und $J(x,t)$ der Massenfluss. Dann finden wir wie oben
$$ 
m(t) = \int_{\Omega} u(x,t)\, dx,\quad
\frac d {dt} m(t) = \int_{\partial \Omega}  n^T\,J\,\,do.$$
Hier ist das Integral "uber $\partial \Omega$ ein Integral "uber eine Oberfl"ache, 
wie auf Seite~\pageref{oberlaecheIntegral} eingef"uhrt.
}

%===============================================================================
\subsubsection{Kurvenintegral "uber eine vektorwertige Funktion}
\index{Kurvenintegral!vektorwertige Funktion}
Nun wollen wir uns eine spezielle Situation ansehen: die Funktion $g$ sei nun nicht skalar, 
sondern ordnet jedem Punkt im Raum einen Vektor zu (ein sogenanntes Vektorfeld), $g:\R^n\rightarrow\R^n$. 
\begin{sdefi} Sei $ K: I\rightarrow \R^n$ eine differenzierbare Kurve, und $\|\dot K(t)\|=1$. 
Sei weiter $g:\R^n\rightarrow\R^n$ ein Vektorfeld.
Dann ist das Kurvenintegral des Vektorfelds "uber die Kurve definiert als
$$ \int_{K} g(K) dK = \int_a^b <g(K(t)), \dot K(t)>\, dt.
$$
Im allgemeinen Fall ($\dot K$ beliebig) ist unser Integral schon
durch normiert, und wir setzen
$$ \int_{K} g(K) dK = \int_a^b <g(K(t)), \dot K(t)>\, dt.
$$
\end{sdefi}
%===============================================================================
\begin{auf}\chb\label{block9A2}
\input{../../Aufgabensammlung/hm720.tex}
\end{auf}

\subsection{Gau"sscher Satz}

Im Umfeld von partiellen Differentialgleichungen spielen mehrere und 
insbesondere
der Gau"ssche  Integralsatz eine Rolle. In der Physik findet man
diese Integrals"atze bei der klassischen Beschreibung von Elektromagnetischen Feldern wieder.\par
%===============================================================================
\fpbox{
Wir m"ochten die wohl bekannte Formel
$$ \int_a^b F'(x)\, dx = F(a)-F(b)$$
auf zwei (oder mehr) Dimensionen verallgemeinern. Letztlich besagt die Formel, dass das Integral
"uber die Ableitung einer Funktion durch die Werte auf dem Rand des Integrationsgebietes ausgedr"uckt werden
kann. Genau so eine Aussage wollen wir in $\R^2$ herleiten.}
%===============================================================================
Gegeben sei ein Gebiet im $\R^2$. Der Einfach halber sehen wir uns ein Quadrat an,
$$\Omega = \{(x,y)\,|\, a < x< b,\quad c<y<d\}.$$

\begin{figure}[htbp] %  figure placement: here, top, bottom, or page
   \centering
   \includegraphics[width=10cm]{../figures/pde3.pdf} 
   \caption{Quadratisches Gebiet und Gau"sscher Satz.}
   \label{pde3}
\end{figure}

%===============================================================================
Der Rand eines Gebietes $\Omega$ wird mit $\partial \Omega$ bezeichnet. Wenn das Gebiet 
in $\R^2$ liegt, bildet $\partial\Omega$ eine Kurve. \par
Weiter betrachten wir ein Vektorfeld, d.h.\ jedem Ort $(x,y)$ wird ein Vektor 
$f(x,y) = (f_1(x,y), f_2(x,y))$ zugeordnet. Nun definieren wir die "au"sere Normale 
$n(x,y)$ auf $\partial\Omega$. D.h., ein Vektor der L"ange eins, der senkrecht 
auf dem Rand des Gebiets liegt, und nach Au"sen zeigt. Wir integrieren nun "uber 
dem Rand $\partial\Omega$ 
des Gebiets das Skalarprodukt $n\, f$ (siehe Abb.~\ref{pde3}), indem wir 
die Kurve in vier Strecken zerlegen,
\begin{eqnarray*}
\int_{\partial\Omega} <n(x,y), f(x,y)> do & :=  &\quad 
\int_a^b \left(\begin{array}{c}0\\1\end{array}\right)\,\, \left(\begin{array}{c}f_1(x,d)\\f_2(x,d)\end{array}\right) dx
+\int_c^d \left(\begin{array}{c}1\\0\end{array}\right)\,\, \left(\begin{array}{c}f_1(b,y)\\f_2(b,y)\end{array}\right) dy\\
&&+\int_a^b \left(\begin{array}{c}0\\-1\end{array}\right)\,\, \left(\begin{array}{c}f_1(x,c)\\f_2(x,c)\end{array}\right) dx
+\int_c^d \left(\begin{array}{c}-1\\0\end{array}\right)\,\, \left(\begin{array}{c}f_1(a,y)\\f_2(a,y)\end{array}\right) dy\\
& = & 
\int_a^b [f_2(x,d)-f_2(x,c)] dx + \int_c^d [f_1(b,y)-f_1(a,y)] dy\\
& = & 
\int_a^b \int_c^d \partial_y f_2(x,y) dy dx + \int_c^d \int_a^b\partial_x f_1(x,y) dx dy\\
&=& \int_\Omega \left(\begin{array}{c}\partial_x\\\partial_y\end{array}\right)\,\, \left(\begin{array}{c}f_1(x,y)\\f_2(x,y)\end{array}\right) dxdy
= \int_\Omega \nabla f(x,y) dx dy
\end{eqnarray*}
(``$do$'' als Symbol f"ur ``Integral "uber die Oberfl"ache von $\Omega$'', d.h.\ da unser Gebiet 
zweidimensional ist, \"uber die Randkurve $\partial\Omega$). 
%===============================================================================
Die Formel klappt nat"urlich f"ur alle Gebiete, die aus der Vereinigung endlich 
vieler Vierecke bestehen. Sogar allgemeinere Gebiete funktionieren, da man diese 
beliebig genau durch die Vereinigung endlich vieler (sehr, sehr vieler) Vierecke 
approximieren kann.\index{Gau"sscher Integralsatz}\index{Divergenztheorem}
\begin{satz} Sei $\Omega$ ein Gebiet, dessen Rand $\partial\Omega$ eine 
differenzierbare Kurve bildet, und $f:\R^2\rightarrow\R^2$ ein 
differenzierbares Vektorfeld. Sei $n(x)\in\R^2$ der "au"sere Normalenvektor 
auf dem Rand des Gebietes.
Dann gilt
$$ \int_{\partial\Omega} n(x)\, f(x) do  = \int_\Omega \nabla f(x) dx.$$
(hier ist $x$ als $(x_1, x_2)\in\R^2$ zu interpretieren). Die Formel gilt auch 
in h"oheren Dimensionen, speziell in $\R^3$. Dann ist $\int_{\partial\Omega}(..)do$ 
das Integral "uber eine zweidimensionale Oberfl"ache. 
\end{satz}

Bemerkung: Man definiert auch $\mbox{div} f= \nabla f$, und nennt den 
Gau"sschen Integralsatz auch Divergenztheorem. 

\subsection{Gesetz "uber die Erhaltung von Masse}
\index{Massenerhaltungsgesetz}
{\bf Erhaltung in einer Dimension}.\par
Sei $u(x,t)$ die Partikeldichte zur Zei $t$ (mit $x\in\R$). Dann ist
$$  \int_a^bu(t,x)\, dx$$
die Partikelmenge im Intervall $[a,b]$ zur Zeit $t$. Wenn Partikel weder erzeugt 
noch vernichtet werden k"onnen, wird sich diese Dichte nur durch einen
Partikelfluss "uber die Intervallgrenzen "andern. Sei
$$ J(x,t) = \mbox{Netto-Partikelfluss von links nach rechts am Ort } x \mbox{ zur Zeit }t.$$
D.h., wir z"ahlen am Ort $x$, wieviel Partikel pro Zeiteinheit 
von links nach rechts laufen, und ziehen davon ab, wieviel Partikel pro Zeiteinheit 
bei $x$ von rechts nach links laufen. Das kann man sich mit folgenden Bild 
verdeutlichen: wir setzen uns mit einem Stuhl an eine Stra"se, und z"ahlen 
f"ur je 
eine Minute die Autos. Allerdings z"ahlen wir nur die von rechts nach links fahrenden 
Autos mit ``+1'', die von links nach rechts fahrenden Autos z"ahlen wir mit ``-1''.

Dann ist die "Anderung der Partikelmenge im Intervall $[a,b]$ gegeben durch den 
Nettofluss hinein bei $a$ und den Nettofluss hinaus bei $b$, d.h.
$$ \int_a^bu_t(t,x)\, dx = \frac d {dt}  \int_a^bu(t,x)\, dx 
  = J(a,t)-J(b,t) = - \int_a^b   J_x(x,t)\, dx.$$
Da $a$ und $b$ beliebig waren, erhalten wir
$$ u_t = -J_x.$$
Die zeitliche Ableitung der Partikeldichte h"angt 
mit der r"aumlichen Ableitung des Partikelflusses zusammen. \par\medskip
%===============================================================================
{\bf Erhaltung in h"oheren Dimensionen}. Nun betrachten wir die Partikeldichte 
$u(t,x)$ f"ur $x\in\R^3$, und den Fluss 
$J(x,t)$, wobei der Fluss jetzt ein Vektor in drei Dimensionen darstellt, 
$J(x,t)\in\R^3$. Die "Anderung der Partikelmenge innerhalb eines dreidimensionalen 
Gebietes $\Omega$ ist gegeben durch den Nettofluss durch die Oberfl"ache $\partial\Omega$  
dieses Gebietes. Nehmen wir ein Oberfl"achenelement. Wenn wir den Fluss dadurch  
berechnen, so m"ussen wir den Fluss senkrecht zu diesem 
Fl"achenelement berechnen
(der Anteil des Flusses der tangential zur Fl"ache ist ver"andert die Partikelmenge 
im Gebiet ja nicht). D.h., wir m"ussen das Skalarprodukt des Flusses mit der "au"seren 
Normale berechnen. Damit erhalten wir
$$ \int_\Omega u_t(x,t)\, dx = - \frac d {dt} \int_\Omega u(x,t)\, dx
= - \int_{\partial\Omega} J(x,t)\, n do = - \int_{\Omega} \nabla J(x,t) dx.$$
Vor dem Volumenintegral m"ussen wir das negative Vorzeichen w"ahlen, da wir die 
"au"sere Normale $n$ nutzen, d.h.\ den Partikelfluss aus dem Volumen hinaus berechnen. 
Beim letzten Schritt haben wir den Gau"sschen Integralsatz genutzt. Da $\Omega$ 
beliebig ist, erhalten wir eine Beziehung zwischen $u_t$ und $\nabla J$.
\par\label{fickschesXXGesetz}
%===============================================================================
\fpbox{Das  Erhaltungsgesetz f"ur Masse (auch erstes Ficksches Gesetz genannt)
lautet $ u_t = - \nabla J(x,t).$}\index{Erstes Ficksches Gesetz}
%===============================================================================
%===============================================================================
\subsection*{Aufgaben}
\begin{auf}\chd\label{block9A3}
\input{../../Aufgabensammlung/hm716.tex}
\end{auf}
\begin{auf}\chc\label{block9A4}
\input{../../Aufgabensammlung/hm717.tex}
\end{auf}
\begin{auf}\chc\label{block9A5}
\input{../../Aufgabensammlung/hm718.tex}
\end{auf}
\begin{auf}\chc\label{block9A6}
\input{../../Aufgabensammlung/hm719.tex}
\end{auf}
