
%% ! T E X  root=hmIIneu.tex
% !TEX root=MA9603.WZW.tex
% !TEX program = pdflatex
% !TEX spellcheck = de_DE

%===============================================================================
\section{Klassen und Typen partieller Differentialgleichungen}
%===============================================================================
\zbox{{\bf Ziele}:
\begin{itemize}
\item Characteristiken-Methode f"ur partielle Differentialgleichungen 1'ter Ordnung anwenden k"onnen 
\item Laplace-Gleichung erkennen, wichtigste Eigenschaften kennen
\item W"armeleitungsgleichung erkennen, wichtigste Eigenschaften kennen
\item Wellengleichung erkennen,  wichtigste Eigenschaften kennen
\end{itemize}}\par
%=============================================================================== 
Wir werden hier (relativ kurz) eine Einf"uhrung in das Gebiet der 
partiellen Differentialgleichngen (PDE's) geben.
Dazu starten wir mit Differenitalgleichungen erster Ordnung (nur erste 
Differentialgleichngen tauchen auf),
um dann zu Differentialgleichungen zweiter Ordnung zu gehen.
%===============================================================================
\subsection{Partielle Differentialgleichungen erster Ordnung}
%===============================================================================
Ziel dieser Sektion ist es, Verhalten und L"osungsmethoden f"ur Differentialgleichungen erster Ordnung zu verstehen. Damit 
k"onnen wir insbesondere auch die Boltzmanngleichung verstehen. 
Diese bildet eine der Grundlagen f"ur das Verst"andnis der Gleichungen der 
Str"omungsmechanik.
%===============================================================================
\subsubsection{Lineare, skalare PDE in einer r"aumliche Dimension}\par
 Dazu betrachten wir zun"achst 
eine Gleichung der Form
$$ h_t(t,x) + a(x)h(t,x)_x =0.$$
%===============================================================================
\fpbox{Diese Gleichung ist linear, da mit den L"osungen $h(t,x)$ und $w(t,x)$ auch
die Funktion 
$$\alpha h(t,x)+\beta w(t,x)$$
eine L"osungen ist ($\alpha,\beta\in\R$, beliebig). Die Koeffizientenfunktion $a(x)$ 
hingegen darf recht beliebig (auch nichtlinear) von $x$ abh"angen.} 


%===============================================================================
\fpbox{ Wir geben nun zus"atzlich eine Anfangsbedienung $h(0,x)=h_0(x)$ vor,
d.h. betrachten 
$$ h_t(t,x) + a(x)h(t,x)_x =0,\qquad h(0,x)=h_0(x).$$
\underline{L"osungsmethode:}\\ Schritt 1: Wir suchen Kurven $x=X(t)$ in der 
Raum-Zeit, "uber denen die L"osung $h(x,t)$ der PDE konstant ist
$$ \forall t\geq 0:\qquad h(t, X(t)) = h(0, X(0)).$$
Die totale Ableitung nach der Zeit der Gleichung $h(t, X(t)) = h(0, X(0))$ ergibt
$$ 0 = \frac d {dt} h(t, X(t)) = h_t + h_x X'(t). $$
Vergleichen wir diese  Gleichungen mit der PDE $ h_t + a(x)h_x =0$, so sehen wir, 
dass
$$ X'(t) = a(X(t)).$$
Dies ist aber eine gew"ohnliche Differentialgleichung (die sog.\ ``Charakteristische 
Gleichung'', nicht zu verwechseln mit dem ``charakteristischen Polynom'  von 
Matrizen und linearen, gew"ohnlichen Differentialgleichungen; die L"osungen dieser 
Differentialgleichungen hei"sen ``Charakteristiken''). Die Charakteristiken 
$X(t;x_0)$ h"angen nat"urlich vom Anfangswert $X(0)=x_0$ ab. \\
Schritt 2: Um $h(x,t)$ zu bestimmen, suchen wir die Kurve $X(t;x_0)$, die
durch $(x,t)$ l"auft, d.h.\ bestimmen $x_0$ aus
$$ x = X(t;x_0).$$
Da wir wissen, dass $h(x,t)$ "uber denk Kurven $(x,t)=(X(t;x_0), t)$ konstant ist, 
so muss $h(x,t) = h_0(x_0)$ sein.
}
%===============================================================================
\begin{figure}[htbp] %  figure placement: here, top, bottom, or page
   \centering
   \includegraphics[width=10cm]{../figures/pde1.pdf} 
   \caption{Charakteristiken f"ur das Beispiel.}
\label{pde1}
\end{figure}

%===============================================================================
\begin{bspX} Gegeben seid die partielle Differentialgleichung $ h_t(t,x) + x h(t,x)_x =0$
mit der Anfangsbedingung $h(0,x) = 5x^2.$ Damit ist $a(x) = x$ und wir 
erhalten als charakteristische Gleichung $  \dot X = X$, 
i.e. (siehe Abb.~\ref{pde1}) $ X(t) = x_0 e^t.$
"Uber diesen Kurven ist die Funktion $u(t,x)$ konstant,
$ h(t, X(t)) = h(0,X(0)) = h(0,x_0) = 5 x_0^2.$
Um $h(t,x)$ zu bestimmen, m"ussen wir also die Kurve $X(t;x_0)$ bestimmen, die 
durch $(t,x)$ l"auft, d.h. 
$ x = X(t;x_0) = x_0e^t$, woraus $x_0 = xe^{-t}$ folgt.
Damit ist 
$$ h(t,x) = h(0, xe^{-t}) = 5(xe^{-t})^2 = 5x^2 e^{-2t}.$$
Wir k"onnen das Ergebnis direkt "uberpr"ufen. F"ur $t=0$ finden wir $h(0,x) = 5x^2$, 
d.h. die Anfangsbedingung wird vn der L"osung korrekt angenommen. 
Wenn wir weiter die Funktion in die Partielle Differentialgleichung einsetzen, 
haben wir
$$ h_t+xh_x 
= \left(5x^2 e^{-2t}\right)_t + x  \left(5x^2 e^{-2t}\right)_x
= (-2)\,\,\left(5x^2 e^{-2t}\right) + 2\,\, x  \left(5x e^{-2t}\right) = 0.$$
Stimmt!
\end{bspX}
\par\bigskip


\subsubsection{Lineare, skalare PDE in drei r"aumlichen Dimensionen}\par
Das gleiche Rezept funktioniert auch bei mehreren Dimensionen:
$$ h_t(t,x,y,z) 
+ a(x,y,z)h(t,x,y,z)_x 
+ b(x,y,z)h(t,x,y,z)_y 
+ c(x,y,z)h(t,x,y,z)_z 
=0.$$
Definieren wir 
$$v(x,y,z)=\left(\begin{array}{c}
a(x,y,z)\\b(x,y,z)\\c(x,y,z)
\end{array}\right)
$$
so k"onnen wir die Gleichung schreiben als
$$ h_t + v\nabla h=0.$$

Suchen wir wieder Kurven $K(t) = (X(t), Y(t), Z(t))$ "uber die die L"osung konstant 
ist, so finden wir
$$ 0 = \frac d {dt} h(t,X(t), Y(t), Z(t)) = h_t+h_x\dot X+h_y \dot Y+h_z \dot Z = h_t + \dot K(t)\nabla h.$$
Koeffizientenvergleich mit der partiellen Differentialgleichung f"uhrt auf
$$ \dot K = v.$$

%===============================================================================
\subsubsection{ Altersstrukturierte Populationen}\par
{\it Die\footnote{Wird in der Vorlesung nicht behandelt} Vorhersage der Entwicklung der Gr"o"se einer (humanen) Bev"olkerung erfolgt 
zweckdienlicherweise mittels alterstrukturierten Modelle. Dabei soll die
zeitliche Entwicklung der Funktion $u(t,a)$ berechnet werden, wobei
$$ \int_{a_1}^{a_2} u(t,a)\, da = \mbox{Zahl der Personen mit Alter zwischen }
a_1 \mbox{ und } a_2 \mbox{ zur Zeit } t.$$
Die  Entwicklung in der Zeit wird durch drei Effekte bestimmt: 
Altern (wird in die Struktur der Gleichung eingearbeitet), 
Mortalit"at und Geburten.\par
\underline{Mortalit"at:} 
Die Zahl der Todesf"alle einem kleinen Zeitintervall $\Delta$ 
(ein Jahr, ein Monat), wird proportional von Personen in 
einem gegebenen Alter $a$, wird proportional zu der L"ange des Zeitintervalls,
und der Zahl der Personen dieses Alters $u(a,t)$ sein. Die Proportionalit"atskonstante h"angt sicherlich vom Alter ab. Wir nennen diese 
 $\mu(a)$, und haben
$$ \Delta u(t,a)\mu(a)\approx
\mbox{Todesf"alle im Zeitintervall } [t,t+\Delta] 
\mbox{ im Alter } a.$$
$\mu(a)$ hei"st Mortalit"at oder auch Sterberate.\par
\underline{Geburt:} Wir w"ahlen uns zuf"allig eine Person 
aus der Population, die das Alter $a$ besitzt. Die Wahrscheinlichkeit, 
dass diese Person in den n"achsten $\Delta$ Monaten ($\Delta$ klein) 
ein Kind bekommt sei h"angt sicher von $a$ ab, und sei $b(a)$.
Bei gegebener Populationsstruktur $u(a,t)$ m"ussen wir $b(a)$ mit
$u(t,a)$ multiplizieren (wie viele Personen sind wie im Moment wie fertil?) und
"uber $a$ integrieren, um die Gesamtzahl der Geburten zu ermitteln,
$$ \int_0^\infty b(a)u(t,a)\, da = \mbox{ Zahl der Geburten zur Zeit }t.$$
\underline{Gleichung } 
Wenn wir grob wissen wollen, wieviel Personen in einem Jahr mit Alter
$a+1$ leben, so finden wir (unter Vernachl"assigung der Todesf"alle) gerade die
Zahl der Personen des Alters $a$, die heute leben. Allgemeiner, zur Zeit 
$t+\Delta$ und Alter $a+\Delta$ leben '(ohne Todesf"alle) genauso viele 
Individuen wie heute mit Alter $a$. Bei Ber"ucksichtigung der Todesf"alle m"ussen wir
$\mu(a) u(t,a)\Delta$ Individuen noch abziehen, d.h.
\begin{eqnarray*}
u(t+\Delta, a+\Delta) &= &  u(t,a) - \mu(a)\Delta u(t,a)\\
\quad\Rightarrow\quad
\frac {u(t+\Delta, a+\Delta)-u(t, a+\Delta) } { \Delta }
+\frac{u(t, a+\Delta)- u(t,a)}{\Delta} & = & - \mu(a) u(t,a).
\end{eqnarray*}
Der zweite Term auf der linken Seite ist der Differenzenquotient in 
Hinblick auf $a$. Wenn $\Delta\rightarrow 0$ geht dieser Term gegen $u_a(a,t)$.
Der erste Term ist analog der Differenzenquotienten in Hinblick auf $t$ 
(man beachte, dass das zweite Argument in diesem Term immer $a+\Delta a$
ist). Also finden wir f"ur $\Delta\rightarrow 0$,
$$ u_t(t,a) + u_a(t,a) = -\mu(a) u(t,a).$$
Wir m"ussen noch die Geburten angeben, d.h.\ die Zahl der Individuen 
mit Alter $a=0$, 
$$ u(t,0) = \int_0^\infty b(a)\, u(t,a)\, da.$$
Die letzte Information, die wir noch ben"otigen ist Anfangsbedingung (Populationsgr"o"se $u_0(a)$ zur Zeit $t=0$). Das altersstrukturierte 
Modell lautet
\begin{eqnarray*}
u_t(t,a) + u_a(t,a) & = & -\mu(a) u(t,a)\\
u(t,0) & = &\int_0^\infty b(a)\, u(t,a)\, da,\qquad u(0,a)=u_0(a).
\end{eqnarray*}
Damit k"onnen wir im Prinzip mittel- und langfristige Prognosen der Bev"olkerungsentwicklung berechnen.
}

%===============================================================================
\subsubsection{ Boltzmann Gleichung}\par
Zun"achst kehren wir zu der eingangs behandelten Gleichung zur"uck.\par
\fpbox{{\bf Ausbreitung ohne Wechselwirklung.} Wenn die ``Teilchendichte'' $h(x,t)$ (mit $x\in\R^3$) 
sich mit Geschwindigkeit $v$ ausbreitet, so folgen einzelne Teilchen 
der Bahn, die gegeben ist durch $\dot x = v$. Daher breitet sich die Dichte 
gem"a"s
$$ h_t(x,t) + \nabla (v h(x,t)) = 0$$
aus (diese Gleichung besitzt die richtigen charakteristischen Differentialgleichungen). \\
 Nun k"onnen wir nat"urlich auch Teilchen mit 
mehreren, verschiedenen Geschwindigkeiten
im System haben. F"ur jede Geschwindigkeit $v$ geben wir $h(t,x,v)$ die Teilchendichte 
am Ort $x$ mit
Geschwindigkeit $v$ zur Zeit $t$ an. Wenn die Teilchen nicht wechselwirken, so 
gen"ugt (f"ur jedes feste $v$) die Funktion der Gleichung
$$ h_t(t,x,v) + v\, \nabla h(t,x,v) = 0.$$
Wir k"onnen also unendlich viele Gleichungen (f"ur die verschiedenen Geschwindigkeiten) 
in einer Gleichung zusammenfassen. }\par\medskip
Nun stimmt es nicht, dass (Gas-)Molek"ule/Teilchen ohne Wechselwirkung nebeneinander 
herfliegen: sie sto"sen
zusammen und interagieren. Wenn zwei Teilchen (an einem Ort) zusammenstossen, so 
werden sie i.a.\ ihre Geschwindigkeit "andern. Es verschwinden also Teilchen mit
einer Geschwindigkeit, und tauchen mit einer anderen Geschwindigkeit wieder auf. 
Man kann diesen Term schreiben als \par
\fpbox{
$$ h_t(t,x,v) + v\, \nabla h(t,x,v) = Q(h(t,x,.));$$
diese Gleichung hei"st Boltzmann-Gleichung.\index{Boltzmann-Gleichung}
}\par\medskip
Dabei wollen wir den Sto"soperator $Q(.)$ nicht n"aher bestimmen. Wichtig ist, dass 
er r"aumlich (und zeitlich) lokal wirkt. Die Physik fordert drei Erhaltungsgr"o"sen: 
Teilchendichte, Impulsdichte und Energiedichte. Welche Eigenschaft verlangen wir 
von $Q(.)$, sodass diese drei Gr"o"sen erhalten bleiben?

Wir verlangen f"ur die Massenerhaltung
 $$ \int Q(h,h)\, dv = 0$$
f"ur die Impulserhaltung
$$ \int Q(h,h)\, v\, dv = 0$$
und f"ur die Energieerhaltung
$$ \int Q(h,h)\, (v-u)^2\, dv = 0$$
(wobei $u=\int v h\, dv$). Warum das die richtigen Gleichungen sind, 
diskutieren wir hier nicht (aber im Anhang).

%===============================================================================
\begin{auf}\chc\label{block10A1}
\input{../../Aufgabensammlung/hm722.tex}
\end{auf}
\begin{auf}\chc\label{block10A2}
\input{../../Aufgabensammlung/hm723.tex}
\end{auf}
\begin{auf}\chd\label{block10A2a}
\input{../../Aufgabensammlung/hm760.tex}
\end{auf}
%===============================================================================
\subsection{Standard-Gleichungen zweiter Ordnung}
\subsubsection{Typen partieller Differentialgleichung zweiter Ordnung}

Partielle Differentialgleichungen zweiter Ordnung sind Differentialgleichungen in denen mindestens eine zweite Ableitung vorkommt. In zwei Dimensionen haben wir 
$$ a u_{xx}+ bu_{xy}+c u_{yy} + du_x+e u_y+f u = 0.$$

Man rechnet leicht nach, dass (falls $a,b,c$ reelle Konstanten sind)
$$ \nabla^T 
\underbrace{\left(\begin{array}{cc}
a & b/2\\
b/2 & c
\end{array}\right)}_{=: A}\nabla u
= a u_{xx}+ bu_{xy}+c u_{yy}.
$$
Die Matrix $A$ bestimmt also die Terme zweiter Ordnung v"ollig. Man kann 
anhand von $A$ eine Typeneinteilung vornehmen. Da $A$ eine (symmetrische) Matrix ist (Symmetrie impliziert zwei reelle Eigenwerte, siehe Seite~\pageref{symmEV}), und
Matrizen weitgehend durch ihre Eigenwerte bestimmt sind, liegt es nahe, bei der Typeneinteilung
"uber die Eigenwerte zu gehen.\par
Diese Einteilung f"allt hier vom Himmel. Wir werden sie angeben, und aus jedem
Typ eine prototypische Gleichung etwas diskutieren. Einerseits spielen diese 
Prototypen in den Anwendungen (Physik) eine zentrale Rolle, andererseits 
wird es klar, dass diese Gleichungen v"ollig verschiedene Charaktere haben.\par
\fpbox{
Fall 1: Zwei Eigenwerte mit gleichem Vorzeichen - elliptische Gleichungen\\
\vspace*{-0.9cm}\\
\noindent\rule{\textwidth}{1pt}
Einfachstes Beispiel:
$$ A = 
\left(\begin{array}{cc}
1 & 0\\
0 & 1
\end{array}\right)$$
(Eigenwerte sind beide $1$). Prototyp ist die {\bf Laplace-Gleichung}
$$ \Delta u = u_{xx}+u_{yy} = 0.$$
Fall 2: Zwei Eigenwerte mit unterschiedlichem Vorzeichen - hyperbolische Gleichungen\\
\vspace*{-0.9cm}\\
\noindent\rule{\textwidth}{1pt}
Einfachstes Beispiel:
$$ A = 
\left(\begin{array}{cc}
1 & 0\\
0 & -1
\end{array}\right)$$
(Eigenwerte sind $\pm 1$). Prototyp ist die {\bf Wellengleichung}
$$  u_{tt}-u_{xx} = 0.$$
(wir schreiben hier $t$ statt $y$, da diese Koordinate in der Wellengleichung
tats"achlich die Rolle der Zeit spielt.\par\medskip
Fall 3: Ein Eigenwerte Null einer ungleich Null - parabolische Gleichungen\\
\vspace*{-0.9cm}\\
\noindent\rule{\textwidth}{1pt}
Einfachstes Beispiel:
$$ A = 
\left(\begin{array}{cc}
1 & 0\\
0 & 0
\end{array}\right)$$
(Eigenwerte $1$ und $0$). Prototyp ist die {\bf W"armeleitungs- oder Diffusionsgleichung}.
$$ u_t =  u_{xx}.$$
(wir haben nur zweite Ableitungen in $x$, und nehmen noch eine 
erste Ableitung hinzu; die entsprechende Variable "ubernimmt wieder die Rolle der Zeit, daher schreiben wir $u_t$ statt $u_y$).
}
\par\medskip

Im dreidimensionalen Raum haben wir statt einer zweidimensionalen Matrix eine 
dreidimensionale Matrix mit drei Eigenwerten; entsprechend mehr Typen gibt es.
Aber auch in h"oheren Dimensionen sind diese drei Standard-Typen (und diese 
drei standard-Gleichungen) die aller- allerwichtigsten.

%===============================================================================
\subsubsection{Wellengleichung}
Betrachte
$$ h_{tt}-c^2h_{xx} = 0.$$
Diese Gleichung beschreibt die Ausbreitung einer Welle mit Geschwindigkeit $c$ 
(im eindimensionalen Raum). Wie kann man L"osungen finden? Nun, wir bemerken, dass allgemein
$$
[ (\partial_t+c\partial_x) (\partial_t-c\partial_x) ]h
= 
[\partial_t^2-c\partial_t \partial_x+c\partial_x \partial_t - c^2 \partial_x^2]h
=  h_{tt}-c^2h_{xx} = 0.
$$
Also insbesondere
$$
[ (\partial_t+c\partial_x) (\partial_t-c\partial_x) ]h
= 0 = 
[ (\partial_t-c\partial_x) (\partial_t+c\partial_x) ]h.
$$
Daher sind  L"osungen der Gleichungen
$$ 
(\partial_t-c\partial_x) ]h_- = 0,\qquad
(\partial_t+c\partial_x) ]h_+ = 0.
$$
auch L"osungen der Wellengleichung. Diese beiden Gleichungen sind aber 
f"ur uns alte Bekannte: die L"osungen k"onnen geschrieben werden als
$$ h_-(t,x) = H_-(x+ct),\quad h_+(t,x) = H_+(x+ct).$$
Alles in allem haben wir die L"osung
$$ h(x,t) = H_-(x-ct)+H_+(x+ct).$$
Die L"osung wird eindeutig bestimmt durch die Angabe von
$$ h(0,x) = h_0(x),\qquad h_x(0,x) = h_1(x).$$
Der Anteil der L"osung, den wir hier bestimmt haben, 
behandelt nur $h_0(x)$. der andere Anteil $h_1(x)$ verlangt einen
weiteren Term (siehe Aufgabe~\ref{block10A4}).  
{\bf Wichtig: } Die Wellengleichung beschreibt die Ausbreitung von Teilchen 
(Energie, Schall...) mit einer festen Geschwindigkeit.
\begin{sdefi}
Die Wellengleichung in drei r"aumliche Dimensionen ist gegeben durch
$$ h_{tt} - c^2\Delta h = 0.$$
\end{sdefi}
Erinnerung: $\Delta h$ bedeutet die Summe der zweiten, r"aumlichen Ableitungen
$\Delta h = h_{xx}+h_{yy}+\cdots$, und hei"st Laplace-Operator.\index{Laplace-Operator}

%===============================================================================
\subsubsection{W"armeleitungsgleichung}

Im Gegensatz zur Wellengleichung, in der die Partikel eine feste Geschwindigkeit 
besitzen, beschreibt die W"armeleitungsgleichung eine ungeordnete (Brownsche) 
Bewegung. Wir werden die W"armeleitungsgleichung 
zwei mal mit unterschiedlichen Argumenten herleiten.\par\medskip

{\bf Herleitung 1: Ungeordnete Bewegung.}\par
Als Modell betrachten wir zun"achst eine lange Reihe von Trittsteinen. 
Ein Kind h"upft von Trittstein zu Trittstein, nach rechts mit Wahrscheinlichkeit 
1/2, und links ebenfalls mit Wahrscheinlichkeit 1/2. 
Sei $p(m,n)$ die Wahrscheinlichkeit nach $n$ Zeitschritten das Kind auf 
Trittstein $m\in\Z$ zu finden, wobei bei Zeit $n=0$ das Kind auf Trittstein $m=0$ 
startete. \par
Man kann $p(m,n)$ tats"achlich explizit ausrechnen via 
$$ p(m,n) = \frac 1 2 p(m-1,n-1) + \frac 1 2 p(m+1,n-1);
$$
allerdings wollen wir $p(m,n)$ approximieren. 


%===============================================================================
\begin{figure}[htbp] %  figure placement: here, top, bottom, or page
   \centering
   \includegraphics[width=10cm]{../figures/p1_ssd1.pdf} 
   \caption{(a) Diskrete Zufallsbewegung. (b) Zeitentwicklung der diskreten Zufallsbewegung.}
\label{pde2}
\end{figure}

%===============================================================================


Dazu fahren wir Skalierung  f"ur den Ort $x_m = m\Delta x$, $m\in\Z$  und der
Zeit  
zur Zeit $t_n=n\,\Delta t$ ein; diese Wahl ist schon ein Hinweis darauf, dass
wir $\Delta t$ und $\Delta h$ gegen Null gehen lassen wollen. Dabei werden wir
verlangen, dass
$$  \Delta x\rightarrow 0,\qquad
    \Delta t\rightarrow 0,\qquad
    \frac{\Delta x^2}{2\Delta t} = D.
$$
Wir vermuten nun, dass die Wahrscheinlichkeiten 
 $p(m,,n)$ eine Funktion $u(x,t)$ approximieren, und zwar via
$$ u(x,t ) = p(x/\Delta x, t/\Delta t).$$
Diese Annahme ergibt
\begin{eqnarray*}
u(x,t) 
& = & \frac 1 2 
\left( u(x+\Delta x, t-\Delta t) + u(x-\Delta x, t-\Delta t) \right)\\
\Rightarrow\quad
\frac{u(x,t)-u(x,t-\Delta t)}{\Delta t}& = & \frac {\Delta x^2}{2\Delta t}\,\, 
\frac{u(x+\Delta x, t-\Delta t)- 2u(x,t-\Delta t)+ u(x-\Delta x, t-\Delta t)}
{\Delta x^2}\\
&=&
 \frac {\Delta x^2}{2\Delta t}\,\, 
\frac{\frac{u(x+\Delta x, t-\Delta t)- u(x,t-\Delta t)}{\Delta x}-\frac{u(x,t-\Delta t)- u(x-\Delta x, t-\Delta t)}{\Delta x}}
{\Delta x}\\
 & \downarrow & \quad_{ (\Delta t,\Delta x \rightarrow 0,\quad 
                 \Delta x^2/(2\Delta t) = D)}\\
u_t & = & D\, u_{xx}
\end{eqnarray*}
Diese Gleichung ist die W"armeleitungsgleichung in einer Dimension.
\par\medskip
%===============================================================================
{\bf Herleitung II: Erhaltungsgesetz}\par\medskip
Eine Bewegung wird die Masse nicht ver"andern, d.h.\ das Erhaltungsgesetz 
(1.'tes Fick'sches Gesetz)
$$ u_t = - \nabla J$$
ist g"ultig. Das zweite Fick'sche Gesetz lautet nun, dass bei Diffusion bzw.\ 
W"armeleitung die Partikel in Richtung des negativen Gradienten laufen, d.h.\ 
der Fluss proportional (mit Proportionalit"atskonstante $D$) dem negativen 
Gradienten von $u$ ist,
$$ J = - D \nabla u.$$
Kombinieren wir erstes und zweites Fick'sches Gesetz, so finden wir
$$ u_t = \nabla D\nabla u = D\Delta u.$$

%===============================================================================
\begin{sdefi}
 Die W"armeleitungsgleichung in drei r"aumliche Dimensionen ist gegeben durch
$$ u_{t} = D \Delta u = 0.$$
\end{sdefi}
%===============================================================================

\begin{sbem} Eine Gau"s-Verteilung mit Varianz $\sqrt{2 D t}$ 
$$ u(x,t) = \frac 1 {2\sqrt{\pi D t}}e^{-x^2/(4Dt)}.$$
erf"ullt die W"armeleitungsgleichung. D.h., wenn wir zur Zeit $t=0$ mit einem Partikel 
im Ort $x=0$ starten, so ist dessen Aufenthaltswahrscheinlichkeit 
durch eine Gau"s-Verteilung mit in der Zeit wie eine Wurzel steigenden 
Varianz gegeben. 
\end{sbem}

%===============================================================================

{\bf Merke: } Die W"armeleitungsgleichung beschreibt eine Zufallsbewegung. 
Die Ausbreitungsgeschwindigkeit dieser Teilchen sind (mit sehr kleiner Wahrscheinlichkeit)
beliebig schnell.

\par\bigskip
%===============================================================================
{\bf Anwendung: Bestimmung der Diffusionsrate eines Schadstoffs im Boden}\par\medskip\label{difMod}
Versuch: Schadstoff wurde radioaktiv markiert, 
Ackerboden wurde mit einem Schadstoff versetzt, die linke H"alfte von 
Test-R"ohrchen mit verseuchten Boden, die rechte H"alfte mit unverseuchten 
Boden versetzt. An vier Zeitpunkten (einigen Wochen) wurde die Verteilung der
Radioaktivit"at gemessen. Die Diffusionsrate soll bestimmt werden.\par\medskip
Modell: $u(x,t)$ Dichte des Schadstoffs (Radioaktivi"at) im Boden, $x\in [0,L]$ 
(L"ange der R"ohrchen ist $L$). Diffusionsgleichung:
$$u_t = D u_{xx},
\qquad u(x,0) = 0 \mbox{ f"ur } x<L/2,\quad
\qquad u(x,0) = u_0 \mbox{ f"ur } x>L/2$$
wobei $u_0$ die Anfangskonzentration des Schadstoffs ist. Wir ben"otigen 
 Informationen "uber die beiden "offnungen des R"ohrchens, sogenannte 
 Randbedingungen.

\fpbox{ Bei Endlichen Gebieten muss man -- zus"atzlich zur Anfangsbedingung -- auch Randbedingungen angeben. Experimente k"onnen die R"ander (``"offnungen der R"ohrchen'') v"ollig unterschiedlich behandeln, entsprechend unterschiedlich 
sind die Randbedingungen:\\
{\bf (a)} Die Enden der R"ohrchen werden mit L"osungsmittel verbunden, das immer wieder
ausgetauscht wird. Damit setzen (und halten) wir die Konzentration an den
Enden der R"ohrchen auf Null ({\bf homogene Dirichlet-Randbedingungen})
$$ u(0,t)=u(L,t)=0.$$
{\bf (b)} Die Enden des R"ohrchens werden mit einem Reservoir des Schadstoffs verbunden, das eine gegebene Schadstoffkonzentration $\hat u_0$ aufweist
 ({\bf inhomogene Dirichlet-Randbedingungen})
$$ u(0,t)=u(L,t)=\hat u_0.$$
{\bf (c)} Nichts flie"st in das R"ohrchen hinein, nichts hinaus (hier kontrollieren wir nicht die Konzentration direkt wie in (a), (b), sondern den Fluss $j=-Du_x$) 
 ({\bf homogene Neumann-Randbedingungen})
$$ - Du_x(0,t) = -Du_x(L,t)=0
\quad\Leftrightarrow\quad
u_x (0,t) = u_x(L,t)=0.$$
{\bf (d)} Ein definierter Fluss in das R"ohrchen / aus dem R"ohrchen wird vorgegeben (Pumpen anschlie"sen?) 
 ({\bf inhomogene Neumann-Randbedingungen})
$$ - Du_x(0,t) = a,\qquad -Du_x(L,t)=b.$$
}
\par\medskip

%===============================================================================

%\begin{figure}[htbp] %  figure placement: here, top, bottom, or page
%   \centering
%   \includegraphics[width=12cm]{../figures/diffu_SImul} 
%   \caption{Vergleich unserer L"osung der Diffusionsgleichung von %Schadstoff 
%in der R"ohre (gestrichelte Linie) mit einer anderen numerischen %Methode.}
%\label{difSim}
%\end{figure}

%===============================================================================


In unserem Fall ging nichts in die R"ohrchen hinein und nichts hinaus, d.h.
wir w"ahlen homogene Neumann-Randbedingungen.\par
Man kann die  L"osung der Differentialgleichung z.B.\ numerisch bestimmen, und (bei adaptiertem $D$)
mit Daten vergleichen (Abb.~\ref{difDat}). Die "ubereinstimmung ist sehr sch"on.


\begin{figure}[htbp] %  figure placement: here, top, bottom, or page
   \centering
   \includegraphics[width=8cm]{../figures/diffu_data} 
   \caption{Diffusionsgleichung, Kapitel~\ref{difMod}. Vergleich der L"osung ($D$ adaptiert) (Kreise) mit Daten 
(Punkte) zu verschiedenen Zeitpunkten. }
\label{difDat}
\end{figure}

\kommentar{
Weiter stellen wir die Anfangsbedingung als Summe von Kosinusfunktionen da
(dass dies m"oglich ist, ist eine Konsequenz der Fourier-Transformation),
$$ u_0(x) = \sum_{n=0}^\infty C_n\cos(n\,\pi,x/L),\quad
C_n = \frac 2{\pi n}\, [\sin(n\pi)-\sin(n\pi/2)]\in\R.$$
{\bf Zusammenfassung des Modells:}
\begin{eqnarray*}
u_t & =& D u_{xx},\qquad u_x (0,t) = u_x(L,t)=0\\
u_0(x) & =& \sum_{n=0}^\infty C_n\cos(n\,\pi,x/L).
\end{eqnarray*}

{\bf Konstruktion einer L"osung:}\par
Wir versuchen (im Sinne der Konstruktion eines Fundamentalsystems) zun"achst soviele L"osungen wie m"oglich zu finden. Ein Separationsansatz, 
$u(x,t) = g(g) h(t)$ f"uhrt auf 
($u_t(t,x)=h'(t)g(x)$, $u_{xx}(t,x)=h(t)g''(x)$)
$$ h'(t) g(x) = D h(t) g''(x),\qquad h(t)g'(0)=h(t)g'(L)=0.$$
Insbesondere folgt $h'(t)/(D h(t)) = g''(x)/g(x)$. Wir d"urfen also insbesondere
$t=0$ w"ahlen, und finden mit $\lambda=h'(0)/(D h(0))$ (eine Konstante!)
$$ \frac{g''(x)}{g(x)}=\lambda,
\quad\Leftrightarrow\quad
g'' = \lambda g,\qquad g'(0)=g'(L)=0.$$
Diese Gleichung ist prinzipiell eine gew"ohnliche Differentialgleichung 
(wobei wir $\lambda$ nicht kennen, aber Zusatzbedingungen $g'(0)=g'(L)=0$
haben). Falls $\lambda>0$ sind die L"osungen $C\, e^{\sqrt{\lambda} x}$
und $C\, e^{-\sqrt{\lambda} x}$. Beide L"osungen k"onnen die Randbedingungen
nicht erf"ullen. Also ist $\lambda$ negativ oder Null, wir setzen 
$\lambda = -\omega^2$, und finden die L"osungen $C \sin(\omega x)$, 
 $C\cos(\omega x)$. Da die Ableitung des Sinus bei $x=0$ nicht Null ist, kann
auch $\sin\omega x)$ die Zusatzbedingungen nicht erf"ullen. Bleibt
$\cos(\omega x)$. Hier erzwingen die Zusatzbedingungen spezielle 
Werte f"ur $\omega$,
$$ \frac d{dx}\cos(\omega x) = -\omega\sin(\omega x),
\quad
\frac d{dx}\cos(\omega x)|_{x=0} = 0,
\quad
\frac d{dx}\cos(\omega x)|_{x=L} = -\omega\sin(\omega L) = 0
$$
und also
$$ \sin(\omega L)=0\quad\Rightarrow\quad
\omega = n\,\pi/L,\quad n\in\N_0.$$


Wir haben also eine Folge von L"osungen f"ur $g(x)$, gegeben durch 
$g_n(x) =  \cos(n\pi x/L)$.\par
Da wir nun $\lambda=-\omega^2=-(n\pi/L)^2$ kennen, finden wir wegen $f'=\lambda f$, dass $f(t) = \exp(-(n\pi/L)^2 t)$, und unendlich viele
L"osungen der partiellen Differentialgleichungen (mit Randwerten)
$$ u_n(x,t) = e^{-(n\pi/L)^2 t}\,\,  \cos(n\pi x/L).$$
Da die Gleichung linear ist, d"urfen wir $u_n(x,t)$ mit reellen Konstanten 
$c_n$ multiplizieren und aufaddieren, und erhalten wieder eine L"osung 
$u=\sum c_n u_n$ (falls
die Reihe konvergiert). 
In Anbetracht der Anfangsbedingung w"ahlen wir $c_n=Cn$, und finden
$$ u(x,t) = \sum_{n=0}^\infty C_n e^{-(n\pi/L)^2 t}\,\,  \cos(n\pi x/L).$$
Diese Funktion kann man nun numerisch berechnen (summiere nur bis 
endliches $n$, z.B.\ $n=16$). Abbildung~\ref{difSim} zeigt den Vergeich
unserer Methode mit einer anderen numerischen Methode (die wir hier nicht besprechen wollen). 
Man kann die theoretische L"osung (mit adaptiertem $D$)
mit Daten vergleichen (Abb.~\ref{difDat}). Die "ubereinstimmung ist sehr sch"on.}

%===============================================================================
\subsubsection{Laplace-Gleichung}
%===============================================================================
Die Laplace-Gleichung beschreibt die station"aren Punkte der W"armeleitungsgleichung.
Also
$$ \Delta u = 0.$$
Wie sehen typische L"osungen aus?\par
{\bf Eine Dimension:}\par
Hier lautet die Gleichung $u'' = 0$, d.h. $u(x) = a+b x$.\par\medskip
{\bf Zwei Dimensionen:}\par
Wir bemerken hier nur (ohne Beweis),
dass wir eine L"osung erhalten, 
wenn wir eine differenzierbare Funktion nehmen, in die
Funktion eine komplexe Zahl einsetzen, und den Real- oder Imagin"arteil dieser 
Funktion betrachten.
Also
$$ 
u(x,y) = \Re(h(x+{\rm i} y))
\qquad\mbox{ oder }\qquad
u(x,y) = \Im(h(x+{\rm i} y)).
$$
\begin{bspX}
$h(x) = x^2$, und
$$ u(x,y) = \Re( (x+{\rm i}y)^2 ) = \Re (x^2-y^2+2{\rm i}xy) = x^2-y^2.$$
Setzen wir ein, so finden wir
$$ (x^2-y^2)_{xx}+(x^2-y^2)_{yy} = 2-2 = 0.$$
Stimmt!
\end{bspX}

Die L"osung der Laplace-Gleichung "ahnelt der Form eine Seifenhaut, die 
in einen Drahtrahmen eingespannt wird (die pr"azise Gleichung f"ur Seifenh"aute 
beinhaltet noch eine Nichtlinearit"at, die man aber in vielen F"ullen vernachl"assigen 
darf).
%===============================================================================
\subsection*{Aufgaben}
\begin{auf}\chc\label{block10A3}
\input{../../Aufgabensammlung/hm724.tex}
\end{auf}

\begin{auf}\che\label{block10A4}
\input{../../Aufgabensammlung/hm725.tex}
\end{auf}

\begin{auf}\chb\label{block10A5}
\input{../../Aufgabensammlung/hm076.tex}
\end{auf}

\begin{auf}\che\label{block10A6}
\input{../../Aufgabensammlung/hm726.tex}
\end{auf}